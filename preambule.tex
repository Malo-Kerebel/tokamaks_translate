% Packages utiles
\usepackage[utf8]{inputenc}
\usepackage[T2A,T1]{fontenc}
\usepackage[french]{babel}
\usepackage{amsmath}
\usepackage{ amssymb }
\usepackage{titlesec}
\usepackage{tocloft}
\usepackage{multicol}
\usepackage{graphicx}
\usepackage{xcolor}
\usepackage{lipsum}
\usepackage{float}
\usepackage{bm}

\usepackage{hyperref}

\usepackage{ mathrsfs }

\usepackage{standalone}

%SI
\usepackage{siunitx}
\sisetup{locale = FR} % Pour adapter les séparateurs décimaux aux conventions françaises


% TIKZ
\usepackage{tikz}
\usetikzlibrary{shapes.geometric}
\usetikzlibrary{tikzmark,arrows.meta,calc}

% Configuration de la mise en page
\usepackage[
    inner=5cm,   % Marge intérieure (côté reliure)
    outer=2cm,     % Marge extérieure
    top=2cm,       % Marge supérieure
    bottom=2cm,    % Marge inférieure
    twoside,       % Active le mode "double page" (paires/impaires)
    bindingoffset=0.5cm % Décalage pour la reliure (optionnel)
]{geometry}

% Captions et placement des figures
\usepackage[font=small,labelfont=bf,width=0.5\textwidth]{caption}
\usepackage{wrapfig}
\usepackage{subcaption}
\usepackage[bottom]{footmisc}

\usepackage{placeins}

% Style de numérotation
\numberwithin{equation}{section}
\numberwithin{figure}{section}
\renewcommand{\theequation}{\thesection.\arabic{equation}}
\renewcommand{\thefigure}{\thesection.\arabic{figure}}

\let\cleardoublepage\clearpage

% Macros

\newcommand{\dd}{\text{d}}
\newcommand{\sigmav}{\ensuremath{\left\langle \sigma v \right\rangle}}

