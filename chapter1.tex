\documentclass[main.tex]{subfiles}

\begin{document}

\chapter{Fusion}

\section{Fusion and tokamaks}
\label{sec:fusion_et_tokamaks}

Si un noyau de deutérium fusionne avec un noyau de tritium, une parti cule \( \alpha \) est produite et un neutron est relaché.
Le réarrangement nucléaire  induit une réduction de la masse totale et, par conséquent, de l'énergie est relaché sous la forme d'énergie cinétique des produits de la réaction.
L'énergie relachée par chaque réaction est de \( \SI{17.6}{\mega\electronvolt} \).
En terme macroscopique, un seul \( \SI{}{\kilogram} \) de ce carburant relacherais \( \SI{e8}{\kWh} \) d'énergie et serait ce qu'une centrale de \( \SI{1}{\GW} \) (éllectrique) consommerait en un jour.

Le deutérium est une ressource abondante mais le tritium n'apparait pas naturellement.Il devrait, néanmoins, être possoble d'utiliser les neutrons relachés par la réaction de fusion pour produire du tritum à partir de lithium, dont de larges réserves existent.

Pour déclencher la réaction de fusion entre des noyaux de deutérium et de tritium, il est nécessaire de surmonter la répulsion mutuelle due à leurs charges positives. En conséquence, la section efficace de fusion est faible aux basses énergies.
Néanmoins, la section efficace augmente avec l'énergie, et atteint un maximumà \( \SI{100}{\kilo\electronvolt} \), et une balance énergétique positive est possible s'il est possible de faire réagir les particules du carburant avant qu'elles ne perdent leur énergie.
Pour obtenir ce résultat, les particules doivent garder ler énergie et rester dans la région de réaction pour un temps suffisament long.
Plus exactement, le produit de ce temps et la densité de particules qui réagissent doit être suffisamment grand.

La approches simples consistant à diriger un faisceau de particules vers une cible solide ou à travers un autre faisceau ne permettent pas de satisfaire ce critère.
Dans le premier cas, les particules perdent leur énergie trop rapidement et dans le second la densité est trop faible.

La méthode la plus prometteuse pour fournir l'énergie est de chauffer le carburant de deutérium-tritium à une température suffisante pour que la vélocité thermique des noyaux soit assez haute pour produire les réactions requisent.
La fusion ammené par cette méthodfe est appelé fusion thermonucléaire.
La température optimale n'est pas aussi élevé que celle correspondant à l'énergie de la section efficace maximale parce que les réactions requisent  se produisent dabs ka queue de haute énergie de la distribution Maxwellienne des particules chauffées.
La température nécessaire est autour de \( \SI{10}{\kilo\electronvolt} \), ou environ \( 100 \) million de degrés celcius.
À de telles températures, le carburant est entiérement ionisé.
La charge électrostatique des ions nucléaires est neutralisé par la présence d'un nombre égal d'électrons et le gaz neutre résultant de cela est appelé un plasma.

Puisque de telles températures empechent un confinement par des murs, une autre méthode est nécessaire.
Le tokamak est une méthode possible.
Dans un tokamak, les particules du plasmas sont confinés dans une région toroïdale par un champ magnétique, elles sont tenues par le champ dans de petites orbites de giration.
Bien que les températures nécessaires, la densité et le temps de confinement ont tous été obtenus dans des tokamaks, ils n'ont pas été obtenus dans le même plasma.
Cependant, le progrès vers cet objectif a été remarquable , et une puissance thermonucléaire de plus de quatre-bingt-dix pourcent de la puissance injectée a été produite.
Une étape suivante est d'atteindre l'ignition, où, comme avec des carburants fossiles, le processus de réaction devient auto suffisant sans chauffage additionnel.
Le progrès vers l'ignition pout être mesuré avec un simple paramètre.
La forme de la dépendance de la section efficace de fusion à l'énergie permet d'écrire le pré-requis pour l''ignition approximativement par :

\[
	, \tau_E T > \SI{5e21}{\per\meter\cubed\second\kilo\electronvolt}
\]

\begin{figure}
\centering
    % Minipage pour la légende (à gauche)
    \begin{minipage}[c]{0.35\textwidth}
        \caption{Dans un réacteur le produit \( n\tau_E \) de la densité et du temps de confinement de l'énergie et de la température \( T \), doivent tous deux être dans le bon domaine. En prenant les valeurs de pics, le \( n\tau_E \) nécessaire est \( \SI{2.5e20}{\per\meter\cubed\second} \) et la température autour de \( \qtyrange[range-units=single,range-phrase=-]{10}{20}{\kilo\electronvolt} \). La valeur requise du produit \( n \tau_E T \) est approximativement \( \SI{5e21}{\per\meter\cubed\second\kilo\electronvolt} \) (Section 5.1). La figure montre le progrès dans l'amélioration de ce produit, se dirigeant vers les limites des conditions d'un réacteur.}
        \label{fig:triple_product}
    \end{minipage}\hfill
    % Minipage pour l'image (à droite)
    \begin{minipage}[c]{0.55\textwidth}
        \flushright
        \includegraphics{figures/chapter1/build/triple_product.pdf}
    \end{minipage}
	
\end{figure}

où \( n \) et \( T \) sont la densité et la température maximum dans le plasma et \( \tau_E \) est le temps de confinement.
L'amélioration de la valeur de ce paramèrre peut être vue dans la \autoref{fig:triple_product}.

On considère aujourd’hui qu'un tokamak qui atteindrait l'ignition peut être construit.
Cependant, le design d'un tel réacteur pose un lare panel de questions.
Un réacteur commercial encore plus.
Les recherchent actuelles sont orientés pour répondre à ces question et ce livre donne une introduction à nos connaissances de la physique sous-jacente.

\section{Réactions de fusion}
\label{sec:reaction_de_fusion}

De loin, la réaction de fusion la plus prometeuse est celle dans laquelle les noyaux de deutérium et de tritium fusionne pour produire une particule alpha avec  la libération d'un neutron, en détail :

\[
\begin{array}{r r l c l c l}
_1^2 \text{D} + & _1^3 \text{T} & \rightarrow & _2^4 He & + & _0^1n &\\
  &  &          & \vert & & \vert &\\
  &  &          & \SI{3.5}{\mega\electronvolt} & + & \SI{14.1}{\mega\electronvolt} & =  \SI{17.6}{\mega\electronvolt}\\
\end{array}
\]

Où les énergies données sont les énergies cinétiques des produits de la réaction.
Le bilan de masse et d’énergie vient du déficit de mass \( \delta m \) de la réaction :

\[
\begin{array}{c c c c c}
_1^2 \text{D} & + & _1^3 \text{T} & \\
(2 - 0.000994)m_p & & (3-0.006284)m_p & \\
 & \rightarrow & _2^4 He & + & _0^1n\\
 & & (4- 0.027404)m_p & & (1 + 0.001378)m_p
\end{array}
\]

où \( m_p \) est la masse d'un proton ( \( \SI{1.6726e-27}{\kilogram} \)).
Le déficit de masse est de \( 0.01875 m_p \), et l'énergie libérée est de 
\[
	E = \delta l c^2 = 0.01875 m_p c^2 = \SI{2.818e-12}{joules} = \SI{17.59}{\mega\electronvolt}
\]

\begin{figure}
    \centering
    % Minipage pour la légende (à gauche)
    \begin{minipage}[c]{0.35\textwidth}
        \caption{L'énergie potentiel en fonction de la séparation nucléaire}
        \label{fig:potential_energy}
    \end{minipage}\hfill
    % Minipage pour l'image (à droite)
    \begin{minipage}[c]{0.55\textwidth}
        \centering
        \includegraphics{figures/chapter1/build/potential_energy.pdf}
    \end{minipage}
\end{figure}

La réaction est induite par des collisions entre les particules, la section efficace de la réaction est donc d'une importance fondamentale.
La section efficace à basse énergie d'impact et faible car la barrière de Coulomb empêche les noyaux de s'approcher à des dimensions nucléaires requisent pour que la fusion ait lieu.
Le potentiel est illustré dans la \autoref{fig:potential_energy}.

Grâce à 'effet tunnel de la mécanique quantique, la fusion D-T se produit à des énergies inférieures que celles requse'nt pour outrepasser la barrière de Coulomb.
La section efficace pour les réactions est données dans ma \autoref{fig:cross_section}, et on voit que la section efficace maximale a lieu juste au dessus de \( \SI{100}{\kilo\electronvolt} \).

\begin{figure}
    \centering
    % Minipage pour la légende (à gauche)
    \begin{minipage}[l]{0.35\textwidth}
        \caption{Section efficace pour les réactions D-T, D-D et D-\( ^3 \)He. Les deux réactions D-D ont des section efficaces similaires, le graph montre leur somme.}
        \label{fig:cross_section}
    \end{minipage}\hfill
    % Minipage pour l'image (à droite)
    \begin{minipage}[c]{0.55\textwidth}
        \centering
        \includegraphics[width=\textwidth]{figures/chapter1/build/cross_section.pdf}
    \end{minipage}
\end{figure}

La raison pour laquelle la réaction D-T est priorisé par rapport aux autres réactions est clairs au vu de la \autoref{fig:cross_section}, où les sections efficaces pour :

\begin{align*}
	^2 \text{D} + ^2 \text{D} & \rightarrow ^3 \text{He} + ^1 \text{n} + \SI{3.27}{\mega\electronvolt}\\
	^2 \text{D} + ^2 \text{D} & \rightarrow ^3 \text{T} + ^1 \text{H} + \SI{4.03}{\mega\electronvolt}\\
	^2 \text{D} + ^3 \text{Je} & \rightarrow ^4 \text{jE} + ^1 \text{h} + \SI{18.3}{\mega\electronvolt}
\end{align*}

sont aussi montrées.
On voit que les sections efficaces sont considérablement plus basse que pour D-T sauf à des énergies excessivement élevées.

\section{Fusion thermonucléaire}
\label{sec:fusion_thermonucleaire}

Le calcul du taux de réaction dans un plasma de D-T chaud demande une intégration sur la fonction de distribution des deux espèces.
Le taux de réaction par unité de volume entre les particules d'une espèces avec une vitesse \( v_1 \) et les particules d'une autre espèce avec la vitesse \( v_2 \) est 
\[
	\sigma\left( v'\right) f_1\left( v_1\right) f_2\left( v_2\right)
\]

où

\[
	v' = v_1 - v_2
\]

et \( f_1 \) et \( f_2 \) sont les fonctions de distribution.
Si les distributions sont Maxwellienne

\[
	f_j\left( v_j \right) = nèj \left( \frac{m_j}{2 \pi T} \right)^{3/2} \exp - \frac{m_jv_j^2}{2T}
\]

Le taux de réaction total par unité de volume 

\[
	\mathscr{R} = \int \int  \sigma\left( v'\right) f_1\left( v_1\right) f_2\left( v_2\right) \text{d}^3 v_1 \text{d}^3 v_2 
\]

peut être écrit

\begin{align*}
	\mathscr{R} = n_1 n_2 \frac{\left( m_1 m_2 \right)^{3/2}}{\left( 2 \pi T \right)^3} \int \int &\exp \left( - \frac{m_1 + m_2}{2 T} \left( V + \frac{1}{2} \frac{m_1 - m_2}{m_1 + m_2} v' \right)^2 \right)\\ \times & \sigma \left( v' \right) v' \exp \left(- \frac{\mu v'}{2 T} \right)^2 \text{d}^3 v' \text{d}^3 V 
\end{align*}

où

\[
	V = \frac{v_1 + v_2}{2} \quad \text{et} \mu ) \frac{m_1 m_2}{m_1 + m_2}
\]

\( \mu \) étant la masse réduite.

L'intégrale sur \( V \) est \( \left( 2 \pi T / \left( m_1 + m_2 \right) \right)^{3/2} \) donc 

\begin{equation}
	\mathscr{R} = 4 \pi n_1 n_2 \left(\frac{\mu}{2 \pi T} \right)^{3/2} \int \sigma \left(v' \right) v'^3 \exp \left(- \frac{\mu v'^2}{2 T} \right) \text{d} v'
	\label{eq:taux_reaction_1}
\end{equation}

Les sections efficaces mesurées en laboratoire sont générallement données en terme d'énergie de la particule incidente, par exemple du type 1,  qui est 
\[
	\varepsilon = \frac{1}{2} m_1 v'^2
\]

L'\autoref{eq:taux_reaction_1} peut être plus conventionellement écrite
\begin{equation}
	\mathscr{R} = \left( \frac{8}{\pi} \right)^{1/2} n_1 n_2 \left(\frac{\mu}{T} \right)^{3/2} \int \sigma \left( \varepsilon \right) \varepsilon \exp \left(- \frac{\mu \varepsilon}{m_1 T} \right) \text{d} \varepsilon
	\label{eq:taux_reaction_2}
\end{equation}

\begin{figure}
	\centering
    % Minipage pour la légende (à gauche)
    \begin{minipage}[l]{0.35\textwidth}
        \caption{\( \left\langle \sigma v \right\rangle \) pour les réactions D-T en fonction de la température du plasma.}
        \label{fig:sigma_v}
    \end{minipage}\hfill
    % Minipage pour l'image (à droite)
    \begin{minipage}[c]{0.55\textwidth}
        \centering
        \includegraphics[width=\textwidth]{figures/chapter1/build/sigma_v.pdf}
    \end{minipage}
\end{figure}

Si la section efficace \( \sigma\left(\varepsilon \right) \) pour les réactions D-T donnée en \autoref{sec:reaction_de_fusion} est placée dans l'intégrale de l'\autoref{eq:taux_reaction_2}, on obtient le taux de réaction \( \mathscr{R} = n_1 n_2 \left\langle \sigma v \right\rangle  \)  où \( \left\langle \sigma v \right\rangle \)  est donné dans la \autoref{fig:sigma_v}.
Pour une densité d'ion donnée le taux maximum est obtenu pour \( n_D = n_T \).

\end{document}