\documentclass[main.tex]{subfiles}

\begin{document}

\chapter{Fusion}

\section{Fusion and tokamaks}
\label{sec:fusion_et_tokamaks}

Si un noyau de deutérium fusionne avec un noyau de tritium, une parti cule \( \alpha \) est produite et un neutron est relaché.
Le réarrangement nucléaire  induit une réduction de la masse totale et, par conséquent, de l'énergie est relaché sous la forme d'énergie cinétique des produits de la réaction.
L'énergie relachée par chaque réaction est de \( \SI{17.6}{\mega\electronvolt} \).
En terme macroscopique, un seul \( \SI{}{\kilogram} \) de ce carburant relacherais \( \SI{e8}{\kWh} \) d'énergie et serait ce qu'une centrale de \( \SI{1}{\GW} \) (éllectrique) consommerait en un jour.

Le deutérium est une ressource abondante mais le tritium n'apparait pas naturellement.Il devrait, néanmoins, être possoble d'utiliser les neutrons relachés par la réaction de fusion pour produire du tritum à partir de lithium, dont de larges réserves existent.

Pour déclencher la réaction de fusion entre des noyaux de deutérium et de tritium, il est nécessaire de surmonter la répulsion mutuelle due à leurs charges positives. En conséquence, la section efficace de fusion est faible aux basses énergies.
Néanmoins, la section efficace augmente avec l'énergie, et atteint un maximumà \( \SI{100}{\kilo\electronvolt} \), et une balance énergétique positive est possible s'il est possible de faire réagir les particules du carburant avant qu'elles ne perdent leur énergie.
Pour obtenir ce résultat, les particules doivent garder ler énergie et rester dans la région de réaction pour un temps suffisament long.
Plus exactement, le produit de ce temps et la densité de particules qui réagissent doit être suffisamment grand.

La approches simples consistant à diriger un faisceau de particules vers une cible solide ou à travers un autre faisceau ne permettent pas de satisfaire ce critère.
Dans le premier cas, les particules perdent leur énergie trop rapidement et dans le second la densité est trop faible.

La méthode la plus prometteuse pour fournir l'énergie est de chauffer le carburant de deutérium-tritium à une température suffisante pour que la vélocité thermique des noyaux soit assez haute pour produire les réactions requisent.
La fusion ammené par cette méthodfe est appelé fusion thermonucléaire.
La température optimale n'est pas aussi élevé que celle correspondant à l'énergie de la section efficace maximale parce que les réactions requisent  se produisent dabs ka queue de haute énergie de la distribution Maxwellienne des particules chauffées.
La température nécessaire est autour de \( \SI{10}{\kilo\electronvolt} \), ou environ \( 100 \) million de degrés celcius.
À de telles températures, le carburant est entiérement ionisé.
La charge électrostatique des ions nucléaires est neutralisé par la présence d'un nombre égal d'électrons et le gaz neutre résultant de cela est appelé un plasma.

Puisque de telles températures empechent un confinement par des murs, une autre méthode est nécessaire.
Le tokamak est une méthode possible.
Dans un tokamak, les particules du plasmas sont confinés dans une région toroïdale par un champ magnétique, elles sont tenues par le champ dans de petites orbites de giration.
Bien que les températures nécessaires, la densité et le temps de confinement ont tous été obtenus dans des tokamaks, ils n'ont pas été obtenus dans le même plasma.
Cependant, le progrès vers cet objectif a été remarquable , et une puissance thermonucléaire de plus de quatre-bingt-dix pourcent de la puissance injectée a été produite.
Une étape suivante est d'atteindre l'ignition, où, comme avec des carburants fossiles, le processus de réaction devient auto suffisant sans chauffage additionnel.
Le progrès vers l'ignition pout être mesuré avec un simple paramètre.
La forme de la dépendance de la section efficace de fusion à l'énergie permet d'écrire le pré-requis pour l''ignition approximativement par

\[
	, \tau_E T > \SI{5e21}{\per\meter\cubed\second\kilo\electronvolt}
\]

\begin{figure}
\centering
    % Minipage pour la légende (à gauche)
    \begin{minipage}[c]{0.35\textwidth}
        \caption{Dans un réacteur le produit \( n\tau_E \) de la densité et du temps de confinement de l'énergie et de la température \( T \), doivent tous deux être dans le bon domaine. En prenant les valeurs de pics, le \( n\tau_E \) nécessaire est \( \SI{2.5e20}{\per\meter\cubed\second} \) et la température autour de \( \qtyrange[range-units=single,range-phrase=-]{10}{20}{\kilo\electronvolt} \). La valeur requise du produit \( n \tau_E T \) est approximativement \( \SI{5e21}{\per\meter\cubed\second\kilo\electronvolt} \) (Section 5.1). La figure montre le progrès dans l'amélioration de ce produit, se dirigeant vers les limites des conditions d'un réacteur.}
        \label{fig:triple_product}
    \end{minipage}\hfill
    % Minipage pour l'image (à droite)
    \begin{minipage}[c]{0.55\textwidth}
        \flushright
        \includegraphics{figures/chapter1/triple_product.pdf}
    \end{minipage}
	
\end{figure}

où \( n \) et \( T \) sont la densité et la température maximum dans le plasma et \( \tau_E \) est le temps de confinement.
L'amélioration de la valeur de ce paramèrre peut être vue dans la \autoref{fig:triple_product}.

On considère aujourd’hui qu'un tokamak qui atteindrait l'ignition peut être construit.
Cependant, le design d'un tel réacteur pose un lare panel de questions.
Un réacteur commercial encore plus.
Les recherchent actuelles sont orientés pour répondre à ces question et ce livre donne une introduction à nos connaissances de la physique sous-jacente.

\section{Réactions de fusion}
\label{sec:reaction_de_fusion}

De loin, la réaction de fusion la plus prometeuse est celle dans laquelle les noyaux de deutérium et de tritium fusionne pour produire une particule alpha avec  la libération d'un neutron, en détail

\[
\begin{array}{r r l c l c l}
_1^2 \text{D} + & _1^3 \text{T} & \rightarrow & _2^4 He & + & _0^1n &\\
  &  &          & \vert & & \vert &\\
  &  &          & \SI{3.5}{\mega\electronvolt} & + & \SI{14.1}{\mega\electronvolt} & =  \SI{17.6}{\mega\electronvolt}\\
\end{array}
\]

Où les énergies données sont les énergies cinétiques des produits de la réaction.
Le bilan de masse et d’énergie vient du déficit de mass \( \delta m \) de la réaction

\[
\begin{array}{c c c c c}
_1^2 \text{D} & + & _1^3 \text{T} & \\
(2 - 0.000994)m_p & & (3-0.006284)m_p & \\
 & \rightarrow & _2^4 He & + & _0^1n\\
 & & (4- 0.027404)m_p & & (1 + 0.001378)m_p
\end{array}
\]

où \( m_p \) est la masse d'un proton ( \( \SI{1.6726e-27}{\kilogram} \)).
Le déficit de masse est de \( 0.01875 m_p \), et l'énergie libérée est de 

\[
	E = \delta m c^2 = 0.01875 m_p c^2 = \SI{2.818e-12}{joules} = \SI{17.59}{\mega\electronvolt}.
\]

\begin{figure}
    \centering
    % Minipage pour la légende (à gauche)
    \begin{minipage}[c]{0.35\textwidth}
        \caption{L'énergie potentiel en fonction de la séparation nucléaire}
        \label{fig:potential_energy}
    \end{minipage}\hfill
    % Minipage pour l'image (à droite)
    \begin{minipage}[c]{0.55\textwidth}
        \centering
        \includegraphics{figures/chapter1/potential_energy.pdf}
    \end{minipage}
\end{figure}

La réaction est induite par des collisions entre les particules, la section efficace de la réaction est donc d'une importance fondamentale.
La section efficace à basse énergie d'impact et faible car la barrière de Coulomb empêche les noyaux de s'approcher à des dimensions nucléaires requisent pour que la fusion ait lieu.
Le potentiel est illustré dans la \autoref{fig:potential_energy}.

Grâce à 'effet tunnel de la mécanique quantique, la fusion D-T se produit à des énergies inférieures que celles requse'nt pour outrepasser la barrière de Coulomb.
La section efficace pour les réactions est données dans la \autoref{fig:cross_section}, et on voit que la section efficace maximale a lieu juste au dessus de \( \SI{100}{\kilo\electronvolt} \).

\begin{figure}
    \centering
    % Minipage pour la légende (à gauche)
    \begin{minipage}[l]{0.35\textwidth}
        \caption{Section efficace pour les réactions D-T, D-D et D-\( ^3 \)He. Les deux réactions D-D ont des section efficaces similaires, le graph montre leur somme.}
        \label{fig:cross_section}
    \end{minipage}\hfill
    % Minipage pour l'image (à droite)
    \begin{minipage}[c]{0.55\textwidth}
        \centering
        \includegraphics[width=\textwidth]{figures/chapter1/cross_section.pdf}
    \end{minipage}
\end{figure}

La raison pour laquelle la réaction D-T est priorisé par rapport aux autres réactions est clairs au vu de la \autoref{fig:cross_section}, où les sections efficaces pour

\begin{align*}
	^2 \text{D} + ^2 \text{D} & \rightarrow ^3 \text{He} + ^1 \text{n} + \SI{3.27}{\mega\electronvolt}\\
	^2 \text{D} + ^2 \text{D} & \rightarrow ^3 \text{T} + ^1 \text{H} + \SI{4.03}{\mega\electronvolt}\\
	^2 \text{D} + ^3 \text{Je} & \rightarrow ^4 \text{jE} + ^1 \text{h} + \SI{18.3}{\mega\electronvolt}
\end{align*}

sont aussi montrées.
On voit que les sections efficaces sont considérablement plus basse que pour D-T sauf à des énergies excessivement élevées.

\section{Fusion thermonucléaire}
\label{sec:fusion_thermonucleaire}

Le calcul du taux de réaction dans un plasma de D-T chaud demande une intégration sur la fonction de distribution des deux espèces.
Le taux de réaction par unité de volume entre les particules d'une espèces avec une vitesse \( v_1 \) et les particules d'une autre espèce avec la vitesse \( v_2 \) est 
\[
	\sigma\left( v'\right) f_1\left( v_1\right) f_2\left( v_2\right)
\]

où

\[
	v' = v_1 - v_2
\]

et \( f_1 \) et \( f_2 \) sont les fonctions de distribution.
Si les distributions sont Maxwelliennes

\[
	f_j\left( v_j \right) = n_j \left( \frac{m_j}{2 \pi T} \right)^{3/2} \exp - \frac{m_jv_j^2}{2T},
\]

Le taux de réaction total par unité de volume 

\[
	\mathscr{R} = \int \int  \sigma\left( v'\right) f_1\left( v_1\right) f_2\left( v_2\right) \text{d}^3 v_1 \text{d}^3 v_2 
\]

peut être écrit

\begin{align*}
	\mathscr{R} = n_1 n_2 \frac{\left( m_1 m_2 \right)^{3/2}}{\left( 2 \pi T \right)^3} \int \int &\exp \left( - \frac{m_1 + m_2}{2 T} \left( V + \frac{1}{2} \frac{m_1 - m_2}{m_1 + m_2} v' \right)^2 \right)\\ \times & \sigma \left( v' \right) v' \exp \left(- \frac{\mu v'}{2 T} \right)^2 \text{d}^3 v' \text{d}^3 V 
\end{align*}

où

\[
	V = \frac{v_1 + v_2}{2} \quad \text{et} \mu ) \frac{m_1 m_2}{m_1 + m_2},
\]

\( \mu \) étant la masse réduite.

L'intégrale sur \( V \) est \( \left( 2 \pi T / \left( m_1 + m_2 \right) \right)^{3/2} \) donc 

\begin{equation}
	\mathscr{R} = 4 \pi n_1 n_2 \left(\frac{\mu}{2 \pi T} \right)^{3/2} \int \sigma \left(v' \right) v'^3 \exp \left(- \frac{\mu v'^2}{2 T} \right) \text{d} v'.
	\label{eq:taux_reaction_1}
\end{equation}

Les sections efficaces mesurées en laboratoire sont générallement données en terme d'énergie de la particule incidente, par exemple du type 1,  qui est 
\[
	\varepsilon = \frac{1}{2} m_1 v'^2
\]

L'\autoref{eq:taux_reaction_1} peut être plus conventionellement écrite
\begin{equation}
	\mathscr{R} = \left( \frac{8}{\pi} \right)^{1/2} n_1 n_2 \left(\frac{\mu}{T} \right)^{3/2} \int \sigma \left( \varepsilon \right) \varepsilon \exp \left(- \frac{\mu \varepsilon}{m_1 T} \right) \text{d} \varepsilon.
	\label{eq:taux_reaction_2}
\end{equation}

\begin{figure}
	\centering
    % Minipage pour la légende (à gauche)
    \begin{minipage}[l]{0.35\textwidth}
        \caption{\( \left\langle \sigma v \right\rangle \) pour les réactions D-T en fonction de la température du plasma.}
        \label{fig:sigma_v}
    \end{minipage}\hfill
    % Minipage pour l'image (à droite)
    \begin{minipage}[c]{0.55\textwidth}
        \centering
        \includegraphics[width=\textwidth]{figures/chapter1/sigma_v.pdf}
    \end{minipage}
\end{figure}

Si la section efficace \( \sigma\left(\varepsilon \right) \) pour les réactions D-T donnée en \autoref{sec:reaction_de_fusion} est placée dans l'intégrale de l'\autoref{eq:taux_reaction_2}, on obtient le taux de réaction \( \mathscr{R} = n_1 n_2 \left\langle \sigma v \right\rangle  \)  où \( \left\langle \sigma v \right\rangle \)  est donné dans la \autoref{fig:sigma_v}.
Pour une densité d'ion donnée le taux maximum est obtenu pour \( n_D = n_T \).

\begin{figure}
	\centering
    % Minipage pour la légende (à gauche)
    \begin{minipage}[l]{0.35\textwidth}
        \caption{Tracé de l'\autoref{eq:taux_reaction_2} et de ses deux facteurs \( \sigma \left( \varepsilon \right) \) et \( \varepsilon \exp \left( - \mu \varepsilon / m_D T \right) \) en fonction de l'énergie normalisé \( \varepsilon/T \) pour un plasma de D-T ) une température de \( T = \SI{10}{\kilo\electronvolt} \).}
        \label{fig:integrand}
    \end{minipage}\hfill
    % Minipage pour l'image (à droite)
    \begin{minipage}[c]{0.55\textwidth}
        \centering
        \includegraphics[width=\textwidth]{figures/chapter1/integrand.pdf}
    \end{minipage}
\end{figure}

Aux températures d'intérêts les réactions nucléaires viennent principalement de la queue de la distribution.
Cela est visible dans la \autoref{fig:integrand} où l'intégrande de l'\autoref{eq:taux_reaction_2} est tracé en fonction de \( \varepsilon/T \) ainsi que les  deux facteurs \( \sigma \left( \varepsilon \right) \) et \( \varepsilon \exp \left( - \mu \varepsilon / m_D T \right) \) pour un plasma de D-T ) une température de \( T = \SI{10}{\kilo\electronvolt} \).

\begin{figure}
	\centering
    % Minipage pour la légende (à gauche)
    \begin{minipage}[l]{0.35\textwidth}
        \caption{\( \left\langle \sigma v \right\rangle \) pour les réactions D-D (total) et D-\( ^3 \)He en fonction de la température du plasma. Les valeurs sont nettement plus faibles que pour les réactions D-T, qui sont incluses pour comparaison.}
        \label{fig:sigma_v_3_reaction}
    \end{minipage}\hfill
    % Minipage pour l'image (à droite)
    \begin{minipage}[c]{0.55\textwidth}
        \centering
        \includegraphics[width=\textwidth]{figures/chapter1/sigma_v_3_rections.pdf}
    \end{minipage}
\end{figure}

Les expériences sont générallement réalisées avec di deitéroiù plutôt qu'un mélange de deutérium-tritium.
Une  représentation de \( \left\langle \sigma v \right\rangle \) pour le deutérium est donnée dans la \autoref{fig:sigma_v_3_reaction} avec aussi celui pour D-\( ^3 \)He.
Dans le domaine de température \( \qtyrange[range-units=single,range-phrase=-]{5}{20}{\kilo\electronvolt} \) le rapport de \( \left\langle \sigma v \right\rangle \) pour le D-T et celui du deutérium est d'environ \( 80 \).

\section{Bilan de puissance}
\label{sec:bilan_de_puissance}

\subsubsection*{Puissance thermonucléaire}

La puissance thermonucléaire par unité de volume dans un plasma D-T est

\begin{equation}
	p_{T_n} = n_D n_T  \left\langle \sigma v \right\rangle \mathscr{E},
	\label{eq:puissance_thermonucléaire}
\end{equation}

où \( n_D \) et \( n_T \) sont les densité de deutérium et de tritium, \( \left\langle \sigma v \right\rangle \) est le taux donnée dans la \autoref{fig:sigma_v} et 
\( \mathscr{E} \) est l'énergie relachée par réaction.
La densité totale est
\[
	n ) n_D + n_T,
\]

l'\autoref{eq:puissance_thermonucléaire} peut être écrite

\[
	p_{T_n} = n_D \left(n - n_D\right)  \left\langle \sigma v \right\rangle \mathscr{E}.
\]

Pour une densité donnée \( n \) la puissance est maximisée par \( n_D = \frac{1}{2} n \), les densités de deutérium et de tritium sont égales.
Pour ce mélange optimal, la densité de puissance thermonucléaire est

\begin{equation}
	p_{T_n} = \frac{1}{4} n^2  \left\langle \sigma v \right\rangle \mathscr{E}
	\label{eq:puisance_thermonucleaire_2}.
\end{equation}

\subsubsection*{Perte d'énergie}

Dans un tokamak, le plasma subit une perte d’énergie continue qui doit être compensée par un chauffage du plasma.
L'énergie moyenne du plasma à une température \( T \) est de \( \frac{3}{2} T \), composé de \( \frac{1}{2} T \) par defré de liberté.
Puisqu'il y a un nombre égal d'ions et d'électrons, l'énergie du plasma par unité de volume est de \( 3 n T \).
L'énergie total dans le plasma est donc

\begin{align}
	W & = \int 3 n T \text{d} x \nonumber \\
	 & = 3 \overline{nT} V,
	 \label{eq:puissance_totale}
\end{align}

où la barre représente la valeur moyenne, et \( V \) le volume du plasma.
Le taux de perte d'énergie \( P_L \), est caractérisé par un temps de confinement de l'énergie définie par la relation

\begin{equation}
	P_L = \frac{W}{\tau_E}
	\label{eq:perte_energie}
\end{equation}

Dans les tokamaks actuels, la puissance thermonucléaire est généralement basse et en régime stationnaire la perte d'énergie est compasé par un chauffage externe. Donc, si la puissance externe est \( P_H \),

\begin{equation}
	P_H = P_L
	\label{eq:puissance_chauffage}
\end{equation}

et les équations \ref{eq:perte_energie} et \ref{eq:puissance_chauffage} donnent
\[
	\tau_E = \frac{W}{P_H}
\]

Cette expression permet d'avoir un moyen de déterminer \( \tau_E \) depuis des quantités connues expérimentallement.

\subsubsection*{Chauffage par particule \( \alpha \)}

La puissance thermonucléaire donnée par l'\autoref{eq:puisance_thermonucleaire_2} est composé de deux parties.
Quatre cinquièmes de l'énergie de la réaction est porté par les neutrons et le reste, \( \mathscr{E}_\alpha \)n est porté les particules \( \alpha \).
Les neutrons quittent le plasma sans intéragir, mais les particules \( \alpha \), étant chargées, sont confinées par le champ magnétique.
Les particules \( \alpha \) transfèrent leur énergie de \( \SI{3.5}{\mega\electronvolt} \) au plasma par collisions

Donc le chauffage par particules \( \alpha \) par unité de volume est de
 
\begin{equation}
	p_\alpha = \frac{1}{4} n^2 \left\langle \sigma v \right\rangle \mathscr{E}
	\label{eq:puissance_alpha}
\end{equation}

Et le chauffage par particules \( \alpha \) total est

\begin{align}
	P_\alpha &= \int p_\alpha \text{d}^3 x\nonumber \\
	&= \frac{1}{4} \overline{n  \left\langle \sigma v \right\rangle }\mathscr{E} V.
	\label{eq:chauffage_total}
\end{align}

\subsubsection*{Bilan de puissance}

Dans le bilan de puissance global la perte de puissance est compebsé par un chauffage externe et la puissance des particules \( \alpha \). Autrement dit

\[
	P_H + P_\alpha = P_L
\]

et en utilisant les équations \ref{eq:puissance_totale}, \ref{eq:perte_energie} et \ref{eq:chauffage_total}, ce bilan est donné par 
\begin{equation}
	P_H + \frac{1}{4} \overline{n  \left\langle \sigma v \right\rangle }\mathscr{E} V = \frac{3 \overline{n T}}{\tau_E} V 
	\label{eq:puissance_chauffage_moyenne}
\end{equation}

Les implications de cette équations sont décritents dans la section suivante.

\section{Ignition}
\label{sec:ignition}

\subsubsection*{Conditions de l'ignition}

Lorsqu'un plasma de D-T est chauffé à des conditions thermonucléaire, le chauffage par particules \( \alpha \) fourni une proportion de plus en plus importante du chauffage total.
Lorsque les bonnes conditions de confinement sont obtenues, il arrive un point où la température du plasma peut être maintenue malgré les pertes d'énergie uniquement  grâce au chauffage par particules \( \alpha \).
Le chauffage additionnel peut alors être retiré et la température du plasma est entretenue par le chauffage interne.
Par analogie avec les combustibles fossiles, on parle alors d’ignition.

Le bilan de puissance est décrit dans l'\autoref{eq:puissance_chauffage_moyenne} et en prenant une densité et une température constante pour simplifié, il peut être noté

\begin{equation}
	P_H = \left( \frac{3nT}{\tau_E} - \frac{1}{4}n^2 \left\langle \sigma v \right\rangle \mathscr{E}_\alpha \right) V.
	\label{eq:puissance_chauffage_simplifiee}
\end{equation}

L'\autoref{eq:puissance_chauffage_simplifiee} donne la condition pour l'ignition, le pré-requis pour que le plasma brule de façon autonome étant

\begin{equation}
	n \tau_E > \frac{12}{\left\langle \sigma v \right\rangle} \frac{T}{\mathscr{E}_\alpha}
	\label{eq:ignition_inegalite}
\end{equation}

\begin{figure}
	\centering
    % Minipage pour la légende (à gauche)
    \begin{minipage}[l]{0.35\textwidth}
        \caption{La valeur de \( n \tau_E \) nécessaire pour obtenir l'ignition en fonction de la température.}
        \label{fig:n_tau_E_ignition}
    \end{minipage}\hfill
    % Minipage pour l'image (à droite)
    \begin{minipage}[c]{0.55\textwidth}
        \centering
        \includegraphics[width=\textwidth]{figures/chapter1/n_tau_E_ignition.pdf}
    \end{minipage}
\end{figure}

Le coté droit de l'inégalité \ref{eq:ignition_inegalite} est en fonction de la température uniquement et un tracé de la manière dont la valeur de \( n \tau_E \) requise dépend de la température est donné dans la \autoref{fig:n_tau_E_ignition}.
Le minimum est à \( T = \SI{30}{\kilo\electronvolt} \) et le pré-requis pour l'ignition à cette température est

\begin{equation}
	n \tau_E > \SI{1.5e20}{\per\meter\cubed\second}
\end{equation}

Cependant, puisque \( \tau_E \) est lui-mme une fonction de la temperature, la temperature du minimum ne doit pas être prise comme une condition optimale.
Il se trouve que la temperature d'ignition est probablement quelque peu en dessous.
Par une heureuse coïncidence pour les calculs, dans le domaine de \( \qtyrange[range-units=single,range-phrase=-]{10}{20}{\kilo\electronvolt} \), le taux de réaction peut être pris, avec moins de \( \SI{10}{\percent} \) d'erreur, comme une valeur constante

\begin{equation}
	\left\langle \sigma v \right\rangle = \SI{1.1e-24}{\text{T}\squared \meter\cubed\per\second}, \quad \text{T en} \SI{}{\kilo\electronvolt}
	\label{eq:sigma_v_constant} 
\end{equation}

en utilisant \( \mathscr{E}_\alpha = \SI{3.5}{\mega\electronvolt} \), la condition d'ignition devient

\begin{equation}
	n T \tau_E > \SI{3e21}{\per\meter\cubed\kilo\electronvolt\second}.
	\label{eq:critere_lawson}
\end{equation}

C'est une forme très pratique pour la condition d'ignition puisqu'elle donne clairement les pré-requis sur la densité, la température et le temps de confinement.
Cette condtion peut être atteinte, par exemple, par \( n = \SI{e20}{\per\meter\cubed} \), \( T = \SI{10}{\kilo\electronvolt} \) et \( \tau_E = \SI{3}{\second} \).

La valeur précise de la constante dans la condition \ref{eq:critere_lawson} dépend du profil de \( n \) et de \( T \) et si on prend la moyenne ou les valeurs extremes.
La condition \ref{eq:critere_lawson} est pour un profil constant.
Pour des profils de densité et température paraboliques, le pré-requis de l'ignition sur les valeurs extremes est

\begin{equation}
	\hat{n}\hat{T} \tau_E > \SI{5e21}{\per\meter\cubed\kilo\electronvolt\second}
	\label{eq:critere_lawson_peak}
\end{equation}

La relation \ref{eq:ignition_inegalite} fait pensr au critère de Lawson.
Aux débuts de la recherche sur la fusion, Lawson a identifié le produit de la densité et du temps de confinement, \( n \tau \) comme un paramètre criitique pour un réacteur thermonucléaire.
Cependant, dans ses calculs il a négligé le chauffage par les particules-\( \alpha \) et a supposé que le plasma serait chauffé par une source externe.
C'est donc clairement une condition nécessaire, mais non suffisante, que la puissance produite par le réacteur, après les pertes de la conversion en électricité, doit être capale de fournir le chauffage appliqué

Dans les calculs sur l'ignition ci-dessus, seule la fraction des particles-\( \alpha \), \( \SI{20}{\percent} \), de l'énergie totale est utilisée pour chauffé le plasma.
Dans les calculs de Lawson, le facteur correspondant es lié à l'efficacité de la centrale électrique, \( \eta \), avec \( \eta  \simeq \SI{30}{\percent} \).
Donc le \( n \tau \) de Lawson se trouve être moins contraignant que le critère d'ignition \ref{eq:ignition_inegalite}, nécéssitant \( n \tau > \SI{0.6e20}{\per\meter\cubed\second} \).
Lawson a aussi pris en compte le rayonnement Bremsstrahlung de l'hydrogène mais, comme il sera montré dans la section \color{red}{Section 4.24 (lien à rajouter quand elle sera faite)}\color{black}, cette perte est faible pour un plasma de tokamak.

Une façon de mesuré le succès à s'approcher des conditions d'un réacteurs est donné par le rapport, \( Q \), entre la puissance thermonucléaire et la uissance du chauffage appliqué, donné par

\[
	Q = \frac{\frac{1}{4} n^2 \left\langle \sigma v \right\rangle \mathscr{E} V}{P_H}
\]

Puisque l'énergie \( \mathscr{E} \) relaché par chaque réaction est cinq fois l'énergie de la particule-\( \alpha \), \( \mathscr{E} \), \( Q \) peut aussi être écrit

\[
	Q = \frac{5 P_\alpha}{P_H}
\]

Donc, \( Q= 1 \) correspond à une puissance des particules-\( \alpha \) de \( \SI{20}{\percent} \) la puissance de chauffage appliquée.
À l'ignition, où \( P_H \) peut être mise à \( 0 \), \( Q \rightarrow \infty \).
On peut voir que même si un plasma en ignition a la propriété désirable qu'aucun chauffage n'est nécessaire, il est possible d'obtenir un \( Q \) très large sans ignition.
Cependant, dans ce cas, la puissance fournie, \( P_H \) représente un coût pour le système, puisqu’elle implique de recycler une partie de la puissance du réacteur, avec la perte d’efficacité correspondante.

\subsubsection*{Approche de l'ignition}

L'approche de l'ignition peut être décrite en ajoutant la dépendance en t'emps de l'\autoref{eq:puissance_chauffage_simplifiee}n cela donne

\begin{equation}
	\frac{\dd}{\dd t}3nT = \frac{P_H}{V} + \frac{1}{4} n^2 \sigmav \mathscr{E}_\alpha - \frac{3nT}{\tau_E(n, T)}
	\label{eq:power_balance_time}
\end{equation}

Si la puissance de chauffage est augmentée lentement, la solution de l’\autoref{eq:power_balance_time} correspond à une succession d’états quasi stables, pour lesquels les termes du côté droit de l’équation s’annulent.
Le résultat dépend alors de la dépendance en température et en densité du temps de confinement, qui est incertaine dans le régime considéré.

\begin{figure}
	\centering
    % Minipage pour la légende (à gauche)
    \begin{minipage}[l]{0.35\textwidth}
        \caption{Le chauffage par particule-\( \alpha \) et les pertes pour un temps de confinement constant et une ignition à \( \SI{10}{\kilo\electronvolt} \). Le tracé en dessous montre le chauffage externe nécessaire pour compenser les pertes et maintenir une température stable.}
        \label{fig:power_comp}
    \end{minipage}\hfill
    % Minipage pour l'image (à droite)
    \begin{minipage}[c]{0.55\textwidth}
        \centering
        \includegraphics[width=\textwidth]{figures/chapter1/power_comp.pdf}
    \end{minipage}
\end{figure}

Lz type de comportement attendu est montré dans la \autoref{fig:power_comp}. 
On a la dépendance en température du chauffage par particules-\( \alpha \) et les pertes de puissance ainsi que la puissance de chauffage nécessaire, pour un temps de confinement constant et une ignition à \( \SI{10}{\kilo\electronvolt} \).
On voit donc que dans ce cas, la puissance de chauffage externe maximale est pour environ \( \SI{5}{\kilo\electronvolt} \) et cette puissance est alors moins de \( \SI{40}{\percent} \) de la puissance du chauffage par particules-\( \alpha \) ) la température d'ignition.

\vspace{2\baselineskip}
\begin{wrapfigure}{c}{0.4\textwidth}
    \centering
    % Réduction du retrait interne du wrap
    \vspace{-10pt}

    \begin{subfigure}[b]{0.48\textwidth}
        \centering
        \includegraphics[width=\textwidth]{figures/chapter1/contour.pdf}
        \caption{}
        \label{subfig:puissance_contour}
    \end{subfigure}
    
    \begin{subfigure}[b]{0.48\textwidth}
        \centering
        \includegraphics[width=\textwidth]{{figures/chapter1/wireframe.pdf}}
        \caption{}
        \label{subfig:puissance_3D}
    \end{subfigure}

    \caption{(a) Countours de la puissance nécessaire pour maintenir un étqt stationnaire dans le plan \( 'n, T) \). (b) Un traacé en deux dimension donnant la forme de \( P(n, T)  \) cirrespondant aux contours de la figure (a)}
    \vspace{-10pt}
    \label{fig:contour_3D}
\end{wrapfigure}

Une vue plus générale de l'approche de l'ignition est obtenue en considérant le bilan de puissance dans le plan \( 'n, T) \).
En utilisant l'\autoref{eq:puissance_chauffage_simplifiee} il est possible de dessiner les contours d'égales valeurs de \( P \) nécessaires pour maintenir une température donnée à une densité donnée.
Les incertitudes sur \( \tau_E (n, T) \) empêche le tracé d'un diagram précis, mais la \autoref{subfig:puissance_contour} montre la forme générale attendue \autoref{subfig:puissance_3D} donne la même information dans la forme d'un tracé deux dimentionnel de \( P(n, T) \).
On peut voir que la trajectoire vers l'ignition nécéssitant d'une puissance minimale est celle passant par dela le point de selle.
Ce point de selle est souvent appellé le passage de Cordey.

À l'ignition, la puissance appliquée peut être éteinte et il y a un équilibre entre le chauffage par particule-\( \alpha \) et les pertes de puissance du plasma.
Cependant pour le cas montré dans la \autoref{fig:power_comp} cet équilibre est instable.
Une légère augmentation de la température induit un déséquilibre positif entre le chauffage et les pertes, ce qui amplifie l’élévation de température.

Cette instabilité peut être analysée pour une situation plus générale dans laquelle \( \tau_E \) est pris comme étant en fonction de la température et la densité est considérée comme constante.
Donc, en prenant un chauffage dépendant du temps comme dans l'\autoref{eq:power_balance_time} avec \( P_H = 0 \),

\begin{equation}
	3n \frac{\dd T}{\dd t} = -3n \frac{T}{\tau_E (T)} + \frac{1}{4} n^2 \sigmav \mathscr{E}_\alpha .
	\label{eq:power_balance_time_density_constant}
\end{equation}

L'équilibre est donné par

\begin{equation}
	3 \frac{T}{\tau_E} = \frac{1}{4} n \sigmav \mathscr{E}_\alpha .
	\label{eq:power_balance_time_equilibrium}
\end{equation}

En considérant un petit changement, \( \Delta T \), de la température d'équilibre et en dévelopant \( \tau_E \) et \( \sigmav \) autour de leur valeur d'équilibre, l'\autoref{eq:power_balance_time_density_constant} donne l'équation gouvernant la stabilité.

\begin{equation}
	3, \frac{\dd \Delta T}{\dd t} = \left[ -3, \left( \frac{1}{\tau_E} - \frac{T}{\tau_E^2} \frac{\dd \tau_E}{\dd T} \right) +  \frac{1}{4} n^2 \frac{\dd \sigmav}{\dd T} \mathscr{E}_\alpha \right] \Delta T
	\label{eq:stability_equation}
\end{equation}

En utilisant l'équilibre de l'\autoref{eq:power_balance_time_equilibrium} dans l'\autoref{eq:stability_equation} on obtient

\[
	3, \frac{\dd \Delta T}{\dd t} =  \frac{1}{4} n^2 \sigmav \frac{\mathscr{E}_\alpha}{T} \left( -1 + \frac{T}{\tau_E} \frac{\dd \tau_E}{\dd T} + \frac{T}{\sigmav} \frac{\dd \sigmav}{\dd T} \right) \Delta T.
\]

Si la partie de droite de l'équation est positive, alors \( \Delta T \) croît exponentiellement.
La condition de la stabilité est donc

\[
	\frac{T}{\tau_E} \frac{\dd \tau_E}{\dd T} < 1 - \frac{T}{\sigmav} \frac{\dd \sigmav}{\dd T}.
\]

\begin{figure}
	\centering
    % Minipage pour la légende (à gauche)
    \begin{minipage}[l]{0.35\textwidth}
        \caption{La stabilité pour le chauffage par particules-\( \alpha \) deamnde que \( (T/\tau_E) \dd \tau_E / \dd T \) soit inférieur à une valeur critique. Le tracé conne la dépendance en température de cette valeur critique.}
        \label{fig:stability_condition}
    \end{minipage}\hfill
    % Minipage pour l'image (à droite)
    \begin{minipage}[c]{0.55\textwidth}
        \centering
        \includegraphics[width=\textwidth]{figures/chapter1/stability_condition.pdf}
    \end{minipage}
\end{figure}

\begin{figure}
	\centering
    % Minipage pour la légende (à gauche)
    \begin{minipage}[r]{0.35\textwidth}
        \caption{Une deterioration du confinement avec une augmentation de la température est un effet stabilisant pour le chauffage par particule-\( \alpha \). Le tracé montre la perte de puissance en fonction de la température pour \( \tau_E \propto 1/T \) avec la puissance de chauffage par particule-\( \alpha \) correspondant. Pour le cas montré, les conditions ont été choisies pour que l'ignition instable arrive à \( \SI{13}{\kilo\electronvolt} \). L'instabilité conduit alors la température vers un équilibre stable à \( \SI{15}{\kilo\electronvolt} \). Le graph rtonqué montre le chauffage additionnel nécessaire pour maintenir une température donnée.}
        \label{fig:power_comp_tau_E}
    \end{minipage}\hfill
    % Minipage pour l'image (à droite)
    \begin{minipage}[c]{0.55\textwidth}
        \centering
        \includegraphics[width=0.8\textwidth]{figures/chapter1/power_comp_tau_E.pdf}
    \end{minipage}
\end{figure}

\FloatBarrier
\section{Tokamaks}
\label{sec:tokamaks}

\vspace{2\baselineskip}
\begin{wrapfigure}{r}{0.2\textwidth}
    % Réduction du retrait interne du wrap
    \vspace{-10pt}

    \begin{subfigure}[b]{0.48\textwidth}
        \includegraphics[width=0.63\linewidth]{figures/chapter1/schematic_tokamak_1.pdf}
        \caption{}
        \label{subfig:schematic_tokamak_1}
    \end{subfigure}
    
    \begin{subfigure}[b]{0.48\textwidth}
        \includegraphics[width=0.3\linewidth]{{figures/chapter1/schematic_tokamak_2.pdf}}
        \caption{}
        \label{subfig:schematic_tokamak_2}
    \end{subfigure}

    \caption{(a) Le champ magnetique toroïdal \( B_\phi \) et le champ magnetique poloïdal \( B_p \) induit par le courant toroïdal \( I_\phi \). (b) La combinaison de \( B_\phi \) et \( B_p \) fait que les lignes de champ s’enroulent autour du plasma. }
    \vspace{-10pt}
    \label{fig:schematic_tokamak_1_2}
\end{wrapfigure}

Le tokamak est un système de confinement toroïdal du plasma, le plasma étant confiné par un champ magnétique.
Le champ magnétique principal étant le champ toroïdal.
Cependant, ce champ uniquement ne permet pas le confinement du plasma.
Pour avoir un équilibre dans lequel la pression du plasma est équilibré par les forces magnétiques, il est nécessaire d'avoir aussi un champ magnétique poloïdal.
Dans un tokamak, ce champ est principalement produit par le courant dans le plasma lui-même, ce courant circulant dans la direction toroïdale.
Ces courants et champs sont illustré dans la \autoref{subfig:schematic_tokamak_1}.
La combinaison du champ toroïdal \( B_\phi \) et du champ poloïdal \( B_\theta \) donne des lignes de champ magnétique qui ont une trajectoire hélicoïdale autour du tore, comme montré dans la \autoref{subfig:schematic_tokamak_2}.
Le champ magnétique toroïdal est produit par les courants dans les bobines liant le plasma, comme montré dans la \autoref{fig:schematic_tokamak_3}.

\begin{wrapfigure}{r}{0.2\textwidth}
    \centering
    % Réduction du retrait interne du wrap
    \vspace{-10pt}
    
    \includegraphics[width=\linewidth]{{figures/chapter1/schematic_tokamak_3.pdf}}

    \caption{Le chamm magnétique toroïdal est produit par le courant dans des bobines externes. }
    \vspace{-10pt}
    \label{fig:schematic_tokamak_3}
\end{wrapfigure}

La pression du plasma est le produit de la densité de particules et de la température.
Comme la réactivité du plasma augmente avec ces deux quantités, dans un réacteur la pression doit être suffisamment haute.
La pression qui peut être confinée est déterminée par des considérations de stabilité et augmente avec la force du champ magnétique.
Cependant, l'intensité du champ magnétique toroïdal est limitée par des facteurs technologiques.
Dans des expériences en laboratoire avec des bobines en cuivre, les besoins de refroidissement et les forces magnétiques mettent une limite sur le champ magnétique qu'elles peuvent atteindre.
De plus, dans un réacteur, les pertes par effet Joule dans des bobines normales sont inacceptables et des bobines supraconductrices sont donc nécessaires.
Il y a une perte de supraconductivité au-delà d'un champ magnétique critique, et cela est une autre limitation.
Avec les technologies actuelles, il semble probable que le champ magnétique maximal au niveau des bobines soit limité autour de \( \SI{12}{\tesla} \), mais des conducteurs améliorés avec des champs allant jusqu'à \( \SI{16}{\tesla} \) sont également envisagés\footnote{En 2021, des prototypes de bobines avec un champ de \( \SI{20}{\tesla} \) ont été testés avec succès en laboratoire.}.
Le champ toroïdal maximal se trouve sur le côté intérieur de la bobine de champ toroïdal.
Puisque le champ magnétique toroïdal est inversement proportionnel au grand rayon, le champ au centre du tokamak serait autour de \( \qtyrange[range-units=single,range-phrase=-]{6}{8}{\tesla} \)\footnote{Avec des bobines de \( \SI{20}{\tesla} \), le champ au centre du tokamak SPARC est envisagé d'être de \( \SI{12}{\tesla} \).}.
Le champ toroïdal dans les grands tokamaks actuels est quelque peu en dessous de cette valeur.

Pour un champ magnétique toroïdal donné, la pression du plasma qui peut être stablement confinée augmente avec le courant du plasma jusqu'à une valeur limite.
Les champs magnétiques poloïdaux résultants sont typiquement un ordre de magnitude plus faible que le champ toroïdal.
Dans les grands tokamaks actuels, des courants de plusieurs mégaampères sont utilisés, un courant de \( \SI{7}{\mega\ampere} \) ayant été produit dans le tokamak JET.
Avec des hypothèses conservatrices, un réacteur devrait nécessiter un courant de \( \qtyrange[range-units=single,range-phrase=-]{20}{30}{\mega\ampere} \).
Des avancées technologiques et des connaissances pourraient mener à des valeurs plus basses.

\begin{figure}

\centering
    % Réduction du retrait interne du wrap
    \vspace{-10pt}

    \begin{subfigure}[b]{0.4\textwidth}
        \centering
        \includegraphics[width=\linewidth]{figures/chapter1/schematic_tokamak_4.pdf}
        \caption{}
        \label{subfig:schematic_tokamak_4}
    \end{subfigure}
    \hspace{20pt}
    \begin{subfigure}[b]{0.53\textwidth}
        \centering
        \includegraphics[width=\linewidth]{{figures/chapter1/schematic_tokamak_5.pdf}}
        \caption{}
        \label{subfig:schematic_tokamak_5}
    \end{subfigure}

    \caption{(a) Un changement de flux à travers le tore induit un champ électrique toroïdal qui entraine le courant toroïdal. (b) La variation de flux est produite par l’enroulement primaire, souvent à l’aide d’un noyau de transformateur.}
    \vspace{-10pt}
    \label{fig:schematic_tokamak_4_5}

\end{figure}

Dans les expériences actuelles, le courant du plasma est entraîné par un champ électrique toroïdal induit par l'action du transformateur, au cours de laquelle un changement de flux à travers le tore est généré, comme illustré par la \autoref{subfig:schematic_tokamak_4}.
La variation de flux est provoquée par un courant circulant dans une bobine primaire enroulée autour du tore, comme montré dans la \autoref{subfig:schematic_tokamak_5}.
Bien que non essentiel, un noyau en fer de transformateur est souvent utilisé, celui-ci réduit les besoins en alimentation électrique et présente l'avantage de limiter les champs magnétiques de fuite.

\vspace{2\baselineskip}
\begin{wrapfigure}{r}{0.4\textwidth}
	\centering 
    \includegraphics[width=\linewidth]{{figures/chapter1/schematic_tokamak_6.pdf}}

    \caption{L'arrangement des bobines dans un tokamak. }
    \vspace{-10pt}
    \label{fig:schematic_tokamak_6}
\end{wrapfigure}

Il y a des avantages pour le confinement et la pression atteignable avec les plasmas qui sont allongés verticalement.
Le contrôle de la forme demandes des courants toroïdaux supplémentaires.
De plus, de tels courants sont nécessaires pour contrôler la position du plasma.
Ces courants toroïdaux sont transportés par des bobines judicieusement placées.
Le système complet de bobines toroïdales et poloïdales est illustré dans la \autoref{fig:schematic_tokamak_6}.

Les mécanismes limitant le confinement du plasma dans les tolomaks ne sont pas tous compris.
Cependant l'amélioration attendue du confinement avec la taille a été observée expérimentallement.
Typiquement, les meilleurs temps de confinement de l'énergie pour les tokamaks existants sont autour de \( \frac{1}{2} r_p^2 \) où \( r_p \) est le petit rayon moyen du plasma.
Un temps de confinement de l'énergie supérieur à une seconde a été obtenu dans le tokamakJET.
Il est établi que le temps de confinement de l'énergie augmente avec le courant du plasma mais, malheureusement, décroît avec une augmentation de la pression du plasma.

Les plasmas de tokamak sont chauffés à des températures de quelques \( \SI {}{\kilo\electronvolt}\) par le chauffage ohmique du courant du plasma.
Les températures requisent de \( \gtrsim \SI{10}{\kilo\electronvolt} \) sont alors atteintes avec du chauffage additionnel par des faisceaux de particules ou par ondes électromagnétiques.

Les plasmas de tokamak actuels ont des densités de particules de l'ordre de \(  \qtyrange[range-units=single,range-phrase=-]{1e19}{1e20}{\per\meter\cubed}  \).
C'est un facteur \( 10^6 \) plus bas que la densité de l'atmosphère.
Le plasma est contenu dans une chambre à vide et pour minimiser la présence d'impuretés, des pressions résiduelles faibles doivent être maintenues.

\begin{figure}
	\centering
        \begin{subfigure}[b]{0.5\textwidth}
        \centering
        \includegraphics[width=\linewidth]{figures/chapter1/schematic_tokamak_7.pdf}
        \caption{}
        \label{subfig:schematic_tokamak_7}
    \end{subfigure}
    \hspace{2pt}
    \begin{subfigure}[b]{0.38\textwidth}
        \centering
        \includegraphics[width=\linewidth]{{figures/chapter1/schematic_tokamak_8.pdf}}
        \caption{}
        \label{subfig:schematic_tokamak_8}
    \end{subfigure}
    
    \caption{La séparation du plasma des murs de la chambre à vide par (a) un limiteur et (b) un diverteur.}
    \label{fig:schematic_tokamak_7_8}
\end{figure}

Les impuretés dans le plasma donnent lieu à des pertes par radiation et diluent le carburant.
La restriction de leur entré dans le plasma jooue donc un rôle fondamental dans la bonne opération des tokamaks.
Cela demande une séparation du plasma des murs de la chambre à vide.
Deux techniques sont utilisées actuellement.
La première est de définir une limite externe du plasma avec un limiteur matériel comme montré dans la \autoref{subfig:schematic_tokamak_7}.
La deuxième est de garder les particules loin des murs de la chambre à vide par une modification du champ magnétique pour produire un diverteur magnétique comme montré dans la \autoref{subfig:schematic_tokamak_8}.

Un réacteur tokamak demandera des éléments additionnels dans la structure du tokamak et aura aussi besoin d'un moyen de convertir la puissance de fusion en électricité.
Ces éléments sont décris dans la \autoref{sec:tokamak_reacteur}.

\section{Tokamak réacteur}
\label{sec:tokamak_reacteur}

\subsubsection*{Structure d'un réacteur}

Un tokamak réacteur serait considérablement plus complèxe qu'un tokamak non-thermonucléaire.
La structure générale est illustré dans la \autoref{fig:schematic_tokamak_9} et est décrite ci-dessus.

Le plasma est entouré d'une couverture qui a trois rôles.
Premièrement, elle absorbe les neutrons de \( \SI{14}{\mega\electronvolt} \), transformant leur énergie cinétique en chaleur qui est ensuite évacuée par un fluide caloporteur approprié pour fournir la majeure partie de la puissance.
Deuxièmement, en absorbant les neutrons la couverture protège les bobines supraconductrices et les autres composants externes.
Troisièmement, la couverture permet la production de tritium pour alimenter le réacteur.
Pour ce faire, la couverture est constituée d'un composé de lithium comme le \( \text{Li}_2\text{O} \). 
Un tritium est produit dans chaque réaction neutron-lithium comme décrit dans la \autoref{sec:reserves_de_carburant}, mais il n'est pas possible de concevoir une couverture de manière à ce que tous les neutrons suivent cette réaction.
Pour contrecarrer ce déficit et obtenir un taux de génération supérieur à 1, il est nécessaire d'utiliser un multiplicateur de neutrons comme du beryllium ou du plomb.
Le flux de neutrons venant du plasma décroît avec la distance dans la couverture, une épaisseur de couverture de \( \qtyrange[range-units=single,range-phrase=-]{0.6}{1.0}{\meter} \) est suffisant pour absorber la majorité des neutrons.

\begin{figure}
	\centering
    % Minipage pour la légende (à gauche)
    \begin{minipage}[l]{0.35\textwidth}
        \caption{Arrangement des composants principaux d'un tokamak réacteur conceptuel.}
        \label{fig:schematic_tokamak_9}
    \end{minipage}\hfill
    % Minipage pour l'image (à droite)
    \begin{minipage}[c]{0.55\textwidth}
        \centering
        \includegraphics[width=\textwidth]{figures/chapter1/schematic_tokamak_9.pdf}
    \end{minipage}
\end{figure}

Le flux d'énergie des neutrons qui passe à travers le mur externe de la couverture doit être réduit d'un facteur \( 10^6-10^7 \) avant d'atteindre les bobines supraconductrices afin d'éviter à la fois les dégâts des radiations et le réchauffement des bobines.
Cette protection est obtenue en plaçant un bouclier d'une épaisseur d'environ \( \SI {1}{\meter}\) en un matériau à numéro atomique Z élevé comme de l'acier entre la couverture et les bobines.

Dans les tokamaks expérimentaux, le contact direct entre le plasma et le premier mur est évité soit grâce à un limiteur soit par un diverteur qui mène le champ magnétique loin de la surface du plasma vers une plaque éloignée du plasma comme décris dans la \color{red}{Section 9.10 (lien à rajouter quand elle sera faite)}\color{black}.
Dans un réacteur le flux de puissance thermique sur les surfaces sera nettement supérieur et il y aura un besoin de minimiser le flux d'impureté dans le plasma, en ajoutant à ça, sa plus grande flexibilité de design, les systèmes avec divertors sembleut donc favorisés.

Idéalement, le courant toroïdal dans le plasma serait continu dans le temps.
Cependant avec l'action du transformateur, le champ électrique moteur est induit par une augmentation du flux magnétique liant le tore, et cette augmentation ne peut se faire que durant un temps fini.
L'action du transformateur pourrait permettre des décharges d'une heure.
Bien que ce ne soit pas totalement satisfaisant, il est possible d'accepter une opération pulsée de ce type, à condition que la période d'arrêt soit suffisamment brève pour que le stress thermique répété, lié au refroidissement, soit acceptable.
La solution alternative est de faire en sorte qu'un courant continu soit généré par d'autres moyens qu'un champ électrique.
L'exigence en matière d'entraînement du courant est réduite par le fait que le plasma lui-même générera une partie du courant requis par un mécanisme appelé effet bootstrap.
Les courants restants seraient générés par l'injection de faisceaux de particules neutres ou des ondes électromagnétiques.
Ces méthodes d'entrainement du courant sont décrites dans la   \color{red}{Section 3.14 (lien à rajouter quand elle sera faite)}\color{black}.

\begin{figure}
        \centering
        \includegraphics[width=\textwidth]{figures/chapter1/schematic_tokamak_10.pdf}
        \caption{La puissance thermonucléaire abosrbée dans la couverture devra être converti en électricité par des moyens conventionnels.}
        \label{fig:schematic_tokamak_10}
\end{figure}

La chaleur quittant le plasma et produite dans la couverture, devra être extraite par un caloporteur liquide ou gazeux.
Elle sera ensuite transformée en électricité par des moyens conventionnels comme illustré dans la \autoref{fig:schematic_tokamak_10}.

\subsubsection*{Paramètres d'un réacteur}

Les informations suffisantes sont désormais disponibles grâce aux tokamaks expérimentaux pour permettre d'estimer les paramètres généraux pour un tokamak réacteur.
La taille et le courant du plasma requis pour un réacteur sont déterminés à partir de trois considérations.
La première est le temps de confinement de l'énergie.
Le temps de confinement doit être suffisamment long pour satisfaire le bilan de puissance pour un plasma auto-entretenu comme décrit dans la \autoref{sec:bilan_de_puissance}.
Cela nécessite que le plasma soit suffisamment grand, permettant l'amélioration de confinement qui vient de la capacité de transport du courant augmenté des plasmas plus grand.
Cependant, cela introduit une deuxième considération, l'instabilité issue des disruptions du plasma qui met subitement fin au plasma.
Pour une machine donnée, le courant du plasma doit être accompagné d'un grand champ magnétoqie toroïdal pour éviter les disruptions.
La troisième contrainte est que le champ magnétique toroïdal est lui-même limité à la fois par le champ critique permis par les supraconducteurs et par le stress magnétique sur les bobines.
Il y a quelques incertitudes associées avec chacun de ces facteurs mais il est néanmoins possible de faire des calculs approximatifs des paramètres requis.

Un plasma auto-entretenu doit satisfaire des conditions sur la densité, la température et du temps de confinement.
Comme montré dans la \autoref{sec:ignition}, ces conditions peuvent être résumé par l'inéquation

\begin{equation}
	\hat{n}\hat{T} \tau_E \gtrsim \SI{5e21}{\per\meter\cubed\kilo\electronvolt\second}
	\label{eq:inequation}
\end{equation}

Pour lier cette inéquation aux paramètres du plasma, une expression pour le temps de confinement de l'énergie est nécessaire.
Plusieurs formules empiriques ont été proposées et, comme décrit dans le \color{red}{chapitre 4 (lien à rajouter quand elle sera faite)}\color{black}, plusieurs modes d'opération ont été découvert.
Il est pratique de commencer cette discussion en présentant une des premières formules, qui a tenu l'épreuve du temps.
Cette formule, donné par Goldston, peut être approximé par

\begin{equation}
	\tau_E = \frac{I^2}{\overline{n T}} f \left(\frac{R}{a}, \frac{b}{a} \right),
	\label{eq:goldston}
\end{equation}

où \( f \) est le courant du plasma, \( n \) et \( T \) sont la densité et la température, \( R/a \) est le rapport d'aspect et \( b/a \) est l'élongation du plasma mesuré par le rapport entre la demi-hauteur et la demi-largeur.

Il y a des incertitudes dans la formule \ref{eq:goldston} au regard d'à la fois la magnitude et de la mise à l'échelle.
En particulier, des régimes qui ont un meilleur temps de confinement ont été trouvés.
De plus, à cause de la diversité limitée des géométrie des tokamaks expérimentaux, la dépendance de \( f \) sur le rapport d'aspect et l'élongation est incertaine.

La possibilité d'opérer dans un régime de confinement amélioré sera rendu possible en introduisant un facteur d'amélioration dans l'\autoref{eq:goldston}.
Cette équation est communément écrite en termes de la puissance de chauffage plutôt que de la température.
Donc, en équilibrant la puissance injectée \( P \) et les pertes \( 3 \overline{n T}/\tau_E \), l'\autoref{eq:goldston} peut être écrite de manière équivalente

\[
	\tau_E = (3f)^{1/2} \frac{1}{P^{1/2}}
\]

Il est habituel d'introduire un facteur d'amélioration \( H \) dans cette forme pour avoir

\begin{equation}
	\tau_E = H (3f)^{1/2} \frac{1}{P^{1/2}}
	\label{eq:facteur_amelioration}
\end{equation}

En revenant à la forme de l'\autoref{eq:goldston}, l'\autoref{eq:facteur_amelioration} devient

\[
	\tau_E = \frac{H^2 I^2}{\overline{n T}} f.
\]

Actuellement, dans les expériences  les rapports géométriques typiques sont \( R/a =3 \) et \( b/a = 5/3 \).
Pour ces valeurs \(f = \SI{2e6}{\per\meter\cubed\kilo\electronvolt\second\per\ampere\squared} \).
En utilisant cette valeur de \( f \) et en prenant des profils de densité et de température paraboliques, avec des valeurs maximales \( \hat{n} \) et \( \hat{T} \), de sorte que \( \overline{n T} = \frac{1}{3} \hat{n}\hat{T}\), donne

\[
	\hat{n}\hat{T}\tau_E = \SI{6e6}{} H^2I^2 \SI{}{\per\meter\cubed\kilo\electronvolt\second}.
\]

en éliminant \( \hat{n}\hat{T}\tau_E \) grâce à la relation \ref{eq:inequation}, le courant minimum dans un tokamak pour avoir un plasma auto-entretenu  est

\begin{equation}
	I = \frac{30}{H} \SI{}{\mega\ampere}
	\label{eq:minimum_courant}
\end{equation}

Donc, pour \( H = 1 \) le courant nécessaire est de \( \SI{30}{\mega\ampere} \).
Des facteurs \( H \) de \( 2-3 \) ont été obtenus expérimentallement, et s'ils peuvent être obtenus dans un réacteur sans effets délétères le courant nécessaire serait réduit d'autant.

Il a été trouvé expérimentallement que pour éviter les disruptions induites par le courant, le champ magnétique toroïdal doit être suffisamment grand pour satisfaire

\begin{equation}
	\frac{B_\phi}{B_{\theta s}} \gtrsim 2 \frac{R}{\overline{a}},
	\label{eq:eviter_disruption}
\end{equation}

où \( B_{\theta s} \) est le champ magnétique poloïdal moyen à la surface du plasma et \( \overline{a} \) est le rayon moyen du plasma.

En approximant la loi d'Ampère par 

\begin{equation}
	I = \frac{2 \pi}{\mu_0} \overline{a} B_{\theta s},
	\label{eq:approx_ampere}
\end{equation}

et en prenant \( \overline{a} = \left( ab\right)^{1/2} \), les relation \ref{eq:minimum_courant} à \ref{eq:approx_ampere} avec \( R/a = 3 \) et \( b/a = \frac{5}{3} \), donne l'approximation d'une condition pour un réacteur

\begin{equation}
	R B_\phi \gtrsim \frac{65}{H}\SI{}{\milli\tesla}.
	\label{eq:condition_reacteur}
\end{equation}

Le compromis entre la taille et le chqmp magnétique st déterminé par des considerations techniques et économiques.
Pour de basses valureurs de \( B_\phi \), le coût global décroît quand \( B_\phi \) augmente, mais cette dépendance est limité par la difficulté à obtenir des \( B_\phi \) élevé dans des matériaux supraconducteurs.
Avec les technologies actuelles, le \( B_\phi \) optimal est autour de \( \SI{12}{\tesla} \) au côté interne de la bobine de champ toroïdal, et en prenant la dépendance en \( 1/R \) ça implique un champ d'approximativement \( \SI{6}{\tesla} \) au centre du plasma.

En prenant \( B_\phi = \SI{6}{\tesla} \), la condition \ref{eq:condition_reacteur} demande un grand rayon de \( 11/H \) et pour \( H \) entre \( 1 \) et \( 2 \) cela donne \( R \) entre \( \SI{11}{\meter} \) et \( \SI{5.5}{\meter} \).
Les valeurs de \( a \) se trouvent environ dans l'intervalle \(  \qtyrange[range-units=single,range-phrase=-]{3.5}{2}{\meter} \) et pour \( b \) dans l'interalle \(  \qtyrange[range-units=single,range-phrase=-]{6}{3}{\meter} \)
\footnote{Comme vu dans la \autoref{sec:tokamaks}, il serait possible d'avoir \( B_\phi = \SI{12}{\tesla} \), dans ce cas, le grand rayon serait de \( 5.5/H \), donc pour un \( H \) entre \( 1 \) et \( 2 \) cela donne \( R \) entre \( \SI{5.5}{\meter} \) et \( \SI{2.75}{\meter} \). Avec \( a \) entre \( \qtyrange[range-units=single,range-phrase=-]{2}{1}{\meter} \) et \( b \) entre \( \qtyrange[range-units=single,range-phrase=-]{3.5}{1.75}{\meter} \)}

Cependant, d'autres considérations entrent en jeu pour choisir les meilleurs paramètres du plasma.
La volonté d'éviter le fonctionnement en régime pulsé en utilisant le courant bootstrap mène à un design avec un plus grand rapport d'aspect.
La condition \ref{eq:eviter_disruption} pour éviter les disruptions implique un plus grand champ magnétique toroïdal.
Cela demande alors le développement d'aimants supraconducteurs avancés.
Il est aussi préférable d'avoir un courant plus faible pour augmenter la fraction que représente le courant bootstrap.
La possibilité d'avoir un courant plus faible dépend des améliorations de confinement, soit par l'amélioration du facteur \( H \), soit avec une amélioration avec un rapport d'aspect plus grand.

\subsubsection*{Puissance du réacteur}

La densité de puissance thermonucléaire pour des densités de deutérium et de tritium égales est donné par l'\autoref{eq:puisance_thermonucleaire_2} et puissance totale du réacteur est donc

\begin{equation}
	P = \frac{\pi}{2} \mathscr{E}\int n^2 \sigmav R \dd S
	\label{eq:puissance_thermonucleaire_totale}
\end{equation}

où \( \dd S \) est un élément de surface d'une section poloïdale.
Cette puissance peut-être calculée numériquement pour n'importe quel cas particuler, mais il est instructif d'obtenir une expression analytique en faisant des approximations.
La géométrie est simplifiée en prenant \( R \) comme la valeur centrale, et le plasma allongé est représenté par un plasma circulaire équivalent avec un rayon \( \overline{a} = (ab)^{1/2}\).
L'\autoref{eq:puissance_thermonucleaire_totale} devient alors

\begin{equation}
	P = \pi^2 R \mathscr{E} \int_0^{\overline{a}} n^2 \sigmav r \dd r
	\label{eq:puissance_totale_simplifie}
\end{equation}

Pour les températures d'intérêt, \( \sigmav \) est bien représenté par l'\autoref{eq:sigma_v_constant}.
En utilisant cette approximation  et en prenant un profil de pression de la forme

\begin{equation}
	n T = \hat{n} \hat{T} \left( 1 - \frac{r^2}{\overline{a}^2} \right)^v
	\label{eq:density_temperature_profil}
\end{equation}

l'\autoref{eq:puissance_totale_simplifie} donne

\[
	P = \frac{0.15}{2v + 1} R a b \left( \frac{\hat{n}}{10^{20}} \right)^2 \hat{T}^2 \SI{}{\mega\watt}, \quad \hat{T} \text{en} \SI{}{\kilo\electronvolt}.
\]

\begin{figure}
	\centering
    % Minipage pour la légende (à gauche)
    \begin{minipage}[l]{0.35\textwidth}
        \caption{Graphique donnant la dépendance de la puissance thermonucléaire à la densité maximale \( \hat{n} \) et l'index de la forme du profil de densité \( v \) pour deux taille de réacteur avec \( \hat{T} = \SI{20}{\kilo\electronvolt} \).}
        \label{fig:puissance_dependance}
    \end{minipage}\hfill
    % Minipage pour l'image (à droite)
    \begin{minipage}[c]{0.55\textwidth}
        \centering
        \includegraphics[width=\textwidth]{figures/chapter1/power_density.pdf}
    \end{minipage}
\end{figure}

Les graphiques de cette puissance en fonction de \( \hat{n}^2/(2v+1) \) sont montrés dans la \autoref{fig:puissance_dependance} pour les cas \( R = \SI{9}{\meter}, a = \SI{3}{\meter}, b=\SI{5}{\meter} \) et \( R = \SI{6}{\meter}, a = \SI{2}{\meter}, b=\SI{3.3}{\meter} \), en prenant \( \hat{T} = \SI{20}{\kilo\electronvolt} \).
On voit que la puissance produite est de l'ordre du \( \SI{}{\giga\watt} \).
La valeur exacte est sensible au profil de pression et de densité, qui sont tous deux contraints par des considérations de stabilité, comme discuté dans les \color{red}{chapitres 6 et 7 (lien à rajouter quand elle sera faite)}\color{black}.

\subsubsection*{Impuretés}

Un des problèmes les plus importants pour un réacteur est la présence d'impuretés dans le plasma.
Il y a deux types d'impuretés.
Premièrement, les impuretés provenant des ions issus des surfaces solides et deuxièmement, les particules-\( \alpha \), \( ^4\text{He} \), qui viennent des réactions de fusion..
Le design du réacteur doit minimiser l'arrivée d'impuretés, mais, clairement, la présence de particules-\( \alpha \) est intrinsèque au fonctionnement du tokamak.
Il faudrait alors que les "cendres d'hélium" aient un temps de confinement suffisamment court dans le plasma.

Les impuretés issues des parois produisent des espèces partiellement ionisées qui mènent à des pertes d'énergie du plasma par rayonnement.
Ceci est expliqué quantitativement dans la \color{red}{section 4.25 (lien à rajouter quand elle sera faite)}\color{black}.
Il y a, de plus, le problème de la dilution du combustible.
Chaque ion est associé à un nombre d'électrons libres déterminé par son degré d'ionisation.
Ces électrons, issus des impuretés, auront la même température que le plasma, et, pour une energie de confinement du plasma donnée, peut être vu comme ayant déplacé les ions du carburant de deutérium et de tritium.
Un atome métallique lourd pour libérer des dizaines d'électrons dans le centre du plasma.

Le contrôle de l'arrivée d'impureté dépend du bon design à la fois de la structure magnétique et aussi des matériaux de surface recevant l'énergie en provenance du plasma.
Actuellement, on estime que cela demande un divertor magnétique.
Le but du divertor est de guider les particules sortantes vers une surface cible bien séparée du plasma et de limiter le reflux d'impuretés.
Un problème difficile avec les divertors est de limiter la densité de puissance arrivant sur la surface cible.
Ceci est nécessaire pour éviter les hautes températures de surface qui peuvent mené à la fusion de la surface ou ç la libératio,  d'impuretés catastrophiques par évaporation ou d'autres procedés.

\section{Réserves de combustible}
\label{sec:reserves_de_combustible}

Il y a deux questions fondamentales pour évaluer la disponibilité des réserves en combustible pour un réacteur à fusion.
La première concerne la taille des réserves naturelles pour le combustible de base et la deuxième, est le coût de production.
Pour traiter complètement le sujet il faudrait regarder les rézerves en fonction du coût mais dans le contexte actuel ce n'est pas nécessaire.
Les réponses à ces questions doivent être mises en relation avec les besoins en électricité et le coût de cet électricité.

La consommation mondiale d'énergie primaire est autour de \( \SI{6.5e11}{\giga\joule} \).
La demande mondiale d'électricité est autour de \( \SI{1e11}{\giga\joule} \).
Le prix moyen de l'électricité sur les marché de gros est autour de \( 16 \text{€} \) par \( \SI{}{\giga\joule} \).

L'abondance naturelle du deutérium parmi l'hydrogène est d'une part par \( 6700 \). La masse d'eau dans les océans est de \( \SI{4e16}{\kilogram} \).
Dans des réacteurs de D-T avec une efficacité de \( 1/3 \) cela permettrait une production de \( \left( \SI{4e16}{}/m_{\text{D}} \right) \left( 17.6/3 \right) \SI{}{\mega\electronvolt} \), ou \( \SI{e22}{\giga\joule} \) (el).
Ce qui correspond à environ \( 10^{11} \) années de la consommation d'électricité mondiale.
Il n'y a clairement aucun problème avec les ressources en deutérium.

Le coût du deutérium est de l'ordre de \( 1 \)€ par gramme.
Un gramme de deutérium produit \( \left( \SI{e-3}{}/m_{\text{D}} \right) \left( 17.6/3 \right) \SI{}{\mega\electronvolt} \), ou \( \SI{300}{\giga\joule} \) (el).
La contribution du deutérium dans le coût est donc de \( 0.003 \)€ par \( \SI{}{\giga\joule} \) (el), ce qui est négligeable par rapport au prix de l'électricité de \( 16 \text{€} \) par \( \SI{}{\giga\joule} \).

Pour le tritium la situation est plus complexe.
Le tritium a une demi-vie de seulement \( 12.3 \) abs et est pratiquement inexistant  dans la nature.
Cependant, le tritium peut-être obtenu en utilisant la fission induite par neutron du lithium.

\begin{align*}
	&^6\text{Li} + \text{n} \rightarrow \text{T} + {}^4\text{He} + \SI{4.8}{\mega\electronvolt}\\
	&^6\text{Li} + \text{n} \rightarrow \text{T} + {}^4\text{He} + \text{n} - \SI{2.5}{\mega\electronvolt}
\end{align*}

Les abondances sont de \( \SI{7.4}{\percent} {}^6\text{Li} \) et \( \SI{92.6}{\percent} {}^7\text{Li} \).
Avec ce système, les combustibles de base sont le deutérium et le lithium.

C'est principalement le \( ^6\text{Li} \) qui est consommé dans la couverture, et l'énergie potentiellement disponible dans \( \SI{1}{\kilogram} \) de \( ^6\text{Li} \) est d'environ \( \SI{22}{\mega\electronvolt} / m_{^6\text{Li}} \) ou \( \SI{3e5}{\giga\joule} \).
Avec une efficacité de conversion en électricité de \( 1/3 \) cela donne \( \SI{1e5}{\giga\joule}(\text{el}) \).
Donc, en prenant en compte l'abondance naturelle du \( ^6\text{Li} \), \( \SI{1}{\kilogram} \) de lithium fournirait autour de \( \SI{7e3}{\giga\joule}(\text{el}) \).
Le coût du lithium est de l'ordre de \( 20\text{€} \) par \( \SI{}{\kilogram} \) et donc la contribution du lithium dans le coût du combustible est de ùoins de \( 0.001\text{€} \) par \( \SI{}{\giga\joule}(\text{el}) \).
Ce coût est très faible comparé au prix de l'électricité et même une forte augmentation du prix du lithium resterait acceptable avant que cela ne devienne un facteur important.

Les réserves mondiales de lithium au prix actuel sont estimés à \( \SI{30e9}{\kilogram} \).
Cette quantité de lithium pourrait produire \( \left( \SI{30e9}{\kilogram} \right) \left( \SI{7e3}{\giga\joule (el)\per\kilogram)} \right) \simeq \SI{2e14}{\giga\joule (el)} \) ce qui est de l'ordre de \( 300 \) ans de la demande d'énergie primaire actuelle.
Il n'est donc pas iiraisonable de supposer que les réserves mondiales de lithium pourrait permettre de générer un centaines de fois la demande annuelle totale d'énergie, même si on suppose un prix très bas du lithium.

\begin{table}

	\centering

	\caption{Estimation des réserves mondiales d’énergie. Les valeurs indiquées sont uniquement approximatives, elles dépendent notamment des prix du marché et restent incertaines en raison d’une prospection géologique incomplète.}

	\begin{tabular}{>{\raggedright\arraybackslash}p{0.35\linewidth} >{\raggedleft\arraybackslash}p{0.15\linewidth} >{\raggedright\arraybackslash}p{0.2\linewidth}}
	\hline
	& Gigajoules (\( 10^9 \) joules) & Années de consommation annuelle d'énergie\\
	\hline
	Consommation annuelle d'énergie primaire & \( \SI{6.5e11}{} \) & \( 1 \) an\\
	\textbf{Ressources} & & \\
	Charbon & \( \SI{2.3e14}{} \) & \( 350 \) ans  \\
	Pétrole & \( \SI{9e12}{} \)  & \( 14 \) ans \\
	Gaz fossile & \( \SI{8e12}{} \)  & \( 12 \) ans  \\
	Uranium~235 & \( \SI{4e12}{} \)  & \( 6 \) ans  \\
	Uranium~238 et Thorium~232 (surgénération) & \( \SI{1e15}{} \)   & \( 1745 \) ans  \\
	Lithium (fusion D-T) &  &   \\
	 \quad Terrestre & \( \SI{2e14}{} \)  & \( 323 \) ans  \\
	 \quad Océans & \( \SI{1e18}{} \)  & \( \SI{2e6}{} \) ans  \\	
	\end{tabular}
	
	\label{tab:ressources}

\end{table}

Pour comparer ces chiffres avec l'énergie disponible grâce à d'aures combustibles, le \autoref{tab:ressources} donne un résumé des estimations des réserves de ces différentes ressources.

\FloatBarrier
\section{Économie des tokamaks}
\label{sec:economie_tokamak}

L’objectif de la recherche sur les tokamaks est de construire un système de production d’électricité fiable. 
À mesure que l’on s’en approche, les questions économiques deviennent de plus en plus importantes.
La question ultime est de déterminer si un réacteur tokamak sera compétitif par rapport aux autres systèmes de production d’électricité, en particulier ceux basés sur des combustibles fossiles ou fissiles. 
Cela soulève la question du coût d’un tel réacteur et de son environnement économique futur.
Il n’existe pas de réponse définitive aujourd’hui, mais il est possible d’aborder ces problématiques.

Le coût d’une centrale électrique comporte deux composantes principales.
La première est la production de chaleur, qui dépend du combustible utilisé.
La seconde est la conversion de cette chaleur en électricité, réalisée par un ensemble de turbines et de générateurs.
Dans le cas d’une centrale à fusion, le coût du réacteur lui-même domine largement par rapport à celui d’une centrale conventionnelle.

\begin{table}

\centering

\caption{Composition des coûts possibles pour un tokamak réacteur. Les coûts indirects proviennent principalement des intérêts, et pour les centrales conventionnelles le plus grand coût est celui des turbogénérateurs. Le coût de la couverture est celui de la première installation, plusieurs remplacements seraient nécessaires dans la vie d'un réacteur.}

\begin{tabular}{l c}
\hline
\textbf{Coûts totaux} & \\
\tikzmark{CDhaut} Coûts directs & \( \SI{65}{\percent} \) \\
Coûts indirects & \( \SI{25}{\percent} \) \\
Marge pour aléas & \( \SI{10}{\percent} \) \\
\cline{2-2} & \( \SI{100}{\percent} \) \\
\tikzmark{CDbas} \textbf{Coûts directs} & \\
\tikzmark{reacteur} Réacteur & \( \qtyrange[range-units=single,range-phrase=-]{50}{60}{\percent} \) \\
 Centrale conventionnelle &  \( \qtyrange[range-units=single,range-phrase=-]{35}{30}{\percent} \) \\
 Structures & \( \qtyrange[range-units=single,range-phrase=-]{15}{10}{\percent} \) \\
\cline{2-2} & \( \SI{100}{\percent} \) \\
\tikzmark{coutreacteur}\textbf{Coûts du réacteur} & \\
Bobines & \( \SI{30}{\percent} \) \\
Bouclier & \( \SI{10}{\percent} \) \\
Couverture & \( \SI{10}{\percent} \) \\
Échangeur de chaleur & \( \SI{15}{\percent} \) \\
Chauffage externe & \( \SI{15}{\percent} \) \\
Autres composants & \( \SI{20}{\percent} \) \\
\cline{2-2}  & \( \SI{100}{\percent} \) \\
\hline
\end{tabular}

\begin{tikzpicture}[remember picture, overlay]
    \draw[->, thick]
        ([yshift=1mm]pic cs:reacteur) to [out=180, in=180] ([yshift=1mm]pic cs:coutreacteur);
\end{tikzpicture}

\begin{tikzpicture}[remember picture, overlay]
	\draw[->, thick]
        ([yshift=1mm]pic cs:CDhaut) to [out=180, in=180] ([yshift=1mm]pic cs:CDbas);
\end{tikzpicture}

\label{table:couts}
\end{table}

Le coût des composants est fortement corrélé à la taille, et donc aux dimensions du réacteur.
Pour un réacteur donné, le volume du blanket est déterminé par l’épaisseur nécessaire pour arrêter les neutrons.
Pour les bobines, la taille dépend de la force magnétique associée aux courants qui les parcourent.
La chambre à vide doit résister à la pression atmosphérique ainsi qu’aux forces induites par d’éventuels changements rapides du courant plasma.
S’ajoutent enfin les coûts liés aux générateurs nécessaires pour alimenter les bobines et au système de chauffage du plasma.
Le \autoref{table:couts} présente une estimation actuelle de la répartition de ces coûts.

Pour pouvoir comparer le coût de l'électricité produite par un tokamak avec celle produite par d'autres systèmes, il est nécessaire de connaître le coût du réacteur.
Des estimations des coûts ont été faites pour plusieurs designs de réacteur avec une variété de résultats.
Celles-ci se situent généralement entre \( 1 \) et \( 3 \) fois le coût d'un  réacteur à fission comparable.
Pour faire ces estimations il y a des attentes de progrès technologiques et aussi un sur-optimisme compréhensible.

Une autre information est donnée par le projet ITER ou International Thermonuclear Experimental Reactor (Réacteur Thermonucleaire Expérimental International) qui est conçu pour atteindre l'ignition.
Son coût estimé était de \( \approx 6 \) milliards de \$(1989).
Puisque le coût est lié à un projet existant, il est raisonnable de le prendre au sérieux \footnote{Le coût d'ITER a dépassé \( 25 \) milliards de \$(2018).}.
Un réacteur thermonucléaire de cette taille produirait plusieurs gigawatts thermiques.

Le coût du capital actuel pour une centrale à fission est autour de 0.7 milliard de \$ par gigawatt thermique.
Cependant, une comparaison n'est pas évidente puisque ITER serait le premier tokamak réacteur expérimental tandis que les réacteurs à fission ont eu des décennies de développement.

L'autre problématique économique est l'environnement économique qu'un futur tokamak réacteur devra affronter.
Les prévisions ont tendance à placer le premier réacteur commercial dans la seconde moitié du \( 21^e \) siècle.
Il est dur de prédire comment la situation économique aura évolué d'ici là.
Les inquiétudes vis à vis de la pollution et des émissions de \( \text{CO}_2 \) des énergies fossiles et les possibles accidents nucléaires et la production de déchets radioactifs des centrales à fission pourraient rendre ces centrales innacceptable ou le coût des modifications nécessaire pour les rendre acceptables rendraient les réacteurs à fusion  plus compétitifs.

Peut-être qu'une vue raisonnable de la situation est qu'un réacteur tokamak semble coûteux, mais étant donné le faible prix d'un programme de développement d'un tokamak par rapport aux dépenses mondiales de production d'électricité, les bénéfices possibles peuvent justifier ce programme.
Il est bon de rappeler que notre incapacité  à prévoir l'avenir est bien établie.

\FloatBarrier
\section{Recherche sur les tokamaks}
\label{sec:recherhce_tokamak}

Le mot \textit{tokamak} a deux étimologies débattues vennant du russe, soit 
%тороидальная камера с магнитными magnetic катушками (
"toroidalnaïa kamera s magnitnymi katouchkami' Chambre toroïdale avec bobines magnétiques%)
, soit 
%тороидальная камера с аксиальным магнитным полем (
"toroidalnaïa kamera s aksial'nym magnitnym polem" chambre toroïdale avec un champ magnétique axial%)
.

Le concept du tokamak a été inventé en Union soviétique, les premiers développements ayant commencé dans les années 1950.
À cette époque, la recherche sur la configuration de pincement toroïdal a été fortement étudiée au Royaume-Uni et aux États-Unis.
L'avantage du tokamak vient simplement de la stabilité augmentée grâce au champ magnétique toroïdal plus important.

Le développement réussi du tokamak est principalement du fait de l'attention particulière à la réduction des impuretés et la séparation entre le plasma et la chambre à vide grâce à un limiteur.
Dans les années 1960, cela a mené à des plasma purs comparativement avec des températures d'électrons d'environ \( \SI{1}{\kilo\electronvolt} \).
En 1970, ces résultats sont généralement acceptés et considérés comme significatifs.

Les premiers tokamaks avaient des temps de confinement de l'énergie de quelques millisecondes et des températures d'ion de quelques centaines d'\( \SI{}{\electronvolt} \).
Dans les années 1970, la priorité a été de savoir si ces conditions pouvaient être améliorées.
Cette tache a été prise en charge et de nombreux tokamaks ont été construits dans plusieurs pays.

Il est rapidement devenu évident que le transport est anormal.
Les pertes d’énergie des électrons dépassent le transport collisionnel d’environ deux ordres de grandeur, et les pertes d’énergie des ions sont également plusieurs fois supérieures au transport collisionnel.
Ainsi, bien que la théorie collisionnelle prédise que les pertes d’énergie devraient être dominées par la conduction thermique ionique, en pratique les énergies des électrons et des ions sont généralement comparables.
Donc même si la théorie collisionnelle prédit que le temps de confinement collisionnel décroît comme \( 1/n \), où \( n \) est la densité d'électrons, expérimentalement le temps de confinement augmente avec \( n \).

Tandis que les tokamaks plus grands ont été construits, les améliorations attendues du confinement ont été atteintes et des temps de confinement de \( \SI{100}{\milli\second} \) ont été obtenues dès 1980.
Puisqu'un temps de confinement de l'ordre de la seconde est nécessaire pour un réacteur, il était important de faire la meilleure estimation des conditions nécessaires pour cela.
Le modèle le plus simple compatible avec les résultats expérimentaux donnait \( \tau_E \approx n a^2 \), où \( a \) est le petit rayon du plasma, et une extrapolation des données expérimentales indiquait qu'un petit rayon \( \sim \SI{2}{meter} \) pourrait être suffisant pour un réacteur.

Ce pré-requis sur la taille du plasma semblait acceptable et la conception de grands tokamaks expérimentaux ont été fait dans plusieurs pays.
Le plus grand étant l'expérience collaborative du Joint European Torus (Torus européen commun) JET, ayant un petit rayon de \( \SI{1.5}{\meter} \).

L'autre tâche des années 1970 était l'investigation du chauffage des ions à haute températures.
Les premières expériences utilisaient uniquement le chauffage ohmique du plasma par le courant toroïdal.
Malheureusement, le chauffage ohmique perd en efficacité à haute température car la résistance électrique du plasma chute quand la température des électrons augmente, évoluant comme \( T_e^{3/2} \).
La première méthode de chauffage additionnel à fonctionner avec succès était l'injection de particules neutres.
Des atomes neutres à haute vitesse sont injectés dans le plasma et donnent leur énergie par des collisions.
Une autre méthode de chauffage est de lancer des ondes à fréquences radio (RF) dans le plasma de façon à ce qu'elles soient absorbées à des surfaces de résonnances à l'intérieur du plasma.
Plusieurs fréquences ont été utilisées mais la plus prometteuse était le chauffage par résonnance cyclotron des ions.
L'injection de neutres et le chauffage RF ont atteint des températures de plusieurs \( \SI{}{\kilo\electronvolt} \) au début des années 1980.

Les plus hautes températures produites par les chauffages additionnels ont permis de clarifier la dépendance du temps de confinement à la température.
Une prédiction théorique basée sur le transport d'énergie par de simples collisions donnerait un confinement qui se dégraderait avec l'augmentation de la densité mais s'améliorerait avec une température plus grande.
Malheureusement, dans ces plasmas plus chaud, le temps de confinement a été trouvé de décroitre avec la densité et la température.
Lorsque la puissance de chauffage a été appliquée au plasma, le temps de confinement a chuté.
Ces résultats étaient décevant mais n'ont pas présenté un obstacle fondamental.
En utilisant des résultats empiriques donnés par une variété d'expériences, le temps de confinement a été prédit de s'améliorer avec la taille du tokamak et son courant du plasma plus grand.

Une heureuse découverte faite par le tokamak ASDEX était l'existence de modes d'opération du plasma qui ont un plus long temps de confinement que l'opération normale.
Ce mode est appelé H-mode.
La transition à ce mode de confinement amélioré a été trouvé dans un plasma avec un divertor et à un niveau de chauffage du plasma suffisant.
La transition apparaît brusquement et semble être associée à un confinement amélioré au bord du plasma.

Un problème qui est apparent depuis les premières expériences est l'apparition d'instabilités.
Parmi celles-ci, la plus dangereuse sont\ les disruptions, dans lesquelles l'instabilité cause une perte rapide d'énergie du plasma et la fin de la décharge.
Durant ces événements des courants puissants sont induits dans la chambre à vide et produisent des forces importantes sur la chambre.
Les disruptions réduisent sérieusement le régime de fonctionnement du tokamak, limitant à la fois la densité et le courant du plasma pour un champ magnétique toroïdal donné.
Une autre instabilité qui limite la densité de courant au centre du plasma est l'oscillation de relaxation appellé instabilité en dents de scie ("sawtooth").
Ces oscillations causent des pertes répétées d'énergie dans la région centrale.

Un type supplémentaire d'instabilité limite le rapport de pression du plasma par densité d'énergie magnétique.
Le rapport de ces deux quantitées est appelé \( \beta \) et la restriction imposée par l'instabilité est appelée limite \( \beta \).
La valeur de \( \beta \) a été augmentée jusqu'à cette limite dans plusieurs tokamaks.
En modifiant les conditions de fonctionnement, la limite \( \beta \) peut être augmentée et une valeur de \( \beta \) supérieure à \( \SI{10}{\percent} \) a été obtenue dans le tokamak DIII-D.
Les valeurs de \( \beta \) requises pour un réacteur sont un peu inférieures à cela, accentuant la pensée que la limite \( \beta \) ne sera pas un problème sérieux.

Au début des années 1980, deux grands tokamaks ont été construits, TFTR et JET.
Ils ont été conçus pour atteindre plusieurs mégaampères et produire un montant significatifs d'énergie thermonucléaire par des plasmas de deutérium-tritium.
Dans ces expériences, le plasma a été chauffé à des températures nécessaires pour un réacteur, des températures au-delà de \( \SI{30}{\kilo\electronvolt} \) ont été atteintes.
Dans JET, un temps de confinement supérieur à la seconde a été obtenu, et le régime de fonctionnement du tokamak a été étendu avec des courant du plasma jusqu'à \( \SI{7}{\mega\ampere} \) et des durées de décharge jusqu'à une minute.

En 1991, un autre palier a été franchi.
Des évaluations ont montré que les conditions du plasma atteintes dans les plasmas de deutérium de JET étaient telles que si le deutérium était remplacé par un mélange 50-50 de deutérium et de tritium, la puissance de réaction de fusion serait comparable à la puissance de chauffage injectée dans le plasma.
Même si cette expérience ne peut pas être mené à bien à cette époque à cause du niveau de radioactivité dans la structure du tokamak, il a été décidé de faire une expérience au tritium  préliminaire.
Une quantité controlée de tritium a été injecté dans un plasma de deutérium en tant que faisceau de neutres.
Cela a mené à une puissance de réaction de fusion supérieure à un mégawatt pour plus d'une seconde.
Des expériences suivantes utilisant un mélange 50-50 de deutérium et de tritium ont permis la production d'une puissance de fusion de plus de \( \SI{10}{\mega\watt} \) dans TFTR et de \( \SI{16}{\mega\watt} \) dans JET.

Il est clair que de nombreux problèmes de la recherche en fusion ont été surmontés dans des tokamaks.
Il est nécessaire de se touner vers l'avenir et la prochaine étape du développement des tokamaks.
Ce projet se concentrera sur les problèmes critiques et s'il y parvient, démentrera la faisabilité d'un tokamak réacteur.
Il semble probable que le premier tokamak à atteindre l'ignition sera un projet commun de plusieurs pays.
Il y a déjà une équipe de coopération internationale dont la tâche de concevoir un tokamak qui produira une puissance de fusion significative. Ce tokamak est ITER pour International Thermonuclear Experimental Reactor (Réacteur thermonucléaire expérimental international).

\clearpage
\twocolumn
\section*{Bibliographie}

Quelques livres ont été écrits avec la fusion thermonucléaire comme sujet principal.
De ceux-ci, les plus anciens ne sont plus à jour sur tout mais néanmoins, donnent toujours des informations et des points de vue utiles, et sont donc inclus dans la liste chronologique suivante:

\begin{itemize}
	\item Glasstone, S. et Loveberg, R.H. controlled thermonuclear reactions. Van Nostrand, Princeton, New Jersey (1960)
	\item Rose, D.J. et Clark, M. Plasmas and controlled fusion. MIT Press, Cambridge, Mass (1961)
	\item Artimovich, L.A. Controlled thermonuclear reactions. Gordon and Breach, New York (1974)
	\item Kammash, T. Fusion reactor physics. Ann Arbor Science, Ann Arbor, Michigan (1975)
	\item Teller, E. (ed). Fusion. Academic Press, New York (1981)
	\item Gill; R.D. (ed). Plasma physics and nuclear fusion research. (Culham Summer School on Plasma Physics). Academic Press, London (1981)
	\item Dolan, T.J. Fusion research. Pergamon Press, New York (1982)
	Gross, J. Fusion energy. Wiley, New York (1984)
	\item Miyamoto, K. Plasma physics for nuclear fusion, 2nd edn. MIT Press, Cambridge, Mass (1989)	
\end{itemize}

\subsubsection*{Réactions de fusion et fusion thermonucléaire}

Une compilation des sections efficaces et des taux de réaction est donnée par

\begin{itemize}
	\item Miley, G.H. Towner, H. et Ivich, N? Fusion cross-sections and reactivities. University of Illinois Nuclear Engineering Report COO-2218-17. Urbana, Illinois (1974)
\end{itemize}

\subsubsection*{Critère de Lawson}

\begin{itemize}
	\item Lawson, J.D. Some criteria for a power producing thermonuclear reactor. Proceedings of the Physical Society B70, 6 (1957).
\end{itemize}

La determination du critère est donnée dans la première édition de ce livre.

\subsubsection*{Bases des tokamaks}

Un des premiers papiers décrivant les principes de base et les premiers résultats est
\begin{itemize}
	\item Artsimovich, L.A. Tokamak devices. Nuclear Fusion, 12, 215 (1972)
\end{itemize}

\subsubsection*{Tokamak réacteur}

Un compte-rendu des réacteurs à fusion dont une discussion sur les tokamaks est donné dans l'article

\begin{itemize}
	\item Conn, R.W. Magnetic fusion reactors. Fusion (ed. E. Teller) Vol. 1, Academic Press, New York (1981)
\end{itemize}

Une description des réacteurs à fusion est inclue dans le chapitre 7 du livre
\begin{itemize}
	\item Dolan, T.J. Fusion research. Pergamon Press, New York (1982)
	Gross, J. Fusion energy. Wiley, New York (1984)
\end{itemize}

\subsubsection*{Réserves en combustible}

Les besoins mondiaux en énergie et en électricité sont donnés par
\begin{itemize}
	\item IEA, Global Energy Review 2025
\end{itemize}
Pour les réserves en pétrole et en gaz fossile
\begin{itemize}
	\item OPEC, Annual Statistical Bulletin 2025 
\end{itemize}
pour l'uranium
\begin{itemize}
	\item IAEA-NEA, Uranium 2020 ressources, production and demand
\end{itemize}
pour le lithium
\begin{itemize}
	\item US Geology survey.  Mineral Commodity Summaries 2025
\end{itemize}

\subsubsection*{Économies}

Les analystes de l'économie des tokamaks réacteurs semblent avoir un large éventail de vues.
Un compte rendu des potentiels réacteurs à fusion ainsi qu'une discussion des coûts de l'électricité possibles est donnné dans
\begin{itemize}
	\item Conn, R.W. et al. Economics, safety and environmental prospects of fusion reactors. Nuclear Fusion 30, 1919 (1990)
\end{itemize}
Un autre plus en détail est
\begin{itemize}
	\item Krakowski, R.A. et Delene, J.G. Connection between physics and economics for tokamak fusion power plants. J. Fusion Energy 7, 49 (1988)
\end{itemize}
Une vision moins optimiste est exprimée dans
\begin{itemize}
	\item Pfirsch, D. et Schmitter, K.H. Some critical observations on the prospects of fusion power. Fourth international conference on energy options. Londres, I.E.E. Conference publication N° 233, 350, I.E.E., Londres (1984)
\end{itemize}

\subsubsection*{Recherche sur les tokamaks}

Les chapitres 11 et 12 sont dédiés à la description d'expériences particulières et les références des articles utilisés y sont données.

Un livre résumant la recherche sur les tokamaks et donnant un bon point de vue de la physique sous-jacente est
\begin{itemize}
	\item Kadomtsev B.B. Tokamak plasma, a complex physical system. Institute of Physics Publishing, Bristol (1992)
\end{itemize}

La formule de Goldston est décrite dans

\begin{itemize}
	\item Goldston, R.J. Energy confinement scaling in tokamaks: some implications of recent experiments with ohmic and strong auxiliary heating. Plasma Physics and Controlled Fusion, 26, N° IA, 87 (1984)
\end{itemize}

\subsubsection*{Constantes physiques}

Une liste des constantes physiques fondamentales ou autres est disponible dans

\begin{itemize}
	\item Tables of physical and chemical constants, (initiallement par) G.W.C. Kaye et T.H. Laby. Longman, Londres
\end{itemize}

\onecolumn

\end{document}
