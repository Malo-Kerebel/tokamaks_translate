\documentclass[main.tex]{subfiles}

\begin{document}

\chapter{Fusion}

\section{Fusion and tokamaks}
\label{sec:fusion_et_tokamaks}

Si un noyau de deutérium fusionne avec un noyau de tritium, une parti cule \( \alpha \) est produite et un neutron est relaché.
Le réarrangement nucléaire  induit une réduction de la masse totale et, par conséquent, de l'énergie est relaché sous la forme d'énergie cinétique des produits de la réaction.
L'énergie relachée par chaque réaction est de \( \SI{17.6}{\mega\electronvolt} \).
En terme macroscopique, un seul \( \SI{}{\kilogram} \) de ce carburant relacherais \( \SI{e8}{\kWh} \) d'énergie et serait ce qu'une centrale de \( \SI{1}{\GW} \) (éllectrique) consommerait en un jour.

Le deutérium est une ressource abondante mais le tritium n'apparait pas naturellement.Il devrait, néanmoins, être possoble d'utiliser les neutrons relachés par la réaction de fusion pour produire du tritum à partir de lithium, dont de larges réserves existent.

Pour déclencher la réaction de fusion entre des noyaux de deutérium et de tritium, il est nécessaire de surmonter la répulsion mutuelle due à leurs charges positives. En conséquence, la section efficace de fusion est faible aux basses énergies.
Néanmoins, la section efficace augmente avec l'énergie, et atteint un maximumà \( \SI{100}{\kilo\electronvolt} \), et une balance énergétique positive est possible s'il est possible de faire réagir les particules du carburant avant qu'elles ne perdent leur énergie.
Pour obtenir ce résultat, les particules doivent garder ler énergie et rester dans la région de réaction pour un temps suffisament long.
Plus exactement, le produit de ce temps et la densité de particules qui réagissent doit être suffisamment grand.

La approches simples consistant à diriger un faisceau de particules vers une cible solide ou à travers un autre faisceau ne permettent pas de satisfaire ce critère.
Dans le premier cas, les particules perdent leur énergie trop rapidement et dans le second la densité est trop faible.

La méthode la plus prometteuse pour fournir l'énergie est de chauffer le carburant de deutérium-tritium à une température suffisante pour que la vélocité thermique des noyaux soit assez haute pour produire les réactions requisent.
La fusion ammené par cette méthodfe est appelé fusion thermonucléaire.
La température optimale n'est pas aussi élevé que celle correspondant à l'énergie de la section efficace maximale parce que les réactions requisent  se produisent dabs ka queue de haute énergie de la distribution Maxwellienne des particules chauffées.
La température nécessaire est autour de \( \SI{10}{\kilo\electronvolt} \), ou environ \( 100 \) million de degrés celcius.
À de telles températures, le carburant est entiérement ionisé.
La charge électrostatique des ions nucléaires est neutralisé par la présence d'un nombre égal d'électrons et le gaz neutre résultant de cela est appelé un plasma.

Puisque de telles températures empechent un confinement par des murs, une autre méthode est nécessaire.
Le tokamak est une méthode possible.
Dans un tokamak, les particules du plasmas sont confinés dans une région toroïdale par un champ magnétique, elles sont tenues par le champ dans de petites orbites de giration.
Bien que les températures nécessaires, la densité et le temps de confinement ont tous été obtenus dans des tokamaks, ils n'ont pas été obtenus dans le même plasma.
Cependant, le progrès vers cet objectif a été remarquable , et une puissance thermonucléaire de plus de quatre-bingt-dix pourcent de la puissance injectée a été produite.
Une étape suivante est d'atteindre l'ignition, où, comme avec des carburants fossiles, le processus de réaction devient auto suffisant sans chauffage additionnel.
Le progrès vers l'ignition pout être mesuré avec un simple paramètre.
La forme de la dépendance de la section efficace de fusion à l'énergie permet d'écrire le pré-requis pour l''ignition approximativement par :

\[
	, \tau_E T > \SI{5e21}{\per\meter\cubed\second\kilo\electronvolt}
\]

\begin{figure}
\centering
    % Minipage pour la légende (à gauche)
    \begin{minipage}[c]{0.35\textwidth}
        \caption{Dans un réacteur le produit \( n\tau_E \) de la densité et du temps de confinement de l'énergie et de la température \( T \), doivent tous deux être dans le bon domaine. En prenant les valeurs de pics, le \( n\tau_E \) nécessaire est \( \SI{2.5e20}{\per\meter\cubed\second} \) et la température autour de \( \qtyrange[range-units=single,range-phrase=-]{10}{20}{\kilo\electronvolt} \). La valeur requise du produit \( n \tau_E T \) est approximativement \( \SI{5e21}{\per\meter\cubed\second\kilo\electronvolt} \) (Section 5.1). La figure montre le progrès dans l'amélioration de ce produit, se dirigeant vers les limites des conditions d'un réacteur.}
        \label{fig:triple_product}
    \end{minipage}\hfill
    % Minipage pour l'image (à droite)
    \begin{minipage}[c]{0.55\textwidth}
        \flushright
        \includegraphics{figures/chapter1/triple_product.pdf}
    \end{minipage}
	
\end{figure}

où \( n \) et \( T \) sont la densité et la température maximum dans le plasma et \( \tau_E \) est le temps de confinement.
L'amélioration de la valeur de ce paramèrre peut être vue dans la \autoref{fig:triple_product}.

On considère aujourd’hui qu'un tokamak qui atteindrait l'ignition peut être construit.
Cependant, le design d'un tel réacteur pose un lare panel de questions.
Un réacteur commercial encore plus.
Les recherchent actuelles sont orientés pour répondre à ces question et ce livre donne une introduction à nos connaissances de la physique sous-jacente.

\section{Réactions de fusion}
\label{sec:reaction_de_fusion}

De loin, la réaction de fusion la plus prometeuse est celle dans laquelle les noyaux de deutérium et de tritium fusionne pour produire une particule alpha avec  la libération d'un neutron, en détail

\[
\begin{array}{r r l c l c l}
_1^2 \text{D} + & _1^3 \text{T} & \rightarrow & _2^4 He & + & _0^1n &\\
  &  &          & \vert & & \vert &\\
  &  &          & \SI{3.5}{\mega\electronvolt} & + & \SI{14.1}{\mega\electronvolt} & =  \SI{17.6}{\mega\electronvolt}\\
\end{array}
\]

Où les énergies données sont les énergies cinétiques des produits de la réaction.
Le bilan de masse et d’énergie vient du déficit de mass \( \delta m \) de la réaction

\[
\begin{array}{c c c c c}
_1^2 \text{D} & + & _1^3 \text{T} & \\
(2 - 0.000994)m_p & & (3-0.006284)m_p & \\
 & \rightarrow & _2^4 He & + & _0^1n\\
 & & (4- 0.027404)m_p & & (1 + 0.001378)m_p
\end{array}
\]

où \( m_p \) est la masse d'un proton ( \( \SI{1.6726e-27}{\kilogram} \)).
Le déficit de masse est de \( 0.01875 m_p \), et l'énergie libérée est de 

\[
	E = \delta m c^2 = 0.01875 m_p c^2 = \SI{2.818e-12}{joules} = \SI{17.59}{\mega\electronvolt}.
\]

\begin{figure}
    \centering
    % Minipage pour la légende (à gauche)
    \begin{minipage}[c]{0.35\textwidth}
        \caption{L'énergie potentiel en fonction de la séparation nucléaire}
        \label{fig:potential_energy}
    \end{minipage}\hfill
    % Minipage pour l'image (à droite)
    \begin{minipage}[c]{0.55\textwidth}
        \centering
        \includegraphics{figures/chapter1/potential_energy.pdf}
    \end{minipage}
\end{figure}

La réaction est induite par des collisions entre les particules, la section efficace de la réaction est donc d'une importance fondamentale.
La section efficace à basse énergie d'impact et faible car la barrière de Coulomb empêche les noyaux de s'approcher à des dimensions nucléaires requisent pour que la fusion ait lieu.
Le potentiel est illustré dans la \autoref{fig:potential_energy}.

Grâce à 'effet tunnel de la mécanique quantique, la fusion D-T se produit à des énergies inférieures que celles requse'nt pour outrepasser la barrière de Coulomb.
La section efficace pour les réactions est données dans ma \autoref{fig:cross_section}, et on voit que la section efficace maximale a lieu juste au dessus de \( \SI{100}{\kilo\electronvolt} \).

\begin{figure}
    \centering
    % Minipage pour la légende (à gauche)
    \begin{minipage}[l]{0.35\textwidth}
        \caption{Section efficace pour les réactions D-T, D-D et D-\( ^3 \)He. Les deux réactions D-D ont des section efficaces similaires, le graph montre leur somme.}
        \label{fig:cross_section}
    \end{minipage}\hfill
    % Minipage pour l'image (à droite)
    \begin{minipage}[c]{0.55\textwidth}
        \centering
        \includegraphics[width=\textwidth]{figures/chapter1/cross_section.pdf}
    \end{minipage}
\end{figure}

La raison pour laquelle la réaction D-T est priorisé par rapport aux autres réactions est clairs au vu de la \autoref{fig:cross_section}, où les sections efficaces pour

\begin{align*}
	^2 \text{D} + ^2 \text{D} & \rightarrow ^3 \text{He} + ^1 \text{n} + \SI{3.27}{\mega\electronvolt}\\
	^2 \text{D} + ^2 \text{D} & \rightarrow ^3 \text{T} + ^1 \text{H} + \SI{4.03}{\mega\electronvolt}\\
	^2 \text{D} + ^3 \text{Je} & \rightarrow ^4 \text{jE} + ^1 \text{h} + \SI{18.3}{\mega\electronvolt}
\end{align*}

sont aussi montrées.
On voit que les sections efficaces sont considérablement plus basse que pour D-T sauf à des énergies excessivement élevées.

\section{Fusion thermonucléaire}
\label{sec:fusion_thermonucleaire}

Le calcul du taux de réaction dans un plasma de D-T chaud demande une intégration sur la fonction de distribution des deux espèces.
Le taux de réaction par unité de volume entre les particules d'une espèces avec une vitesse \( v_1 \) et les particules d'une autre espèce avec la vitesse \( v_2 \) est 
\[
	\sigma\left( v'\right) f_1\left( v_1\right) f_2\left( v_2\right)
\]

où

\[
	v' = v_1 - v_2
\]

et \( f_1 \) et \( f_2 \) sont les fonctions de distribution.
Si les distributions sont Maxwelliennes

\[
	f_j\left( v_j \right) = nèj \left( \frac{m_j}{2 \pi T} \right)^{3/2} \exp - \frac{m_jv_j^2}{2T},
\]

Le taux de réaction total par unité de volume 

\[
	\mathscr{R} = \int \int  \sigma\left( v'\right) f_1\left( v_1\right) f_2\left( v_2\right) \text{d}^3 v_1 \text{d}^3 v_2 
\]

peut être écrit

\begin{align*}
	\mathscr{R} = n_1 n_2 \frac{\left( m_1 m_2 \right)^{3/2}}{\left( 2 \pi T \right)^3} \int \int &\exp \left( - \frac{m_1 + m_2}{2 T} \left( V + \frac{1}{2} \frac{m_1 - m_2}{m_1 + m_2} v' \right)^2 \right)\\ \times & \sigma \left( v' \right) v' \exp \left(- \frac{\mu v'}{2 T} \right)^2 \text{d}^3 v' \text{d}^3 V 
\end{align*}

où

\[
	V = \frac{v_1 + v_2}{2} \quad \text{et} \mu ) \frac{m_1 m_2}{m_1 + m_2},
\]

\( \mu \) étant la masse réduite.

L'intégrale sur \( V \) est \( \left( 2 \pi T / \left( m_1 + m_2 \right) \right)^{3/2} \) donc 

\begin{equation}
	\mathscr{R} = 4 \pi n_1 n_2 \left(\frac{\mu}{2 \pi T} \right)^{3/2} \int \sigma \left(v' \right) v'^3 \exp \left(- \frac{\mu v'^2}{2 T} \right) \text{d} v'.
	\label{eq:taux_reaction_1}
\end{equation}

Les sections efficaces mesurées en laboratoire sont générallement données en terme d'énergie de la particule incidente, par exemple du type 1,  qui est 
\[
	\varepsilon = \frac{1}{2} m_1 v'^2
\]

L'\autoref{eq:taux_reaction_1} peut être plus conventionellement écrite
\begin{equation}
	\mathscr{R} = \left( \frac{8}{\pi} \right)^{1/2} n_1 n_2 \left(\frac{\mu}{T} \right)^{3/2} \int \sigma \left( \varepsilon \right) \varepsilon \exp \left(- \frac{\mu \varepsilon}{m_1 T} \right) \text{d} \varepsilon.
	\label{eq:taux_reaction_2}
\end{equation}

\begin{figure}
	\centering
    % Minipage pour la légende (à gauche)
    \begin{minipage}[l]{0.35\textwidth}
        \caption{\( \left\langle \sigma v \right\rangle \) pour les réactions D-T en fonction de la température du plasma.}
        \label{fig:sigma_v}
    \end{minipage}\hfill
    % Minipage pour l'image (à droite)
    \begin{minipage}[c]{0.55\textwidth}
        \centering
        \includegraphics[width=\textwidth]{figures/chapter1/sigma_v.pdf}
    \end{minipage}
\end{figure}

Si la section efficace \( \sigma\left(\varepsilon \right) \) pour les réactions D-T donnée en \autoref{sec:reaction_de_fusion} est placée dans l'intégrale de l'\autoref{eq:taux_reaction_2}, on obtient le taux de réaction \( \mathscr{R} = n_1 n_2 \left\langle \sigma v \right\rangle  \)  où \( \left\langle \sigma v \right\rangle \)  est donné dans la \autoref{fig:sigma_v}.
Pour une densité d'ion donnée le taux maximum est obtenu pour \( n_D = n_T \).

\begin{figure}
	\centering
    % Minipage pour la légende (à gauche)
    \begin{minipage}[l]{0.35\textwidth}
        \caption{Tracé de l'\autoref{eq:taux_reaction_2} et de ses deux facteurs \( \sigma \left( \varepsilon \right) \) et \( \varepsilon \exp \left( - \mu \varepsilon / m_D T \right) \) en fonction de l'énergie normalisé \( \varepsilon/T \) pour un plasma de D-T ) une température de \( T = \SI{10}{\kilo\electronvolt} \).}
        \label{fig:integrand}
    \end{minipage}\hfill
    % Minipage pour l'image (à droite)
    \begin{minipage}[c]{0.55\textwidth}
        \centering
        \includegraphics[width=\textwidth]{figures/chapter1/integrand.pdf}
    \end{minipage}
\end{figure}

Aux températures d'intérêts les réactions nucléaires viennent principalement de la queue de la distribution.
Cela est visible dans la \autoref{fig:integrand} où l'intégrande de l'\autoref{eq:taux_reaction_2} est tracé en fonction de \( \varepsilon/T \) ainsi que les  deux facteurs \( \sigma \left( \varepsilon \right) \) et \( \varepsilon \exp \left( - \mu \varepsilon / m_D T \right) \) pour un plasma de D-T ) une température de \( T = \SI{10}{\kilo\electronvolt} \).

\begin{figure}
	\centering
    % Minipage pour la légende (à gauche)
    \begin{minipage}[l]{0.35\textwidth}
        \caption{\( \left\langle \sigma v \right\rangle \) pour les réactions D-D (total) et D-\( ^3 \)He en fonction de la température du plasma. Les valeurs sont nettement plus faibles que pour les réactions D-T, qui sont incluses pour comparaison.}
        \label{fig:sigma_v_3_reaction}
    \end{minipage}\hfill
    % Minipage pour l'image (à droite)
    \begin{minipage}[c]{0.55\textwidth}
        \centering
        \includegraphics[width=\textwidth]{figures/chapter1/sigma_v_3_rections.pdf}
    \end{minipage}
\end{figure}

Les expériences sont générallement réalisées avec di deitéroiù plutôt qu'un mélange de deutérium-tritium.
Une  représentation de \( \left\langle \sigma v \right\rangle \) pour le deutérium est donnée dans la \autoref{fig:sigma_v_3_reaction} avec aussi celui pour D-\( ^3 \)He.
Dans le domaine de température \( \qtyrange[range-units=single,range-phrase=-]{5}{20}{\kilo\electronvolt} \) le rapport de \( \left\langle \sigma v \right\rangle \) pour le D-T et celui du deutérium est d'environ \( 80 \).

\section{Bilan de puissance}
\label{sec:bilan_de_puissance}

\subsubsection*{Puissance thermonucléaire}

La puissance thermonucléaire par unité de volume dans un plasma D-T est

\begin{equation}
	p_{T_n} = n_D n_T  \left\langle \sigma v \right\rangle \mathscr{E},
	\label{eq:puissance_thermonucléaire}
\end{equation}

où \( n_D \) et \( n_T \) sont les densité de deutérium et de tritium, \( \left\langle \sigma v \right\rangle \) est le taux donnée dans la \autoref{fig:sigma_v} et 
\( \mathscr{E} \) est l'énergie relachée par réaction.
La densité totale est
\[
	n ) n_D + n_T,
\]

l'\autoref{eq:puissance_thermonucléaire} peut être écrite

\[
	p_{T_n} = n_D \left(n - n_D\right)  \left\langle \sigma v \right\rangle \mathscr{E}.
\]

Pour une densité donnée \( n \) la puissance est maximisée par \( n_D = \frac{1}{2} n \), les densités de deutérium et de tritium sont égales.
Pour ce mélange optimal, la densité de puissance thermonucléaire est

\begin{equation}
	p_{T_n} = \frac{1}{4} n^2  \left\langle \sigma v \right\rangle \mathscr{E}
	\label{eq:puisance_thermonucleaire_2}.
\end{equation}

\subsubsection*{Perte d'énergie}

Dans un tokamak, le plasma subit une perte d’énergie continue qui doit être compensée par un chauffage du plasma.
L'énergie moyenne du plasma à une température \( T \) est de \( \frac{3}{2} T \), composé de \( \frac{1}{2} T \) par defré de liberté.
Puisqu'il y a un nombre égal d'ions et d'électrons, l'énergie du plasma par unité de volume est de \( 3 n T \).
L'énergie total dans le plasma est donc

\begin{align}
	W & = \int 3 n T \text{d} x \nonumber \\
	 & = 3 \overline{nT} V,
	 \label{eq:puissance_totale}
\end{align}

où la barre représente la valeur moyenne, et \( V \) le volume du plasma.
Le taux de perte d'énergie \( P_L \), est caractérisé par un temps de confinement de l'énergie définie par la relation

\begin{equation}
	P_L = \frac{W}{\tau_E}
	\label{eq:perte_energie}
\end{equation}

Dans les tokamaks actuels, la puissance thermonucléaire est généralement basse et en régime stationnaire la perte d'énergie est compasé par un chauffage externe. Donc, si la puissance externe est \( P_H \),

\begin{equation}
	P_H = P_L
	\label{eq:puissance_chauffage}
\end{equation}

et les équations \ref{eq:perte_energie} et \ref{eq:puissance_chauffage} donnent
\[
	\tau_E = \frac{W}{P_H}
\]

Cette expression permet d'avoir un moyen de déterminer \( \tau_E \) depuis des quantités connues expérimentallement.

\subsubsection*{Chauffage par particule \( \alpha \)}

La puissance thermonucléaire donnée par l'\autoref{eq:puisance_thermonucleaire_2} est composé de deux parties.
Quatre cinquièmes de l'énergie de la réaction est porté par les neutrons et le reste, \( \mathscr{E}_\alpha \)n est porté les particules \( \alpha \).
Les neutrons quittent le plasma sans intéragir, mais les particules \( \alpha \), étant chargées, sont confinées par le champ magnétique.
Les particules \( \alpha \) transfèrent leur énergie de \( \SI{3.5}{\mega\electronvolt} \) au plasma par collisions

Donc le chauffage par particules \( \alpha \) par unité de volume est de
 
\begin{equation}
	p_\alpha = \frac{1}{4} n^2 \left\langle \sigma v \right\rangle \mathscr{E}
	\label{eq:puissance_alpha}
\end{equation}

Et le chauffage par particules \( \alpha \) total est

\begin{align}
	P_\alpha &= \int p_\alpha \text{d}^3 x\nonumber \\
	&= \frac{1}{4} \overline{n  \left\langle \sigma v \right\rangle }\mathscr{E} V.
	\label{eq:chauffage_total}
\end{align}

\subsubsection*{Bilan de puissance}

Dans le bilan de puissance global la perte de puissance est compebsé par un chauffage externe et la puissance des particules \( \alpha \). Autrement dit

\[
	P_H + Pè\alpha = P_L
\]

et en utilisant les équations \ref{eq:puissance_totale}, \ref{eq:perte_energie} et \ref{eq:chauffage_total}, ce bilan est donné par 
\begin{equation}
	P_H + \frac{1}{4} \overline{n  \left\langle \sigma v \right\rangle }\mathscr{E} V = \frac{3 \overline{n T}}{\tau_E} V 
	\label{eq:puissance_chauffage_moyenne}
\end{equation}

Les implications de cette équations sont décritents dans la section suivante.

\section{Ignition}

\subsubsection*{Conditions de l'ignition}

Lorsqu'un plasma de D-T est chauffé à des conditions thermonucléaire, le chauffage par particules \( \alpha \) fourni une proportion de plus en plus importante du chauffage total.
Lorsque les bonnes conditions de confinement sont obtenues, il arrive un point où la température du plasma peut être maintenue malgré les pertes d'énergie uniquement  grâce au chauffage par particules \( \alpha \).
Le chauffage additionnel peut alors être retiré et la température du plasma est entretenue par le chauffage interne.
Par analogie avec les combustibles fossiles, on parle alors d’ignition.

Le bilan de puissance est décrit dans l'\autoref{eq:puissance_chauffage_moyenne} et en prenant une densité et une température constante pour simplifié, il peut être noté

\begin{equation}
	P_H = \left( \frac{3nT}{\tau_E} - \frac{1}{4}n^2 \left\langle \sigma v \right\rangle \mathscr{E}_\alpha \right) V.
	\label{eq:puissance_chauffage_simplifiee}
\end{equation}

L'\autoref{eq:puissance_chauffage_simplifiee} donne la condition pour l'ignition, le pré-requis pour que le plasma brule de façon autonome étant

\begin{equation}
	n \tau_E > \frac{12}{\left\langle \sigma v \right\rangle} \frac{T}{\mathscr{E}_\alpha}
	\label{eq:ignition_inegalite}
\end{equation}

\begin{figure}
	\centering
    % Minipage pour la légende (à gauche)
    \begin{minipage}[l]{0.35\textwidth}
        \caption{La valeur de \( n \tau_E \) nécessaire pour obtenir l'ignition en fonction de la température.}
        \label{fig:n_tau_E_ignition}
    \end{minipage}\hfill
    % Minipage pour l'image (à droite)
    \begin{minipage}[c]{0.55\textwidth}
        \centering
        \includegraphics[width=\textwidth]{figures/chapter1/n_tau_E_ignition.pdf}
    \end{minipage}
\end{figure}

Le coté droit de l'inégalité \ref{eq:ignition_inegalite} est en fonction de la température uniquement et un tracé de la manière dont la valeur de \( n \tau_E \) requise dépend de la température est donné dans la \autoref{fig:n_tau_E_ignition}.
Le minimum est à \( T = \SI{30}{\kilo\electronvolt} \) et le pré-requis pour l'ignition à cette température est

\begin{equation}
	n \tau_E > \SI{1.5e20}{\per\meter\cubed\second}
\end{equation}

Cependant, puisque \( \tau_E \) est lui-mme une fonction de la temperature, la temperature du minimum ne doit pas être prise comme une condition optimale.
Il se trouve que la temperature d'ignition est probablement quelque peu en dessous.
Par une heureuse coïncidence pour les calculs, dans le domaine de \( \qtyrange[range-units=single,range-phrase=-]{10}{20}{\kilo\electronvolt} \), le taux de réaction peut être pris, avec moins de \( \SI{10}{\percent} \) d'erreur, commeune valeur constante

\begin{equation}
	\left\langle \sigma v \right\rangle = \SI{1.1e-24}{\text{T}\squared \meter\cubed\per\second}, \quad \text{T en} \SI{}{\kilo\electronvolt}
	\label{eq:sigma_v_constant} 
\end{equation}

en utilisant \( \mathscr{E}ç\alpha = \SI{3.5}{\mega\electronvolt} \), la condition d'ignition devient

\begin{equation}
	n T \tau_E > \SI{3e21}{\per\meter\cubed\kilo\electronvolt\second}.
	\label{eq:critere_lawson}
\end{equation}

C'est une forme très pratique pour la condition d'ignition puisqu'elle donne clairement les pré-requis sur la densité, la température et le temps de confinement.
Cette condtion peut être atteinte, par exemple, par \( n = \SI{e20}{\per\meter\cubed} \), \( T = \SI{10}{\kilo\electronvolt} \) et \( \tau_E = \SI{3}{\second} \).

La valeur précise de la constante dans la condition \ref{eq:critere_lawson} dépend du profil de \( n \) et de \( T \) et si on prend la moyenne ou les valeurs extremes.
La condition \ref{eq:critere_lawson} est pour un profil constant.
Pour des profils de densité et température paraboliques, le pré-requis de l'ignition sur les valeurs extremes est

\begin{equation}
	\hat{n}\hat{T} \tau_E > \SI{5e21}{\per\meter\cubed\kilo\electronvolt\second}
	\label{eq:critere_lawson_peak}
\end{equation}

La relation \ref{eq:ignition_inegalite} fait pensr au critère de Lawson.
Aux débuts de la recherche sur la fusion, Lawson a identifié le produit de la densité et du temps de confinement, \( n \tau \) comme un paramètre criitique pour un réacteur thermonucléaire.
Cependant, dans ses calculs il a négligé le chauffage par les particules-\( \alpha \) et a supposé que le plasma serait chauffé par une source externe.
C'est donc clairement une condition nécessaire, mais non suffisante, que la puissance produite par le réacteur, après les pertes de la conversion en électricité, doit être capale de fournir le chauffage appliqué

Dans les calculs sur l'ignition ci-dessus, seule la fraction des particles-\( \alpha \), \( \SI{20}{\percent} \), de l'énergie totale est utilisée pour chauffé le plasma.
Dans les calculs de Lawson, le facteur correspondant es lié à l'efficacité de la centrale électrique, \( \eta \), avec \( \eta  \simeq \SI{30}{\percent} \).
Donc le \( n \tau \) de Lawson se trouve être moins contraignant que le critère d'ignition \ref{eq:ignition_inegalite}, nécéssitant \( n \tau > \SI{0.6e20}{\per\meter\cubed\second} \).
Lawson a aussi pris en compte le rayonnement Bremsstrahlung de l'hydrogène mais, comme il sera montré dans la section \color{red}{Section 4.24 (lien à rajouter quand elle sert faite)}\color{black}, cette perte est faible pour un plasma de tokamak.

Une façon de mesuré le succès à s'approcher des conditions d'un réacteurs est donné par le rapport, \( Q \), entre la puissance thermonucléaire et la uissance du chauffage appliqué, donné par

\[
	Q = \frac{\frac{1}{4} n^2 \left\langle \sigma v \right\rangle \mathscr{E} V}{P_H}
\]

Puisque l'énergie \( \mathscr{E} \) relaché par chaque réaction est cinq fois l'énergie de la particule-\( \alpha \), \( \mathscr{E} \), \( Q \) peut aussi être écrit

\[
	Q = \frac{5 P_\alpha}{P_H}
\]

Donc, \( Q= 1 \) correspond à une puissance des particules-\( \alpha \) de \( \SI{20}{\percent} \) la puissance de chauffage appliquée.
À l'ignition, où \( P_H \) peut être mise à \( 0 \), \( Q \rightarrow \infty \).
On peut voir que même si un plasma en ignition a la propriété désirable qu'aucun chauffage n'est nécessaire, il est possible d'obtenir un \( Q \) très large sans ignition.
Cependant, dans ce cas, la puissance fournie, \( P_H \) représente un coût pour le système, puisqu’elle implique de recycler une partie de la puissance du réacteur, avec la perte d’efficacité correspondante.

\subsubsection*{Approche de l'ignition}

L'approche de l'ignition peut être décrite en ajoutant la dépendance en t'emps de l'\autoref{eq:puissance_chauffage_simplifiee}n cela donne

\begin{equation}
	\frac{\dd}{\dd t}3nT = \frac{P_H}{V} + \frac{1}{4} n^2 \sigmav \mathscr{E}_\alpha - \frac{3nT}{\tau_E(n, T)}
	\label{eq:power_balance_time}
\end{equation}

Si la puissance de chauffage est augmentée lentement, la solution de l’\autoref{eq:power_balance_time} correspond à une succession d’états quasi stables, pour lesquels les termes du côté droit de l’équation s’annulent.
Le résultat dépend alors de la dépendance en température et en densité du temps de confinement, qui est incertaine dans le régime considéré.

\begin{figure}
	\centering
    % Minipage pour la légende (à gauche)
    \begin{minipage}[l]{0.35\textwidth}
        \caption{Le chauffage par particule-\( \alpha \) et les pertes pour un temps de confinement constant et une ignition à \( \SI{10}{\kilo\electronvolt} \). Le tracé en dessous montre le chauffage externe nécessaire pour compenser les pertes et maintenir une température stable.}
        \label{fig:power_comp}
    \end{minipage}\hfill
    % Minipage pour l'image (à droite)
    \begin{minipage}[c]{0.55\textwidth}
        \centering
        \includegraphics[width=\textwidth]{figures/chapter1/power_comp.pdf}
    \end{minipage}
\end{figure}

Lz type de comportement attendu est montré dans la \autoref{fig:power_comp}. 
On a la dépendance en température du chauffage par particules-\( \alpha \) et les pertes de puissance ainsi que la puissance de chauffage nécessaire, pour un temps de confinement constant et une ignition à \( \SI{10}{\kilo\electronvolt} \).
On voit donc que dans ce cas, la puissance de chauffage externe maximale est pour environ \( \SI{5}{\kilo\electronvolt} \) et cette puissance est alors moins de \( \SI{40}{\percent} \) de la puissance du chauffage par particules-\( \alpha \) ) la température d'ignition.

\vspace{2\baselineskip}
\begin{wrapfigure}{c}{0.4\textwidth}
    \centering
    % Réduction du retrait interne du wrap
    \vspace{-10pt}

    \begin{subfigure}[b]{0.48\textwidth}
        \centering
        \includegraphics[width=\textwidth]{figures/chapter1/contour.pdf}
        \caption{}
        \label{subfig:puissance_contour}
    \end{subfigure}
    
    \begin{subfigure}[b]{0.48\textwidth}
        \centering
        \includegraphics[width=\textwidth]{{figures/chapter1/wireframe.pdf}}
        \caption{}
        \label{subfig:puissance_3D}
    \end{subfigure}

    \caption{(a) Countours de la puissance nécessaire pour maintenir un étqt stationnaire dans le plan \( 'n, T) \). (b) Un traacé en deux dimension donnant la forme de \( P(n, T)  \) cirrespondant aux contours de la figure (a)}
    \vspace{-10pt}
    \label{fig:contour_3D}
\end{wrapfigure}

Une vue plus générale de l'approche de l'ignition est obtenue en considérant le bilan de puissance dans le plan \( 'n, T) \).
En utilisant l'\autoref{eq:puissance_chauffage_simplifiee} il est possible de dessiner les contours d'égales valeurs de \( P \) nécessaires pour maintenir une température donnée à une densité donnée.
Les incertitudes sur \( \tau_E (n, T) \) empêche le tracé d'un diagram précis, mais la \autoref{subfig:puissance_contour} montre la forme générale attendue \autoref{subfig:puissance_3D} donne la même information dans la forme d'un tracé deux dimentionnel de \( P(n, T) \).
On peut voir que la trajectoire vers l'ignition nécéssitant d'une puissance minimale est celle passant par dela le point de selle.
Ce point de selle est souvent appellé le passafe de Cordey.

À l'ignition, la puissance appliquée peut être éteinte et il y a un équilibre entre le chauffage par particule-\( \alpha \) et les pertes de puissance du plasma.
Cependant pour le cas montré dans la \autoref{fig:power_comp} cet équilibre est instable.
Une légère augmentation de la température induit un déséquilibre positif entre le chauffage et les pertes, ce qui amplifie l’élévation de température.

Cette instabilité peut être analysée pour une situation plus générale dans laquelle \( \tau_E \) est pris comme étant en fonction de la température et la densité est considérée comme constante.
Donc, en prenant un chauffage dépendant du temps comme dans l'\autoref{eq:power_balance_time} avec \( P_H = 0 \),

\begin{equation}
	3n \frac{\dd T}{\dd t} = -3n \frac{T}{\tau_E (T)} + \frac{1}{4} n^2 \sigmav \mathscr{E}_\alpha .
	\label{eq:power_balance_time_density_constant}
\end{equation}

L'équilibre est donné par

\begin{equation}
	3 \frac{T}{\tau_E} = \frac{1}{4} n \sigmav \mathscr{E}_\alpha .
	\label{eq:power_balance_time_equilibrium}
\end{equation}

En considérant un petit changement, \( \Delta T \), de la température d'équilibre et en dévelopant \( \tau_E \) et \( \sigmav \) autour de leur valeur d'équilibre, l'\autoref{eq:power_balance_time_density_constant} donne l'équation gouvernant la stabilité.

\begin{equation}
	3, \frac{\dd \Delta T}{\dd t} = \left[ -3, \left( \frac{1}{\tau_E} - \frac{T}{\tau_E^2} \frac{\dd \tau_E}{\dd T} \right) +  \frac{1}{4} n^2 \frac{\dd \sigmav}{\dd T} \mathscr{E}_\alpha \right] \Delta T
	\label{eq:stability_equation}
\end{equation}

En utilisant l'équilibre de l'\autoref{eq:power_balance_time_equilibrium} dans l'\autoref{eq:stability_equation} on obtient

\[
	3, \frac{\dd \Delta T}{\dd t} =  \frac{1}{4} n^2 \sigmav \frac{\mathscr{E}_\alpha}{T} \left( -1 + \frac{T}{\tau_E} \frac{\dd \tau_E}{\dd T} + \frac{T}{\sigmav} \frac{\dd \sigmav}{\dd T} \right) \Delta T.
\]

Si la partie de droite de l'équation est positive, alors \( \Delta T \) croît exponentiellement.
La condition de la stabilité est donc

\[
	\frac{T}{\tau_E} \frac{\dd \tau_E}{\dd T} < 1 - \frac{T}{\sigmav} \frac{\dd \sigmav}{\dd T}.
\]

\begin{figure}
	\centering
    % Minipage pour la légende (à gauche)
    \begin{minipage}[l]{0.35\textwidth}
        \caption{La stabilité pour le chauffage par particules-\( \alpha \) deamnde que \( (T/\tau_E) \dd \tau_E / \dd T \) soit inférieur à une valeur critique. Le tracé conne la dépendance en température de cette valeur critique.}
        \label{fig:stability_condition}
    \end{minipage}\hfill
    % Minipage pour l'image (à droite)
    \begin{minipage}[c]{0.55\textwidth}
        \centering
        \includegraphics[width=\textwidth]{figures/chapter1/stability_condition.pdf}
    \end{minipage}
\end{figure}

\begin{figure}
	\centering
    % Minipage pour la légende (à gauche)
    \begin{minipage}[r]{0.35\textwidth}
        \caption{Une deterioration du confinement avec une augmentation de la température est un effet stabilisant pour le chauffage par particule-\( \alpha \). Le tracé montre la perte de puissance en fonction de la température pour \( \tau_E \propto 1/T \) avec la puissance de chauffage par particule-\( \alpha \) correspondant. Pour le cas montré, les conditions ont été choisies pour que l'ignition instable arrive à \( \SI{13}{\kilo\electronvolt} \). L'instabilité conduit alors la température vers un équilibre stable à \( \SI{15}{\kilo\electronvolt} \). Le graph rtonqué montre le chauffage additionnel nécessaire pour maintenir une température donnée.}
        \label{fig:power_comp_tau_E}
    \end{minipage}\hfill
    % Minipage pour l'image (à droite)
    \begin{minipage}[c]{0.55\textwidth}
        \centering
        \includegraphics[width=0.8\textwidth]{figures/chapter1/power_comp_tau_E.pdf}
    \end{minipage}
\end{figure}

\clearpage
\section{Tokamaks}

\vspace{2\baselineskip}
\begin{wrapfigure}{r}{0.2\textwidth}
    % Réduction du retrait interne du wrap
    \vspace{-10pt}

    \begin{subfigure}[b]{0.48\textwidth}
        \includegraphics[width=0.63\linewidth]{figures/chapter1/schematic_tokamak_1.pdf}
        \caption{}
        \label{subfig:schematic_tokamak_1}
    \end{subfigure}
    
    \begin{subfigure}[b]{0.48\textwidth}
        \includegraphics[width=0.3\linewidth]{{figures/chapter1/schematic_tokamak_2.pdf}}
        \caption{}
        \label{subfig:schematic_tokamak_2}
    \end{subfigure}

    \caption{(a) Le champ magnetique toroïdal \( B_\phi \) et le champ magnetique poloïdal \( B_p \) induit par le courant toroïdal \( I_\phi \). (b) La combinaison de \( B_\phi \) et \( B_p \) fait que les lignes de champ s’enroulent autour du plasma. }
    \vspace{-10pt}
    \label{fig:schematic_tokamak_1_2}
\end{wrapfigure}

Le tokamak est un système de confinement toroïdal du plasma, le plasma étant confiné par un champ magnétique.
Le champ magnétique principal étant le champ toroïdal.
Cependant, ce champ uniquement ne permet pas le confinement du plasma.
Pour avoir un équilibre dans lequel la pression du plasma est équilibré par les forces magnétiques, il est nécessaire d'avoir aussi un champ magnétique poloïdal.
Dans un tokamak, ce champ est principalement produit par le courant dans le plasma lui-même, ce courant circulant dans la direction toroïdale.
Ces courants et champs sont illustré dans la \autoref{subfig:schematic_tokamak_1}.
La combinaison du champ toroïdal \( B_\phi \) et du champ poloïdal \( B_p \) donne des lignes de champ magnétique qui ont une trajectoire hélicoïdale autour du tore, comme montré dans la \autoref{subfig:schematic_tokamak_2}.
Le champ magnétique toroïdal est produit par les courants dans les bobines liant le plasma, comme montré dans la \autoref{fig:schematic_tokamak_3}.

\begin{wrapfigure}{r}{0.2\textwidth}
    \centering
    % Réduction du retrait interne du wrap
    \vspace{-10pt}
    
    \includegraphics[width=\linewidth]{{figures/chapter1/schematic_tokamak_3.pdf}}

    \caption{Le chamm magnétique toroïdal est produit par le courant dans des bobines externes. }
    \vspace{-10pt}
    \label{fig:schematic_tokamak_3}
\end{wrapfigure}

La pression du plasma est le produit de la densité de particules et de la température.
Comme la réactivité du plasma augmente avec ces deux quantités, dans un réacteur la pression doit être suffisamment haute.
La pression qui peut être confinée est déterminée par des considérations de stabilité et augmente avec la force du champ magnétique.
Cependant, l'intensité du champ magnétique toroïdal est limitée par des facteurs technologiques.
Dans des expériences en laboratoire avec des bobines en cuivre, les besoins de refroidissement et les forces magnétiques mettent une limite sur le champ magnétique qu'elles peuvent atteindre.
De plus, dans un réacteur, les pertes par effet Joule dans des bobines normales sont inacceptables et des bobines supraconductrices sont donc nécessaires.
Il y a une perte de supraconductivité au-delà d'un champ magnétique critique, et cela est une autre limitation.
Avec les technologies actuelles, il semble probable que le champ magnétique maximal au niveau des bobines soit limité autour de \( \SI{12}{\tesla} \), mais des conducteurs améliorés avec des champs allant jusqu'à \( \SI{16}{\tesla} \) sont également envisagés\footnote{En 2021, des prototypes de bobines avec un champ de \( \SI{20}{\tesla} \) ont été testés avec succès en laboratoire.}.
Le champ toroïdal maximal se trouve sur le côté intérieur de la bobine de champ toroïdal.
Puisque le champ magnétique toroïdal est inversement proportionnel au grand rayon, le champ au centre du tokamak serait autour de \( \qtyrange[range-units=single,range-phrase=-]{6}{8}{\tesla} \)\footnote{Avec des bobines de \( \SI{20}{\tesla} \), le champ au centre du tokamak SPARC est envisagé d'être de \( \SI{12}{\tesla} \).}.
Le champ toroïdal dans les grands tokamaks actuels est quelque peu en dessous de cette valeur.

Pour un champ magnétique toroïdal donné, la pression du plasma qui peut être stablement confinée augmente avec le courant du plasma jusqu'à une valeur limite.
Les champs magnétiques poloïdaux résultants sont typiquement un ordre de magnitude plus faible que le champ toroïdal.
Dans les grands tokamaks actuels, des courants de plusieurs mégaampères sont utilisés, un courant de \( \SI{7}{\mega\ampere} \) ayant été produit dans le tokamak JET.
Avec des hypothèses conservatrices, un réacteur devrait nécessiter un courant de \( \qtyrange[range-units=single,range-phrase=-]{20}{30}{\mega\ampere} \).
Des avancées technologiques et des connaissances pourraient mener à des valeurs plus basses.

\begin{figure}

\centering
    % Réduction du retrait interne du wrap
    \vspace{-10pt}

    \begin{subfigure}[b]{0.4\textwidth}
        \centering
        \includegraphics[width=\linewidth]{figures/chapter1/schematic_tokamak_4.pdf}
        \caption{}
        \label{subfig:schematic_tokamak_4}
    \end{subfigure}
    \hspace{20pt}
    \begin{subfigure}[b]{0.53\textwidth}
        \centering
        \includegraphics[width=\linewidth]{{figures/chapter1/schematic_tokamak_5.pdf}}
        \caption{}
        \label{subfig:schematic_tokamak_5}
    \end{subfigure}

    \caption{(a) Un changement de flux à travers le tore induit un champ électrique toroïdal qui entraine le courant toroïdal. (b) La variation de flux est produite par l’enroulement primaire, souvent à l’aide d’un noyau de transformateur.}
    \vspace{-10pt}
    \label{fig:schematic_tokamak_4_5}

\end{figure}

Dans les expériences actuelles, le courant du plasma est entraîné par un champ électrique toroïdal induit par l'action du transformateur, au cours de laquelle un changement de flux à travers le tore est généré, comme illustré par la \autoref{subfig:schematic_tokamak_4}.
La variation de flux est provoquée par un courant circulant dans une bobine primaire enroulée autour du tore, comme montré dans la \autoref{subfig:schematic_tokamak_5}.
Bien que non essentiel, un noyau en fer de transformateur est souvent utilisé, celui-ci réduit les besoins en alimentation électrique et présente l'avantage de limiter les champs magnétiques de fuite.

\vspace{2\baselineskip}
\begin{wrapfigure}{r}{0.4\textwidth}
	\centering 
    \includegraphics[width=\linewidth]{{figures/chapter1/schematic_tokamak_6.pdf}}

    \caption{L'arrangement des bobines dans un tokamak. }
    \vspace{-10pt}
    \label{fig:schematic_tokamak_6}
\end{wrapfigure}

Il y a des avantages pour le confinement et la pression atteignable avec les plasmas qui sont allongés verticalement.
Le contrôle de la forme demandes des courants toroïdaux supplémentaires.
De plus, de tels courants sont nécessaires pour contrôler la position du plasma.
Ces courants toroïdaux sont transportés par des bobines judicieusement placées.
Le système complet de bobines toroïdales et poloïdales est illustré dans la \autoref{fig:schematic_tokamak_6}.

Les mécanismes limitant le confinement du plasma dans les tolomaks ne sont pas tous compris.
Cependant l'amélioration attendue du confinement avec la taille a été observée expérimentallement.
Typiquement, les meilleurs temps de confinement de l'énergie pour les tokamaks existants sont autour de \( \frac{1}{2} r_p^2 \) où \( r_p \) est le petit rayon moyen du plasma.
Un temps de confinement de l'énergie supérieur à une seconde a été obtenu dans le tokamakJET.
Il est établi que le temps de confinement de l'énergie augmente avec le courant du plasma mais, malheureusement, décroît avec une augmentation de la pression du plasma.

Les plasmas de tokamak sont chauffés à des températures de quelques \( \SI {}{\kilo\electronvolt}\) par le chauffage ohmique du courant du plasma.
Les températures requisent de \( \gtrsim \SI{10}{\kilo\electronvolt} \) sont alors atteintes avec du chauffage additionnel par des faisceaux de particules ou par ondes électromagnétiques.

Les plasmas de tokamak actuels ont des densités de particules de l'ordre de \(  \qtyrange[range-units=single,range-phrase=-]{1e19}{1e20}{\per\meter\cubed}  \).
C'est un facteur \( 10^6 \) plus bas que la densité de l'atmosphère.
Le plasma est contenu dans une chambre à vide et pour minimiser la présence d'impuretés, des pressions résiduelles faibles doivent être maintenues.

\begin{figure}
	\centering
        \begin{subfigure}[b]{0.5\textwidth}
        \centering
        \includegraphics[width=\linewidth]{figures/chapter1/schematic_tokamak_7.pdf}
        \caption{}
        \label{subfig:schematic_tokamak_7}
    \end{subfigure}
    \hspace{2pt}
    \begin{subfigure}[b]{0.38\textwidth}
        \centering
        \includegraphics[width=\linewidth]{{figures/chapter1/schematic_tokamak_8.pdf}}
        \caption{}
        \label{subfig:schematic_tokamak_8}
    \end{subfigure}
    
    \caption{La séparation du plasma des murs de la chambre à vide par (a) un limiteur et (b) un diverteur.}
    \label{fig:schematic_tokamak_7_8}
\end{figure}

Les impuretés dans le plasma donnent lieu à des pertes par radiation et diluent le carburant.
La restriction de leur entré dans le plasma jooue donc un rôle fondamental dans la bonne opération des tokamaks.
Cela demande une séparation du plasma des murs de la chambre à vide.
Deux techniques sont utilisées actuellement.
La première est de définir une limite externe du plasma avec un limiteur matériel comme montré dans la \autoref{subfig:schematic_tokamak_7}.
La deuxième est de garder les particules loin des murs de la chambre à vide par une modification du champ magnétique pour produire un diverteur magnétique comme montré dans la \autoref{subfig:schematic_tokamak_8}.

Un réacteur tokamak demandera des éléments additionnels dans la structure du tokamak et aura aussi besoin d'un moyen de convertir la puissance de fusion en électricité.
C'es éléments sont décris dans la \autoref{sec:tokamak_reacteur}.

\section{Tokamak réacteur}
\label{sec:tokamak_reacteur}

\end{document}