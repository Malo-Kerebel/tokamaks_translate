\documentclass[main.tex]{subfiles}

\begin{document}

\setcounter{chapter} {1}
\chapter{Physique des plasmas}

\section{Plasma de tokamak}
\label{sec:plasma_de_tokamak}

Un plasma est un gaz ionisé.
Complètement ioniséc il est composé uniquement d'ions et d'électrons.
Ceux-ci ont beaucoup de propriétés d'un gaz normal.
Par exemple, ils peuvent être décrits par leur densité de particule et leur température.
Cependant, un plasma a deux propriétés caractéristiques.
Premièrement, la densité de charges électriques des deux espèces est si grande qu'une séparation notable mène à des forces de restauration très grandes, et par conséquent, les densités d'ions et d'électrons sont pratiquement égales.
Deuxièmement, un plasma peut conduire un courant électrique du fait d’un mouvement relatif entre les ions et les électrons.
Dans un tokamak, le courant du plasma produit une part importante du champ magnétique.
Là où le courant du plasma traverse le champ magnétique, il engendre une force magnétique qui peut équilibrer le gradient de pression du plasma.

Lorsqu'un plasma est dans un champ magnétique, les particules individuelles sont con\-traintes dans leur déplacement.
Elles sont libres dans la direction parallèle au champ magnétique mais perpendiculairement elles suivent des orbites de Larmor.
Dans un tokamak, les orbites des ions ont un rayon de quelques millimètres et les orbites des électrons sont plus petites d'un facteur égal à la racine carré du ratio de masse électron-ion.
Bien que le comportement précis du plasma soit déterminé par le déplacement des particules individuelles dans le champ électromagnétique local, les contraintes sur le déplacement des particules décrites ci-dessus donnent au plasma des propriétés similaires à un fluide sur des échelles supérieures aux rayons de Larmor.
La majorité de notre compréhension des tokamaks provient de modèles où le plasma est considéré comme un fluide.

La densité de particules dans un tokamak est de l'ordre de \( \sim \SI{e20}{\per\meter\cubed} \) soit environ \( 10^{-5} \) fois la densité de l'atmosphère.
Les plasmas de tokamaks atteignent généralement des températures de quelques \( \SI{}{\kilo\electronvolt} \), ce qui correspond à plusieurs dizaines de millions de Kelvin.
C'est de l'ordre de \( 10^5 \) fois la température de l'atmosphère et par conséquent la pression dans un tokamak est comparable à celle de l'atmosphère.

La force de la pression du plasma allant vers l'extérieur est contrebalancée par le champ magnétique.
Cependant, la densité d'énergie du plasma dans un tokamak est faible comparée à celle du champ magnétique, de l'ordre du pour cent.
Le champ magnétique principal est le champ toroïdal produit par des bobines au dehors du plasma.
Le champ poloïdal produit par le courant toroïdal du plasma  est typiquement dix fois inférieur.

De nombreux prpcessus dans les plasmas sont déterminés par les collisions entre les particules.
Les collisions entre les ions et les électrons donnent lieu à la résistance électrique, qui mène au chauffage ohmique du plasma.
Les collisions causent aussi le transport des particules et de l'énergie, menant à la perte des deux.
Typiquement, le temps de collision des ions et dans l'intervalle de \( \qtyrange[range-units=single,range-phrase=-]{1}{100}{\milli\second} \).
Le temps de collision des électrons est plus petit d'un facteur égal à la racine carrée du ratio de masse entre les ions et les électrons.
Les temps de collision augmentent avec la température, évoluant comme \( T^{3/2} \).
Par conséquent, le chauffage ohmique devient moins efficace à haute température.
D'un autre côté, les pertes collisionnelles du plasma sont réduites.

\begin{table}
	\centering
	\caption{Paramètres typiques des tokamaks}
	\begin{tabular}{l l}
		\hline
		Volume du plasma & \( \qtyrange[range-units=single,range-phrase=-]{1}{100}{\meter\cubed} \)\\
		Masse du plasma totale & \( \qtyrange[range-units=single,range-phrase=-]{e-7}{e-5}{\kilogram} \)\\
		Densité ionique & \( \qtyrange[range-units=single,range-phrase=-]{e19}{e20}{\per\meter\cubed} \)\\
		Température & \( \qtyrange[range-units=single,range-phrase=-]{1}{40}{\kilo\electronvolt} \)\\
		Pression & \( \qtyrange[range-units=single,range-phrase=-]{0.1}{5}{} \) atmosphères\\
		Vitesse thermique des ions & \( \qtyrange[range-units=single,range-phrase=-]{100}{1000}{\kilo\meter\per\second} \)\\
		Vitesse thermique des électrons & \( \qtyrange[range-units=single,range-phrase=-]{0.01}{0.1}{c} \)\\
		Champ magnétique & \( \qtyrange[range-units=single,range-phrase=-]{1}{10}{\tesla} \)\\
		Courant du plasma total & \( \qtyrange[range-units=single,range-phrase=-]{0.1}{7}{\mega\ampere} \)\\
		\hline 
	\end{tabular}
	\label{table:plasma_parametre}
\end{table}

Le comportement basique d'un tokamak échappe majoritairement à notre connaissance.
La perte d'énergie dépasse significativement celle prédite par de simple collisions et cela ne s'explique pas.
Cette anomalie pourrait venir d'instabilités du plasma à petite échelle.

Les plasmas de tokamaks typiques (\autoref{table:plasma_parametre}) sont loin d'être calmes et de nombreuses instabilités macroscopiques sont observées régulièrement.
Dans certains cas le plasma s'adapte à l'instabilité et il n'y a pas de détérioration des performances observée.
Cependant, dans le cas des bien-nommées disruptions de tokamak, les dégâts de l'instabilités sont irréparables.

Ce chapitre donne une introduction à la physique utilisée pour l'analyse et la compréhension des plasmas de tokamaks.

\section{Écrantage de Debye}
\label{sec:ecrantage_debye}

\begin{wrapfigure}{c}{0.2\textwidth}
    \centering
    
    \includegraphics{figures/chapter2/charge_sheet}

    \caption{Couches de charges ioniques et électroniques séparées par une distance \( d \)}
    \label{fig:charge_sheet}
\end{wrapfigure}

La densité de charge électrique des ions et des électrons séparés composant un plasma est assez grande pour que seulement une faible séparation des charges ne soit possible.
Cet effet peut être interprété en imaginant la séparations des ions et des électrons en couches d'une épaisseur \( d \) comme montré dans la \autoref{fig:charge_sheet}.

Si les ions sont simplement ionisés et que la densité des ions et des électrons est \( n \), la charge par unité de surface des couches est \( dne \).En ignorant les facteurs numériques le champ électrique entre les couches est \( \sim dne/\varepsilon_0 \) et la force par unité de surface est 

\begin{equation}
	F \sim \frac{\left( dne \right)^2}{\varepsilon_0}.
	\label{eq:force_unite_surface}
\end{equation}

Dans un tokamak, la densité \( n \) est autour de \( \SI{e20}{\per\meter\cubed} \) et pour cette densité on a 

\[
	F \sim 10^{13} d^2 \SI{}{\newton\per\meter\squared}.
\]

Par exemple, pour une épaisseur de \( \SI{1}{\centi\meter} \), cela donne une force par unité de surface de \( \SI{e9}{\newton\per\meter\squared} \).

Puisque cette force est si grande par la séparation des charges, les densités d'ions et d'électrons sont tenus quasiment égales au travers du plasma. Ce que l'on appelle la quasi-neutralité et, pour le cas général avec des ions de charge \( Z \), la contrainte s'exprime

\begin{equation}
	n_e = \sum_i n_i Z_i
	\label{eq:quasi_neutral}
\end{equation}

où la somme se fait sur les différentes espèces d'ions.

Il est à noter que l'\autoref{eq:quasi_neutral} ne veut pas dire que \( \nabla \cdot \bm{E} = 0 \), l'\autoref{eq:quasi_neutral} est presque, mais pas totalement, exacte et les petite différences de charges donnent lieu à des champs électriques significatifs.
Donc le champ électrique ne peut pas être déterminé par l'\autoref{eq:quasi_neutral}.
D'un autre côté, étant donné le champ électrique \( \bm{E} \), la densité de charges \( \rho_c \) est déterminé par \( \nabla \cdot \bm{E} = \rho_c/\varepsilon_0 \).

On voit dans l'\autoref{eq:force_unite_surface} que la force décroit avec la baisse de la distance de séparation.
L'argument pour justifier la quasi-neutralité est donc invalide à des échelles suffisamment petite.
Il est alors possible d'obtenir une longueur fondamentale caractérisant un plasma en calculant l'épaisseur \( d \) pour laquelle l'énergie interne du plasma pourrait fournir l'énergie pour une séparation complète des ions et des électrons comme montré dans la \autoref{fig:charge_sheet}.
L'énergie requise est \( F d \), la force \( F \) étant donnée par la relation \ref{eq:force_unite_surface} et l'énergie interne est \( \sim dn T \) . En mettant ces deux énergies égales, on obtient la longueur caractéristique \( \lambda_D \) où

\begin{align}
	\lambda_D &= \left( \frac{\varepsilon_0 T}{n e^2} \right)^{1/2} \nonumber \\
	&= 2.35 \times 10^5 \left( \frac{T}{n} \right)^{1/2} \SI{}{\meter}, \quad T \text{en} \SI{}{\kilo\electronvolt}.
	\label{eq:longueur_debye}
\end{align}

Cette longueur est appelée longueur de Debye.
Dans un plasma de tokamak typique \( \lambda_D \) est de l'ordre de \( \qtyrange[range-units=single,range-phrase=-]{e-2}{e-1}{\milli\meter}  \).

Même si la séparation de charge décrite au dessus est énergétiquement possible au-delà de la longueur de Debye, cela ne se produit évidemment pas de façon spontanée dans le corps du plasma parce que la vitesse des particules est aléatoire et la coherence du mouvement des particules requise pour ce déplacement imaginé ne se produit pas.
Une situation où une séparation de charge notable se produit est lorsque le plasma est en contact avec une surface solide.
La séparation de charge apparait alors dans une gaine proche de la surface, et cette gaine a une épaisseur \( \sim \lambda_D \).
La longueur de Debye apparait aussi de façon plus subtile dans le plasma, caractérisant un phénomène que l'on appelle écrantage de Debye.

Si on considère un ion simplement ionisé dans un plasma, le champ électrique de l'ion est alors

\[
	\bm{E} = \frac{e}{4\pi \varepsilon_0 r^2d}.
\]

Bien que ce champ électrique soit directement associé à cet ion, les autres particules dans le plasma s'ajustent à ce champ, et la distribution résultante de leurs charges écrante la charge de l’ion et modifie le champ électrique effectif.
Les trajectoires des électrons proches de l'ion sont légèrement déplacées vers l'ion et les trajectoires des autres ions sont légèrement déplacées à l'opposé de celui-ci.
L'écrantage apparait pour chaque ion et l'effet inverse aussi pour chaque électron.
Ce comportement est similaire à celui observable dans les électrolytes.
Ce cas a été analysé par Debye et cet écrantage porte donc son nom.

La forme de l'écrantage peut être calculé pour un ion stationnaire en résolvant l'équation de Poisson pour le potentiel \( \phi \).
En géométrie sphérique l'équation est

\begin{equation}
	\frac{1}{r^2} \frac{\dd}{\dd r}r^2 \frac{\dd \phi}{\dd r} = - \frac{\rho_c}{\varepsilon_0},
	\label{eq:poisson}
\end{equation}

où la densité de charge \( \rho_c = \sum_j n_j e_j \). Les ions et les électrons vont chacun avoir une distribution de Boltzmann dans le potentiel, et donc leur densité \( n_j \) sera donnée par

\begin{equation}
	n_j = n_0 \exp - \frac{e_j \phi}{T},
	\label{eq:Boltzmann_densite}
\end{equation}

\( n_0 \) étant la densité d'électrons, et d'ions, à une grande distance de l'ion choisi, où le potentiel est pris égal à \( 0 \).
Donc en prenant l'\autoref{eq:Boltzmann_densite} dans l'\autoref{eq:poisson} et en prenant des ions simplement ionisés

\begin{equation}
	\frac{1}{r^2} \frac{\dd}{\dd r}r^2 \frac{\dd \phi}{\dd r} = - \frac{n_0 e}{\varepsilon_0} \left( e^{-e\phi / T} - e^{e\phi/T} \right).
	\label{eq:poisson_boltzmann}
\end{equation}

Il sera trouvé que \( e\phi/T \ll 1\) et en anticipant ce résultat, l'\autoref{eq:poisson_boltzmann} devient

\begin{equation}
	\frac{1}{r^2} \frac{\dd}{\dd r}r^2 \frac{\dd \phi}{\dd r} = \frac{2 n_0 e^2}{\varepsilon_0 T} \phi.
	\label{eq:poisson_ephi_T_1}
\end{equation}

En prenant le produit \( \left( r \phi \right) \) en variable, l'\autoref{eq:poisson_ephi_T_1} peut être écrite

\[
	\frac{\dd^2}{\dd r^2} \left( r \phi \right) = \frac{2}{\lambda_D^2} \left( r \phi \right)  
\]

et la solution qui décroit pour un grand \( r \) est

\begin{equation}
	\phi = \alpha \frac{1}{r} e^{-\sqrt{2} r / \lambda_D}
	\label{eq:solution_phi_alpha}
\end{equation}

où \( \alpha \) est une constante. Puisque \( \phi = e/4\pi \varepsilon_0 r \) lorsque \( r \rightarrow 0 \), l'\autoref{eq:solution_phi_alpha} devient

\begin{equation}
	\phi = \frac{e}{4\pi\varepsilon_0 r} e^{-\sqrt{2} r / \lambda_D}
	\label{eq:solution_phi}
\end{equation}

\vspace{2\baselineskip}
\begin{wrapfigure}{c}{0.4\textwidth}
    \centering
    
    \includegraphics{figures/chapter2/potential}

    \caption{Tracés du potentiel écranté et non-écranté autour d'un ion.}
    \label{fig:potential}
\end{wrapfigure}

Cette équation décrit l'écrantage de Debye.
Il est visible que le potentiel \( e/4\pi \varepsilon_0 r \) de l'ion est écranté selon un facteur exponentiel, la longueur caractéristique de l'écrantage étant la longueur de Debye
Les tracés des potentiels écrantés et non-écrantés sont donnés dans la \autoref{fig:potential}.

Avec l'\autoref{eq:solution_phi}, l'approximation \( e\phi/T \ll 1 \) utilisée pour le calcul implique

\begin{equation}
	r \gg \frac{e^2}{4\pi \varepsilon_0 T}.
	\label{eq:inegalite_r}
\end{equation}

L'aspect physique de ce pré-requis peut être visualisé en observant que la plus petite valeur pertinente de \( r \) est de l'ordre de la séparation entre les particules, qui est de \( 1/n^{1/3} \).
En utilisant cette valeur pour \( r \), l'inégalité \ref{eq:inegalite_r} devient

\[
	n \lambda_D^3 \gg \left(\frac{1}{4\pi} \right)^{3/2}
\]

et cette approximation est valide s'il y a beaucoup de particules dans un cube de côté \( \lambda_D \), autrement dit

\begin{equation}
	n \lambda_D^3 \gg 1.
	\label{eq:inegalite_nlambda}
\end{equation}

Dans un plasma de tokamak les valeurs typiques sont de \( n \simeq \SI{e20}{\per\meter\cubed} \) et \( \lambda_D \simeq \SI{e-4}{\meter} \).
Donc \( n \lambda_D^3 \simeq 10^8\) et l'inégalité \ref{eq:inegalite_nlambda} est satisfaite.
La valeur \( n \lambda_D^3 \) est appelé le paramètre du plasma.

L'hypothèse que la particule de test soit stationnaire a donné un compte rendu simplifié de l'écrantage.
En réalité, chaque particule a une vitesse relativement à la vitesse moyenne des particules de fond.
Quand la particule de test a une vitesse comparable à la vitesse thermique des particules d'écrantage, il y a un délai dans la formation de la charge d'écrantage et seul un écrantage partiel se produit.
Dans la limite de particules de test très rapides, la réponse de l'écrantage est négligeable et le potentiel associé à la charge de test approche le potentiel non écranté de la forme \( e/4\pi \varepsilon_0 r \).

Le comportement global est tel que chaque particule génère sa propre charge d'écrantage lorsqu'elle se déplace à travers le plasma et cela donne un champ électrique fluctuant.
Les particules reçoivent aussi de l'énergie du champ électrique et l'équilibre entre ces processus de forçage et d'amortissement donne le spectre thermique des fluctuations électrostatiques

\section{Fréquence plasma}
\label{sec:frequence_plasma}

\vspace{2\baselineskip}
\begin{wrapfigure}{c}{0.4\textwidth}
    \centering
    % Réduction du retrait interne du wrap
    \vspace{-10pt}

    \begin{subfigure}[b]{0.48\textwidth}
        \centering
        \includegraphics[width=\textwidth]{figures/chapter2/displaced_electrons.pdf}
        \caption{}
        \label{subfig:displaced_electrons}
    \end{subfigure}
    
    \begin{subfigure}[b]{0.48\textwidth}
        \centering
        \includegraphics[width=\textwidth]{{figures/chapter2/displaced_electrons_2.pdf}}
        \caption{}
        \label{subfig:displaced_electrons_2}
    \end{subfigure}

    \caption{(a) Les électrons déplacés créent une force de restauration et une accélération, (b) après un demi cycle, les charges sont interchangées}
    \vspace{-10pt}
    \label{fig:displaced_electrons}
\end{wrapfigure}

Dans la \autoref{sec:ecrantage_debye}, il a été montré qu'un plasma a une longueur caractéristique, la longueur de Debye.
Un plasma a aussi une fréquence caractéristique.
Cela peut être compris en considérant une couche déplacée d'électrons comme montré dans la \autoref{subfig:displaced_electrons}.
Il est admis qu'à cause de la plus grande masse des ions, ils sont stationnaires et ont une densité uniforme.
Les charges positives dans la figure représentent les défauts d'électrons dans les couches desquelles les électrons ont été déplacés.
Le champ électrique résultant est montré dans la figure.
Les électrons sont accélérés par le champ électrique et se déplacent pour annuler la charge positive.
Au moment de l'annulation, la quantité de mouvement des électrons est maximum et cela permet aux électrons de récréer la séparation des charges, maintenant dans la phase opposée.
Ce processus est répété et le déplacement des électrons en résultants constitue les oscillations du plasma avec la fréquence plasma caractéristique.

La fréquence plasma peut être calculée en utilisant les équations fluides.
Pour un plasma froid, l'équation du mouvement pour le fluide électron est

\begin{equation}
	m_e \frac{\partial v}{\partial t} = - e \bm{E}
	\label{eq:motion_electron}
\end{equation}

et en assumant une oerturbation de la densité faible \( \tilde{n} \), le champ électrique est donné par

\begin{equation}
	\nabla \cdot \bm{E} = - \frac{e \tilde{n}}{\varepsilon_0}
	\label{eq:E_perturbation_faible}
\end{equation}

Donc en prenant la divergeance de l'\autoref{eq:motion_electron} et en éliminant \( \nabla \cdot \bm{E} \) en utilisant l'\autoref{eq:E_perturbation_faible}

\begin{equation}
	m_e \frac{\partial }{\partial t} \nabla \cdot v = \frac{e^2}{\varepsilon_0} \tilde{n}.
	\label{eq:motion_sans_E}
\end{equation}

L'équation de continuité est

\[
	\frac{\partial n}{\partial t} = - \nabla \cdot (nv)
\]

et la forme linéarisée est

\begin{equation}
	\frac{\partial \tilde{n}}{\partial t} = -n \nabla \cdot v.
	\label{eq:forme_linearise}
\end{equation}

En substituant \( \nabla \cdot v \) de l'\autoref{eq:forme_linearise} dans l'\autoref{eq:motion_sans_E} donne l'équation nécessaire pour les oscillations du plasma

\[
	\frac{\partial^2 \tilde{n}}{\partial t^2} = - \omega_p^2 \tilde{n}
\]

où \( \omega_p \) est la fréquence plasma d'un électron

\begin{equation}
	\omega_p = \left( \frac{n e^2}{\varepsilon_0 m_e} \right)^{1/2}.
	\label{eq:frequence_plasma}
\end{equation}

Une fréquence similaire est associée à l'oscillation des ions, la fréquence plasma d'un ion étant

\begin{equation}
	\omega_{pi} = \left( \frac{n e_i^2}{\varepsilon_0 m_i} \right)^{1/2}
	\label{eq:frequence_plasma_ion}
\end{equation}

où \( e_i \) est la charge d'un ion.

La \color{red}{Section 2.25 (lien à rajouter quand elle sera faite)}\color{black} donne un compte rendu plus général des oscillations du plasma et il y est montré que l'\autoref{eq:frequence_plasma} donne la fréquence des ondes du plasma d'électron dans la limite où la longueur d'onde est beaucoup plus grande que la longueur de Debye.

En utilisant des valeurs numériques dans l'\autoref{eq:frequence_plasma}

\[
	\omega_p = 56.4 n^{1/2} \SI{}{\per\second}.
\]

Pour un tokamak cette fréquence est très élevée. Par exemple, une densité \( n = \SI{e20}{\per\meter\cubed} \) donne \( \omega_p = \SI{5.6e11}{\per\second} \) avec un temps caractéristique \( \omega_p^{-1} \) inférieur à un centième de nanoseconde.

\section{Orbites de Larmor}
\label{sec:orbites_larmor}

L'équation du mouvement d'une particule de masse \( m_j \) et de charge \( e_j \) dans un champ magnétique est

\[
	m_j \frac{\dd v}{\dd t} = e_j v \times \bm{B}
\]

Si le champ magnétique est uniforme et dans la direction \( z \), les composantes de l'équation sont

\begin{align}
	\frac{\dd v_x}{\dd t} = \omega_{cj} v_y, \quad \frac{\dd v_y}{\dd t} = -\omega_{cj} v_x, \label{eq:composantes_x_y}\\
	\frac{\dd v_z}{\dd t} = 0 \label{eq:composante_z}
\end{align}

où

\[
	\omega_{cj} = \frac{e_j \bm{B}}{m_j}
\]

est la fréquence cyclotron.
Par l'\autoref{eq:composante_z}, on a que la vitesse dans la direction du champ magnétique est constante.
La séparation des variables des équations \ref{eq:composantes_x_y} mène à

\[
	\frac{\dd^2 v_x}{\dd t^2} = -\omega_{cj}^2 v_x, \quad \frac{\dd^2 v_y}{\dd t^2} = -\omega_{cj}^2 v_y,
\]

et les solutions à ces équations peuvent être écrites

\begin{equation}
	v_x = v_\perp \sin \omega_{cj} t, \quad v_y = v_\perp \cos \omega_{cj} t.
	\label{eq:solutions_vx_vy}
\end{equation}

En utilisant \( v_x = \dd x / \dd t \) et \( v_y = \dd y / \dd t \), les équations \ref{eq:solutions_vx_vy} peuvent être intégrées pour avoir

\begin{equation}
	x = - \rho_j \cos \omega_{cj} t, \quad y = \rho_j \sin \omega_{cj} t,
	\label{eq:solution_x_y}
\end{equation}

où

\[
	\rho_j = \frac{v_\perp}{\omega_{cj}} = \frac{m_j v_\perp}{e_j B}
\]

est le rayon de Larmor.
Donc la particule a une orbite hélicoïdale composée des mouvements circulaires des équations \ref{eq:solution_x_y} et une vitesse constante dans la direction du champ magnétique.

Une particule ayant l'énergie thermique moyenne dans le plan perpendiculaire au champ magnétique a \( v_\perp^2 = 2 v_{Tj}^2 \) où \( \frac{1}{2} m_j v_{Tj}^2 = \frac{1}{2} T_j \).
Le facteur \( 2 \) apparait à cause des deux degrés de liberté en jeu.
Donc pour une particule thermique

\begin{equation}
	\rho_j = \sqrt{2} \frac{m_j v_{Tj}}{\vert e_j \vert \bm{B}}
	\label{eq:rayon_larmor}
\end{equation}

En remplaçant par les valeurs numériques de la charge de l'électron \( e = \SI{1.605e-19}{\coulomb} \), de la masse de l'électron \( m_e = \SI{9.11e-31}{\kilogram} \) et de la masse du proton \( m_p = \SI{1.673e-27}{\kilogram} \),ainsi que par \( v_{Tj} = 1.27 \times 10^{-8} \left( T_j / m_j \right)^{1/2} \SI{}{\meter\per\second} \) avec la température exprimée en \( \SI {}{\kilo\electronvolt}\), donne

\[
	\left. \begin{array}{l l}
		\text{Électron} & \left\vert \omega_{ce} \right\vert = 1.76 \times 10^{11} B \SI{}{\per\second},\\
		 & \rho_e = 1.07 \times 10^{-4} T_e^{1/2}/ B \SI{}{\meter},\\
		 \text{Proton} & \omega_{cp} = 9.58 \times 10^{7} B \SI{}{\per\second},\\
		  & \rho_p = 4.57\times 10^{-3} T_p^{1/2}/B \SI{}{\meter},\\
		  \text{Particule } j & \omega_{cj} = 9.58 \times 10^{7} \left( Z/A\right) B \SI{}{\per\second},\\
		  & \rho_j = 4.57\times 10^{-3} \left(A^{1/2}/Z\right) T_j^{1/2}/B \SI{}{\meter},
	\end{array} \right\rbrace T \text{in} \SI{}{\kilo\electronvolt},
\]

où \( Z \) et \( A \) sont la charge et le nombre de masse de la particule \( j \).

\begin{table}
	\centering
	\begin{tabular}{llll}
\hline 
 & \multicolumn{3}{l}{Champ magnétique}\\ 
\hline 
Fréquence & \( 1 \) Tesla & \( 3 \) Teslas & \( 5 \) Teslas \\ 
\hline 
\(  \left\vert \omega_{ce} \right\vert (\SI{}{\per\second}) \) & \( \SI{1.76e11}{} \) & \( \SI{5.28e11}{} \) & \( \SI{8.79e11}{} \) \\ 
\( \omega_{cp} (\SI{}{\per\second}) \) & \( \SI{9.58e7}{} \) & \( \SI{2.87e8}{} \) & \( \SI{4.79e8}{} \) \\ 
\( f_{ce} ( \SI{}{\giga\hertz}) \) & \( 28 \) & \( 84 \) & \( 140 \) \\ 
\( f_{cp} ( \SI{}{\giga\hertz}) \) & \( 15 \) & \( 46 \) & \( 76 \) \\ 
\hline 
\end{tabular} 
\caption{Valeur de la fréquence cyclotron \( \omega_c \) et \( f_c( =\vert \omega_c\vert/2\pi) \) pour les électrons et les protons.}
\label{tab:frequence_larmor}
\end{table}

\begin{table}
	\centering
	\begin{tabular}{l l l l l l}
	\hline
	 &Rayon de & Température & & & \\
	 \hline
	 \( B \) & Larmor & \( \SI{10}{\electronvolt} \) & \( \SI{100}{\electronvolt} \) & \( \SI{1}{\kilo\electronvolt} \) & \( \SI{10}{\kilo\electronvolt} \)\\
	 \hline
	 \( 3 \) & \( \rho_e \) & \( \SI{0.003}{\milli\meter} \) & \( \SI{0.011}{\milli\meter} \) & \( \SI{0.035}{\milli\meter} \) & \( \SI{0.11}{\milli\meter} \)\\
	 Teslas  & \( \rho_p \) & \( \SI{0.15}{\milli\meter} \) & \( \SI{0.48}{\milli\meter} \) & \( \SI{1.5}{\milli\meter} \) & \( \SI{4.8}{\milli\meter} \)\\
	 \( 5 \) & \( \rho_e \) & \( \SI{0.002}{\milli\meter} \) & \( \SI{0.007}{\milli\meter} \) & \( \SI{0.021}{\milli\meter} \) & \( \SI{0.67}{\milli\meter} \)\\
	 Teslas  & \( \rho_p \) & \( \SI{0.09}{\milli\meter} \) & \( \SI{0.29}{\milli\meter} \) & \( \SI{0.91}{\milli\meter} \) & \( \SI{2.9}{\milli\meter} \)\\
	\end{tabular}
	\caption{Valeurs du rayon de Larmor, \( \rho \), pour les électrons et les protons ayant une vitesse thermique.}
	\label{tab:rayon_larmor}
\end{table}

Le \autoref{tab:frequence_larmor} et le \autoref{tab:rayon_larmor} donnent quelques valeurs de \( \omega_c \) et de \( \rho \).
Il faut noter que le rayon de Larmor d'une particule thermique est parfois défini sans le facteur \( \sqrt{2} \) de l'\autoref{eq:rayon_larmor}. 

\section{Déplacement d'une particule selon \( B \)}
\label{sec:deplacement_B}

Comme décrit dans la \autoref{sec:orbites_larmor}, le déplacement d'une particule chargée dans un champ magnétique uniforme est composé de deux parties; une orbite circulaire perpendiculaire au champ magnétique et une vitesse uniforme selon le champ.
Une accélération de la particule selon le champ magnétique est introduite s'il y a un champ électrique parallèle à \( \bm{B} \) ou un gradient de \( \bm{B} \) pzrallèle à \( \bm{B} \).

\subsubsection*{Accélérayion par \( \bm{E}_\parallel \)}

Un champ électrique parallèle donne simplement une accélération donnée par

\[
	\frac{\dd}{\dd t} \left( m_j v_\parallel \right) = e_j \bm{E_\parallel} 
\]

si \( \bm{E_\parallel} \) est une fonction de \( t \), \( v_\parallel \) est donné par

\[
	m_j v_\parallel = e_j \int \mathbf{E_\parallel} \dd t.
\]

Dans certains cas la vitesse résultantes de ces accelérations est relativistique et la masse de la particule est alors liée à la masse au repos \( m_{j0} \) par la relation

\[
	m_j = \frac{m_{j0}}{\sqrt{1-v^2/c^2}}.
\]

Si \( \bm{E_\parallel} \) est une fonction de la distance \( x_\parallel \) selon la magnitude du champ et du temps, il faut alors résoudre l'équation

\[
	\frac{\dd}{\dd t} \left( m_j \frac{\dd x_\parallel}{\dd t} \right) = e_j \bm{E_\parallel} \left( x_\parallel, t \right).
\]

\subsubsection*{Accélérayion par \( \nabla_\parallel \bm{B} \)}

Une particule chargée se déplaçant parallèlement à un champ magnétique n'est soumise à aucune force magnétique.
Cependant, si la particule a aussi une vitesse perpendiculaire au champ magnétique et lz champ magnétique a un gradient parallèle à \( \bm{B} \), il y a une force parallèle au champ magnétique sur la particule au centre de la gyro-orbite.

\begin{figure}
\centering
    % Minipage pour la légende (à gauche)
    \begin{minipage}[c]{0.35\textwidth}
        \caption{Le gradient de \( \bm{B} \) parallèle ) \( \bm{B} \) fait apparaitre une composante de la force \( e_j \left( \bm{v} \times \bm{B} \right) \) selon la direction du déplacement du centre guide.}
        \label{fig:gradB}
    \end{minipage}\hfill
    % Minipage pour l'image (à droite)
    \begin{minipage}[c]{0.55\textwidth}
        \flushright
        \includegraphics{figures/chapter2/grad_B.pdf}
    \end{minipage}
\end{figure}

La géométrie de la situation est montrée dans la \autoref{fig:gradB}.
La force sur la particule est \( e_j \left( \bm{v} \times \bm{B} \right) \).
Si le champ magnéiique change lentement, la composante de cette force parallèle à la ligne de champ au centre guide de l'orbite est

\begin{equation}
	F = \alpha\left\vert e_j \left( \bm{v} \times \bm{B} \right) \right\vert
	\label{eq:expression_force}
\end{equation}

où \( \alpha \) est le petit angle entre le champ magnétique à la position de la particule et celui au centre guide et cette force est clairement dans la direction du champ magnétique le plus faible.

En prenant des coordonnées cylindriques avec l'axe \( z \) le long de la ligne du centre guide et une coordonée radiale \( r \), l'angle \( \alpha \) est donné par

\begin{equation}
	\alpha = \frac{B_r}{B_z}
	\label{eq:def_alpha}
\end{equation}

où

\begin{equation}
	B_r = \frac{\partial B_r}{\partial r} \rho,
	\label{eq:def_Br}
\end{equation}

\( \rho \) étant le rayon de Larmor. Puisque \( \nabla \cdot \bm{B} \) et \( B_r = r \partial B_r / \partial r \),

\begin{equation}
	\frac{1}{r} \frac{\partial}{\partial r} \left( r B_r \right) = 2 \frac{\partial B_r}{\partial r} = - \frac{\partial B_z}{\partial z}.
	\label{eq:relation_Br_Bz}
\end{equation}

En prenant \( \partial B_z / \partial z \) au centre guide,

\begin{equation}
	\frac{\partial B_z}{\partial z} = \left\vert \nabla_\parallel B \right\vert
	\label{eq:rel_Bz_gradB}
\end{equation}

En combinant les équations de \ref{eq:def_alpha} à \ref{eq:rel_Bz_gradB} et en mettant \( B_z = B \), l'angle \( \alpha \) est donné par

\begin{equation}
	\alpha = \frac{1}{2} \rho \frac{\left\vert \nabla_\parallel B \right\vert}{B}.
	\label{eq:expr_alpha}
\end{equation}

En équilibrant les forces pour l'orbite de Larmor,

\begin{equation}
	\left\vert e_j \left( \bm{v} \times \bm{B} \right) \right\vert ) \frac{m v_\perp^2}{\rho}
	\label{eq:larmor_balance}
\end{equation}

et, donc, en utilisant les équations \ref{eq:expression_force}, \ref{eq:expr_alpha} et \ref{eq:larmor_balance} la force issue de \( \nabla\parallel \bm{B} \) est

\begin{equation}
	F = - \frac{\frac{1}{2} m v_\perp^2}{B} \nabla_parallel B.
	\label{eq:force_nabla_B}
\end{equation}

Une particule se déplaçant dans un champ magnétique d'intensité croissante peut être réfléchie par cette force, cet effet étant appelé l'effet de mirroir.
Si un champ magnétique a un minimum sur une ligne de champ, les particules dans cette région de champ plus faible peuvent être piégées entre les deux mirroirs qui en résultent.

La quantité \( \mu = \frac{1}{2} m v_\perp^2 / B \) qui apparait dans l'\autoref{eq:force_nabla_B} est un invariant adiabatique étant quasiment constant dans un champ magnétique qui varie faiblement.
Cette invariance est décrite dans la \color{red}{Section 2.7 (lien à rajouter quand elle sera faite)}\color{black}.

\section{Dérive des particules}
\label{sec:derive_particule}

Les orbites de Larmor circulaires décrites dans la \autoref{sec:orbites_larmor} venaient de l'hypothèse d'un cxhamp magnétique uniforme ayant des lignes de champ droites et aucun champ électrique.
Un quelquonque chamgement de cet état de base mène à une accélération parallèles au champ magnétique ou à une dérive des particules perpendiculaire au champ magnétique.
Les déplacements accélérés ont été expliqués dans la \autoref{sec:deplacement_B}.

À l'échelle du rayon de Larmor, les particules chargées tournoient rapidemment autour du centre guide de leur mouvement, mais  des dérives perpendiculaires à des échelles plus grande du centre guide provoquent  un des cas suivants:

\begin{itemize}
	\item[ (1) ] un champ électrique perpendiculaire au champ magnétique;
	\item[ (2) ] un gradient du champ magnétique perpendiculaire au champ magnétique;
	\item[ (3) ] une courbure du champ magnétique;
	\item[ (4) ] un champ électrique dépendant du temps;
\end{itemize}

La vitesse de la érive de chacun de ces cas est décris ci-dessous.

\subsubsection*{Dérive \( \bm{E} \times \bm{B} \)}

S'il y a un champ électrique perpendiculaire au champ magnétique l'orbite de la particule subit une dérive perpendiculaire aux deux champs.
Cela est appelé la dérive \( \bm{E} \times \bm{B} \).

L'équation du mouvement est

\begin{equation}
	m_j \frac{\dd \bm{v}}{\dd t} = e_j \left( \bm{E} + \bm{v} \times \bm{B} \right).
	\label{eq:eq_mouvement_ExB}
\end{equation}

En choisissant la coordonnée \( z \) le long du champ magnétique et la coordonnée \( y \) le long du champ électrique perpendiculaire. les composant de l'\autoref{eq:eq_mouvement_ExB} sont

\[
	m_j \frac{\dd v_x}{\dd t} = e_j v_y B, \quad m_j \\frac{\dd v_y}{\dd t} = e_j \left( E - v_x B \right).
\]

Les solutions à ces équations peuvent s'écrire

\begin{wrapfigure}{c}{0.4\textwidth}
    \centering
    
    \includegraphics{figures/chapter2/ExB}

    \caption{ La dérive \( \bm{E} \times \bm{B} \) d'un ion et d'un électron, \( v_d = E/B \)}
    \label{fig:ExB}
\end{wrapfigure}

\begin{equation}
	v_x = v_\perp \sin \omega_{cj} t + \frac{E}{B}, \quad v_y = v_\perp \cos \omega_{cj} t
	\label{eq:sol_ExB}
\end{equation}

La dérive est indépendante de la charge, de la masse et de l'énergie de la particule.
L'entiéreté du plasma est donc sujet à cette dérive.
Le trajectoires d'un ion et d'un électron sont montrées dans la \autoref{fig:ExB}.

La dérive \( \bm{E} \times \bm{B} \) peut être mieux comprise en reconnaissant que le champ électrique perpendiculaire est dépendant du référentiel.
Donc, ce champ électrique peut être transformé vers un référentiel ayant une vitesse perpendiculaire \( v_f \) de sorte que 

\begin{equation}
	\bm{E} + \bm{v_f} \times \bm{B} = 0.
	\label{eq:referentiel_0}
\end{equation}

Dans ce référentiel l'orbite de la particule sera simplement circulaire.
On voit donc que dans le référentiel de base la particule va avoir son mouvement circulaire avec une vitesse de dérive \( \bm{v_d} = \bm{v_f} \).
En croisant l'\autoref{eq:referentiel_0} avec \( \bm{B} \) donne maintenant la vitesse de dérive

\[
	\bm{v_d} = \frac{\bm{E} \times \bm{B}}{\bm{B}^2}
\]

comme précédemment.

\subsubsection*{ Dérive \( \nabla B \)}

\vspace{2\baselineskip}
\begin{wrapfigure}{c}{0.4\textwidth}
    \centering
    
    \includegraphics{figures/chapter2/drift_grad_B}

    \caption{Un gradient de \( \bm{B} \) perpendiculaire à \( \bm{B} \) donne une dérive aux ions et aux électrons dans des directions opposées.}
    \vspace{10pt}
    \label{fig:grad_B}
\end{wrapfigure}

Dans un champ magnétique avec un gradient transverse, l'orbite de la particule a un rayon de courbure plus petit sur la partie de l'orbite avec un champ magnétique plus important, comme illustré dans la \autoref{fig:grad_B}.
Cela mène à une dérive perpendiculaire à la fois du champ magnétique et de son gradient.

En prenant le champ magnétique le long de l'axe \( z \), et son gradient sur l'axe \( y \), l'intensité de cette dérive peut être calculé avec la composante en \( y \) de l'équation du mouvement, qui est

\begin{equation}
	m_j \frac{\dd v_y}{\dd t} = - e_j v_x B.
	\label{eq:composante_y}
\end{equation}

Si on prend que le gradient de \( \bm{B} \) est faible, de sorte que la variation du champ magnétique le long du rayon de Larmor est petit comparé à \( B \), le champ magnétique peut être écrit

\[
	B = B_0 + B' y
\]

où \( y= 0 \) est le plan médian de l'orbite de la particule.
En utilisant la petitesse de la perturbation de l'orbite, l'\autoref{eq:composante_y} devient

\begin{equation}
	\frac{m_j}{e_j} \frac{\dd v_y}{\dd t} = - v_{x0} \left( B_0 + B' y \right) - v_d B_0
	\label{eq:composante_y_Bprime_petit}
\end{equation}

où \( v_d \) est la vitesse de dérive nécessaire et le mouvement imperturbé d'une particule avec une vitesse perpendiculaire \( v_\perp \) est donné par

\[
	v_{x0} = v_\perp \sin \omega_{cj} t, \quad y = \rho_j \sin \omega_{cj} t
\]

où \( \rho_j = v_\perp/\omega_{cj} \). 
En substituant \( v_{x0} \) et \( y \) dans l'\autoref{eq:composante_y_Bprime_petit}

\[
	\frac{m_j}{e_j} \frac{\dd v_y}{\dd t} = - v_\perp \sin \omega_{cj} t \left( B_0 + B' \rho_j \sin \omega_{cj} t \right) - v_d B_0
\]

En prenant la moyenne temporelle de cette équation avec \( \left\langle \dd v_y / \dd t \right\rangle = 0 \) donne la vitesse de dérive nécessaire dans la direction \( x \)

\begin{equation}
	v_d = - \rho_j \frac{B'}{B} v_\perp
	\label{eq:vd_direction_x}
\end{equation}

ou dans une autre forme

\begin{equation}
	v_d = \frac{1}{2} \rho_j \frac{\bm{B} \times \nabla \bm{B}}{B^2} v_\perp .
	\label{eq:vd_direction_x_2}
\end{equation}

Les ions et les électrons ont des dérives opposées, le signe dans l'\autoref{eq:vd_direction_x_2} étant déterminé par le signe de \( e_j \) dans \( \rho_j = m_j v_\perp/ e_j B \).

\FloatBarrier
\subsubsection*{Dérive de courbure}

\begin{figure}
\centering
    % Minipage pour la légende (à gauche)
    \begin{minipage}[c]{0.35\textwidth}
        \caption{Dérive d'un ion par la courbure du champ magnétique. Les électrons dérivent dans la direction opposée.}
        \label{fig:polarisation_drift}
    \end{minipage}\hfill
    % Minipage pour l'image (à droite)
    \begin{minipage}[c]{0.55\textwidth}
        \flushright
        \includegraphics{figures/chapter2/ion_drift.pdf}
    \end{minipage}
\end{figure}

Quand le centre guide d'une particule suit une ligne de champ magnétique courbée, il subit une dérive perpendiculaire au plan dans lequel la courbure est.
Ce comportement est illustré dans la \autoref{fig:polarisation_drift}.

Pour calculer cette dérive il est utile de transposer vers un référentiel tournant avec la vélocité angulaire, \( v_\parallel/R \), de la particule, où \( v_\parallel \) est la vitesse parallèle au champ magnétique et \( R \) est le rayon de courbure de la ligne de champ.
Dans ce référentiel, la particule subit une force centrifuge \( m_j v_\parallel^2 / R \) et l'équation du mouvement est

\begin{equation}
	m_j \frac{\dd \bm{v}}{\dd t} = \frac{m_j v_\parallel^2}{R} \bm{i_c} + e_j \left( \bm{v} \times \bm{B} \right)
	\label{eq:equation_mouvement_courbure} 
\end{equation}

où \( \bm{i_c} \) est le vecteur unitaire allant vers l'extérieur du rayon de courbure.
L'\autoref{eq:equation_mouvement_courbure} est similaire à l'\autoref{eq:eq_mouvement_ExB} pour le cas de la dérive \( \bm{E} \times \bm{B} \) avec la force \( e_j \bm{E} \) remplacé par \( m_j v_\parallel^2/R \).
Donc par analogie avec la dérive \( E/B \), la dérive de courbure est donnée par

\[
	v_d = \frac{v_\parallel^2}{\omega_{cj} R}.
\]

Puisque \( \omega_{cj} \) prend le signe de la charge de la particule, les électrons et les ions ont des dérives opposées, la direction de la dérive des ions étant celle de \( \bm{i_c} \times \bm{B} \).

Si aucun courant n'est présent, la dérive \( \nabla \bm{B} \) est dans la même direction que la dérive de courbure et prend une forme similaire.
Dans ce cas \( \nabla \bm{B} = - \bm{i_c} \bm{B} / R \) et la vitesse de dérive de \( \nabla \bm{B} \) donné par l'\autoref{eq:vd_direction_x_2} est

\[
	v_d = \frac{1}{2} \frac{v_\perp^2}{\omega_{cj}R}.
\]

La dérive combinée est alors

\[
	v_d = \frac{v_\parallel^2 + \frac{1}{2} v_\perp^2}{\omega_{cj} R}
\]

ou, sous forme vectorielle

\begin{equation}
	v_d =\frac{v_\parallel^2 + \frac{1}{2} v_\perp^2}{\omega_{cj}} \frac{\bm{B} \times \nabla \bm{B}}{B^2}
	\label{eq:drift_velocity_forme_vectorielle}
\end{equation}

\FloatBarrier
\subsubsection*{Dérive de polarisation}

Quand un champ électrique perpendiculaire au champ magnétique change dans le temps cela mène à ce qu'on appelle la dérive de polarisation.
Le nom vient du fait que les dérives des ions et des électrons sont dans des directions opposées et font apparaitre un courant de polarisation proportionnel à \( \dd \bm{E} / \dd t \).

L'équation du mouvement est

\[
	m_j \frac{\dd \bm{v}}{\dd t} = e_j \left( \bm{E}(t) + \bm{v} \times \bm{B} \right).
\]

Le champ électrique peut être retiré en passant dans un référentiel accéléré ayant la vitesse

\[
	v_f = \frac{\bm{E} \times \bm{B}}{B^2}.
\]

L'équation du mouvement est donc

\[
	m_j \frac{\dd \bm{v}}{\dd t} = e_j \bm{v} \times \bm{B} - m_j \frac{\dd v_f}{\dd t}
\]

De sorte que

\begin{equation}
	m_j \frac{\dd \bm{v}}{\dd t} = e_j \bm{v} \times \bm{B} - \frac{m_j}{B^2} \frac{\dd \bm{E}}{\dd t} \times \bm{B}.
	\label{eq:eq_mouvement_polarisation}
\end{equation}

L'\autoref{eq:eq_mouvement_polarisation} est similaire à l'\autoref{eq:eq_mouvement_ExB} avec \( e_j \bm{E} \) remplacé par

\[
	-\frac{m_j}{B^2} \frac{\dd \bm{E}}{\dd t} \times \bm{B}.
\]

La dérive de polarisation correspondante à la dérive électrique \( \left( \bm{E} \times \bm{B} / B^2 \right) \) est alors

\[
	\bm{v_d} = - \frac{m_j}{e_j B^2} \left( \frac{\dd \bm{E}}{\dd t} \times \bm{B} \right) \times \bm{B},
\]

et en rappellant que \( \bm{E} \) est perpendiculaire à \( \bm{B} \),

\begin{equation}
	\bm{v_d} = - \frac{1}{\omega_{cj} B} \frac{\dd \bm{E}}{\dd t}.
	\label{eq:vd_polarisation}
\end{equation}

\begin{figure}
\centering
    % Minipage pour la légende (à gauche)
    \begin{minipage}[c]{0.35\textwidth}
        \caption{La dérive de polarisation d'un ion causée par un champ électrique perpendiculaire au champ magnétique croissant. La dérive des électrons est dans la direction opposée.}
        \label{fig:polarisation_drift}
    \end{minipage}\hfill
    % Minipage pour l'image (à droite)
    \begin{minipage}[c]{0.55\textwidth}
        \flushright
        \includegraphics{figures/chapter2/polarisation_drift.pdf}
    \end{minipage}
\end{figure}

Cette dérive est illustrée dans la \autoref{fig:polarisation_drift}.

La dérive est dans la même direction que \( \dd \bm{E} / \dd t \) pour les ions et la direction opposée pour les électrons, et elle est beaucoup plus grande pour les ions.
Si la densité d'électrons est \( n \) la densité de courant de polarisation résultante est 

\[
	\bm{j_p} = \sum_j n e_j \bm{v_{dj}}
\]

et en utilisant l'\autoref{eq:vd_polarisation} pour chaque espèce

\[
	\bm{j_p} = \frac{\rho_m}{B^2} \frac{\dd \bm{E}}{\dd t}
\]


\( \rho_m \) étant la densité de masse.

\section{Invariants adiabatiques}
\label{sec:invariant_adiabatiques}

Les invariants adiabatiques sont des quantités associés avec le mouvement des particules qui restent quasiment constantes pendant les changements du mouvement, tant que les changements sont suffisament petits.
Dans le cas des particules chargées dans un champ magnétique, les pré-requis sont que les changements temporels se font sur des échelles de temps longues comparé au temps de la gyropériode, et les changements spatiaux sont sur une échelle de distance beaucoup plus grande que le rayon de Larmor.
Il y a trois principaux invariants pour une particule chargée dans un champ magnétique, le plus important étant le moment magnétique.

\subsubsection*{Moment magnétique}

La définition basique du moment magnétique d'une boucle de courant est

\begin{equation}
	\mu = I A
	\label{eq:moment_magnetique_basique}
\end{equation}

où \( I \) est le courant dans la boucle et \( A \) est l'aire.
Le courant porté par une particule chargé en rotation sur son orbite de Larmor est

\begin{equation}
	I = \frac{\omega_{cj}}{2\pi} e_j
	\label{eq:courant_particule_rotation}
\end{equation}

et l'aire, \( \pi \rho_j^2 \) de l'orbite est

\[
	A = \pi \left( \frac{v_\perp}{\omega_{cj}} \right)^2
\]

De sorte que

\begin{equation}
	\mu = \frac{\frac{1}{2} m_j v_\perp^2}{B}
	\label{eq:def_mu_2}
\end{equation}

Pour une petite boucle de courant, la force de la boucle est

\begin{equation}
	\bm{F} = - \mu \nabla_\parallel \bm{B}
	\label{eq:force_petite_boucle}
\end{equation}

en accord avec le calcul détaillé de la \autoref{sec:deplacement_B}.

L'invariance de \( \mu \) est démontré en calculant sa dérivé temporelle, qui est

\begin{equation}
	\frac{\dd \mu}{\dd t} = \frac{1}{B} \left( \frac{\dd}{\dd t} \left( \frac{1}{2} m_j v_\perp^2 \right) - \mu \frac{\dd B}{\dd t} \right).
	\label{eq:derive_moment_magnetique}
\end{equation}

La conservation de l'énergie donne

\begin{equation}
	\frac{\dd}{\dd t} \left( \frac{1}{2} m_j v_\perp^2 \right) = e_j \bm{E} \cdot \bm{v} - m_j v_\parallel \frac{dd v_\parallel}{\dd t}
	\label{eq:derive_mu_cons_E}
\end{equation}

et le changement de \( \bm{B} \) suivant la vitesse \( \bm{v_\parallel} \) du centre guide, est donné par

\begin{equation}
	\frac{\dd \bm{B}}{\dd t} = \frac{\partial \bm{B}}{\partial t} + \bm{v_\parallel} \cdot \nabla \bm{B}.
	\label{eq:changement_temp_B}
\end{equation}

En utilisant \autoref{eq:derive_mu_cons_E} et \autoref{eq:changement_temp_B} dans l'\autoref{eq:derive_moment_magnetique}

\begin{align}
	\frac{\dd \mu }{\dd t} = & \frac{1}{B} \left\lbrace \left( - m_j \bm{v_\parallel} \frac{\bm{v_\parallel}}{\dd t} - \mu \bm{v_\parallel} \cdot \nabla \bm{B} + e_j \bm{E_\parallel} \bm{v_\parallel} \right) \right. \nonumber\\
	& + \left. \left( e_j \bm{E_\perp} \cdot \bm{v_\perp} - \mu \frac{\partial \bm{B}}{\partial t} \right) \right\rbrace
	\label{eq:derive_moment_magnetique_2}
\end{align}

L'accélération \( \dd \bm{v_\parallel} / \dd t \) est produite par la force donnée dans l'\autoref{eq:force_petite_boucle} avec celle causée par \( \bm{E_\parallel} \), et donc, en prenant le produit scalaire avec \( \bm{v_\parallel} \)

\[
	m_j \bm{v_\parallel} \frac{\dd \bm{v_\parallel}}{\dd t} - \mu \bm{v_\parallel} \cdot \nabla \bm{B} +  e_j \bm{E_\parallel} \bm{v_\parallel}.
\]

Et donc les termes dans la première parenthèse de l'\autoref{eq:derive_moment_magnetique_2} donne zéro.
Puisque \( \bm{v_\perp} = \omega_{cj} \rho_j \) la gyromoyenne de \( \bm{E_\perp} \cdot \bm{v_\perp} \) est donnée par

\begin{equation}
	\left\langle \bm{E_\perp} \cdot \bm{v_\perp} \right\rangle = \frac{\omega_{cj}}{2 \pi} \oint \bm{E} \cdot \bm{\dd s}
	\label{eq:gyromoyenne}
\end{equation}

où l'intégrale est  autour de l'orbite de Larmor et l'élément de contour \( \bm{\dd s} \) est le long de \( \bm{v_\perp} \).
Donc en utilisant l'\autoref{eq:moment_magnetique_basique}, l'\autoref{eq:courant_particule_rotation} et l'\autoref{eq:gyromoyenne}, l'\autoref{eq:derive_moment_magnetique_2} peut être réécrite

\begin{equation}
	\frac{\dd \mu}{\dd t} = - \frac{e_j}{B} \frac{\omega_{cj}}{2\pi} \left( \oint \bm{E} \cdot \bm{\dd s} + A \frac{\partial B}{\partial t} \right).
	\label{eq:reecriture_derive_moment}
\end{equation}

Le dernier terme de la parenthèse de l'\autoref{eq:reecriture_derive_moment} est le taux de changement du flux à travers l'aire \( A \), et par la loi de Faraday

\[
	A \frac{\partial B}{\partial t} = - \oint \bm{E} \cdot \bm{\dd s}
\]

de sorte que 

\[
	\frac{\dd \mu}{\dd t} = 0.
\]

Ce qui montre que, dans la limite des petites variations de \(  \bm{B} \), \( \mu \) est une constante.
Autrement dit, \( \mu \) est un invariant adiabatique.

\subsubsection*{Deuxième et troisième invariant adiabatique}

Le moment magnétique est associé à l'orbite de Larmor de la particule.
Quand la particule a un déplacement périodique sur une échelle plus grande, il y a un deuxième invariant \( J \), défini par

\begin{equation}
	J = \oint \bm{v_\parallel} \bm{\dd l}
	\label{eq:def_J}
\end{equation}

où l'intégrale est prise sur une orbite périodique \( x(t) \) et \( \bm{\dd l} = \bm{b} \dd x \) où \( \bm{b} \) est le vecteur unitaire dans la direction du champ magnétique.
En l'absence d'un champ électrique, l'énergie totale, \( W \), de la particule est constante, où

\[
	W = \frac{1}{2} m_j \left( v_\perp^2 + v_\parallel^2\right)
\]

et, en utilisant la définition pour \( \mu \) donnée par l'\autoref{eq:def_mu_2}, \( v_\parallel \) dans l'\autoref{eq:def_J} peut être écrit en terme des constantes du mouvement et du champ magnétique

\[
	v_\parallel = \left( \frac{2}{m_j} \left( W - \mu B \right) \right)^{1/2}.
\]

Le mouvement périodique impliqué dans l'invariant \( J \) peut être sujet à une dérive et cette dérive peut induire un mouvement périodique sur une plus grande échelle.
Le troisième invariant adiabatique est associé à ce mouvement.
Si la vitesse de dérive est \( v_d \), l'invariant est défini par l'intégrale sur le mouvement périodique suivante

\[
	J_3 = \int v_d \dd l.
\]

Il peut être montré que \( J_3 \) est proportionnel au flux magnétique encadré par l’orbite.

\section{Collisions}
\label{sec:collisions}

\end{document}