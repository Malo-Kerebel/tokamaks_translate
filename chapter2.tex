\documentclass[main.tex]{subfiles}

\begin{document}

\setcounter{chapter} {1}
\chapter{Physique des plasmas}

\section{Plasma de tokamak}
\label{sec:plasma_de_tokamak}

Un plasma est un gaz ionisé.
Complètement ioniséc il est composé uniquement d'ions et d'électrons.
Ceux-ci ont beaucoup de propriétés d'un gaz normal.
Par exemple, ils peuvent être décrits par leur densité de particule et leur température.
Cependant, un plasma a deux propriétés caractéristiques.
Premièrement, la densité de charges électriques des deux espèces est si grande qu'une séparation notable mène à des forces de restauration très grandes, et par conséquent, les densités d'ions et d'électrons sont pratiquement égales.
Deuxièmement, un plasma peut conduire un courant électrique du fait d’un mouvement relatif entre les ions et les électrons.
Dans un tokamak, le courant du plasma produit une part importante du champ magnétique.
Là où le courant du plasma traverse le champ magnétique, il engendre une force magnétique qui peut équilibrer le gradient de pression du plasma.

Lorsqu'un plasma est dans un champ magnétique, les particules individuelles sont con\-traintes dans leur déplacement.
Elles sont libres dans la direction parallèle au champ magnétique mais perpendiculairement elles suivent des orbites de Larmor.
Dans un tokamak, les orbites des ions ont un rayon de quelques millimètres et les orbites des électrons sont plus petites d'un facteur égal à la racine carré du ratio de masse électron-ion.
Bien que le comportement précis du plasma soit déterminé par le déplacement des particules individuelles dans le champ électromagnétique local, les contraintes sur le déplacement des particules décrites ci-dessus donnent au plasma des propriétés similaires à un fluide sur des échelles supérieures aux rayons de Larmor.
La majorité de notre compréhension des tokamaks provient de modèles où le plasma est considéré comme un fluide.

La densité de particules dans un tokamak est de l'ordre de \( \sim \SI{e20}{\per\meter\cubed} \) soit environ \( 10^{-5} \) fois la densité de l'atmosphère.
Les plasmas de tokamaks atteignent généralement des températures de quelques \( \SI{}{\kilo\electronvolt} \), ce qui correspond à plusieurs dizaines de millions de Kelvin.
C'est de l'ordre de \( 10^5 \) fois la température de l'atmosphère et par conséquent la pression dans un tokamak est comparable à celle de l'atmosphère.

La force de la pression du plasma allant vers l'extérieur est contrebalancée par le champ magnétique.
Cependant, la densité d'énergie du plasma dans un tokamak est faible comparée à celle du champ magnétique, de l'ordre du pour cent.
Le champ magnétique principal est le champ toroïdal produit par des bobines au dehors du plasma.
Le champ poloïdal produit par le courant toroïdal du plasma  est typiquement dix fois inférieur.

De nombreux prpcessus dans les plasmas sont déterminés par les collisions entre les particules.
Les collisions entre les ions et les électrons donnent lieu à la résistance électrique, qui mène au chauffage ohmique du plasma.
Les collisions causent aussi le transport des particules et de l'énergie, menant à la perte des deux.
Typiquement, le temps de collision des ions et dans l'intervalle de \( \qtyrange[range-units=single,range-phrase=-]{1}{100}{\milli\second} \).
Le temps de collision des électrons est plus petit d'un facteur égal à la racine carrée du ratio de masse entre les ions et les électrons.
Les temps de collision augmentent avec la température, évoluant comme \( T^{3/2} \).
Par conséquent, le chauffage ohmique devient moins efficace à haute température.
D'un autre côté, les pertes collisionnelles du plasma sont réduites.

\begin{table}
	\centering
	\caption{Paramètres typiques des tokamaks}
	\begin{tabular}{l l}
		\hline
		Volume du plasma & \( \qtyrange[range-units=single,range-phrase=-]{1}{100}{\meter\cubed} \)\\
		Masse du plasma totale & \( \qtyrange[range-units=single,range-phrase=-]{e-7}{e-5}{\kilogram} \)\\
		Densité ionique & \( \qtyrange[range-units=single,range-phrase=-]{e19}{e20}{\per\meter\cubed} \)\\
		Température & \( \qtyrange[range-units=single,range-phrase=-]{1}{40}{\kilo\electronvolt} \)\\
		Pression & \( \qtyrange[range-units=single,range-phrase=-]{0.1}{5}{} \) atmosphères\\
		Vitesse thermique des ions & \( \qtyrange[range-units=single,range-phrase=-]{100}{1000}{\kilo\meter\per\second} \)\\
		Vitesse thermique des électrons & \( \qtyrange[range-units=single,range-phrase=-]{0.01}{0.1}{c} \)\\
		Champ magnétique & \( \qtyrange[range-units=single,range-phrase=-]{1}{10}{\tesla} \)\\
		Courant du plasma total & \( \qtyrange[range-units=single,range-phrase=-]{0.1}{7}{\mega\ampere} \)\\
		\hline 
	\end{tabular}
	\label{table:plasma_parametre}
\end{table}

Le comportement basique d'un tokamak échappe majoritairement à notre connaissance.
La perte d'énergie dépasse significativement celle prédite par de simple collisions et cela ne s'explique pas.
Cette anomalie pourrait venir d'instabilités du plasma à petite échelle.

Les plasmas de tokamaks typiques (\autoref{table:plasma_parametre}) sont loin d'être calmes et de nombreuses instabilités macroscopiques sont observées régulièrement.
Dans certains cas le plasma s'adapte à l'instabilité et il n'y a pas de détérioration des performances observée.
Cependant, dans le cas des bien-nommées disruptions de tokamak, les dégâts de l'instabilités sont irréparables.

Ce chapitre donne une introduction à la physique utilisée pour l'analyse et la compréhension des plasmas de tokamaks.

\section{Écrantage de Debye}
\label{sec:ecrantage_debye}

\begin{wrapfigure}{c}{0.2\textwidth}
    \centering
    
    \includegraphics{figures/chapter2/charge_sheet}

    \caption{Couches de charges ioniques et électroniques séparées par une distance \( d \)}
    \label{fig:charge_sheet}
\end{wrapfigure}

La densité de charge électrique des ions et des électrons séparés composant un plasma est assez grande pour que seulement une faible séparation des charges ne soit possible.
Cet effet peut être interprété en imaginant la séparations des ions et des électrons en couches d'une épaisseur \( d \) comme montré dans la \autoref{fig:charge_sheet}.

Si les ions sont simplement ionisés et que la densité des ions et des électrons est \( n \), la charge par unité de surface des couches est \( dne \).En ignorant les facteurs numériques le champ électrique entre les couches est \( \sim dne/\varepsilon_0 \) et la force par unité de surface est 

\begin{equation}
	F \sim \frac{\left( dne \right)^2}{\varepsilon_0}.
	\label{eq:force_unite_surface}
\end{equation}

Dans un tokamak, la densité \( n \) est autour de \( \SI{e20}{\per\meter\cubed} \) et pour cette densité on a 

\[
	F \sim 10^{13} d^2 \SI{}{\newton\per\meter\squared}.
\]

Par exemple, pour une épaisseur de \( \SI{1}{\centi\meter} \), cela donne une force par unité de surface de \( \SI{e9}{\newton\per\meter\squared} \).

Puisque cette force est si grande par la séparation des charges, les densités d'ions et d'électrons sont tenus quasiment égales au travers du plasma. Ce que l'on appelle la quasi-neutralité et, pour le cas général avec des ions de charge \( Z \), la contrainte s'exprime

\begin{equation}
	n_e = \sum_i n_i Z_i
	\label{eq:quasi_neutral}
\end{equation}

où la somme se fait sur les différentes espèces d'ions.

Il est à noter que l'\autoref{eq:quasi_neutral} ne veut pas dire que \( \nabla \cdot \textbf{E} = 0 \), l'\autoref{eq:quasi_neutral} est presque, mais pas totalement, exacte et les petite différences de charges donnent lieu à des champs électriques significatifs.
Donc le champ électrique ne peut pas être déterminé par l'\autoref{eq:quasi_neutral}.
D'un autre côté, étant donné le champ électrique \( \textbf{E} \), la densité de charges \( \rho_c \) est déterminé par \( \nabla \cdot \textbf{E} = \rho_c/\varepsilon_0 \).

On voit dans l'\autoref{eq:force_unite_surface} que la force décroit avec la baisse de la distance de séparation.
L'argument pour justifier la quasi-neutralité est donc invalide à des échelles suffisamment petite.
Il est alors possible d'obtenir une longueur fondamentale caractérisant un plasma en calculant l'épaisseur \( d \) pour laquelle l'énergie interne du plasma pourrait fournir l'énergie pour une séparation complète des ions et des électrons comme montré dans la \autoref{fig:charge_sheet}.
L'énergie requise est \( F d \), la force \( F \) étant donnée par la relation \ref{eq:force_unite_surface} et l'énergie interne est \( \sim dn T \) . En mettant ces deux énergies égales, on obtient la longueur caractéristique \( \lambda_D \) où

\begin{align}
	\lambda_D &= \left( \frac{\varepsilon_0 T}{n e^2} \right)^{1/2} \nonumber \\
	&= 2.35 \times 10^5 \left( \frac{T}{n} \right)^{1/2} \SI{}{\meter}, \quad T \text{en} \SI{}{\kilo\electronvolt}.
	\label{eq:longueur_debye}
\end{align}

Cette longueur est appelée longueur de Debye.
Dans un plasma de tokamak typique \( \lambda_D \) est de l'ordre de \( \qtyrange[range-units=single,range-phrase=-]{e-2}{e-1}{\milli\meter}  \).

Même si la séparation de charge décrite au dessus est énergétiquement possible au-delà de la longueur de Debye, cela ne se produit évidemment pas de façon spontanée dans le corps du plasma parce que la vitesse des particules est aléatoire et la coherence du mouvement des particules requise pour ce déplacement imaginé ne se produit pas.
Une situation où une séparation de charge notable se produit est lorsque le plasma est en contact avec une surface solide.
La séparation de charge apparait alors dans une gaine proche de la surface, et cette gaine a une épaisseur \( \sim \lambda_D \).
La longueur de Debye apparait aussi de façon plus subtile dans le plasma, caractérisant un phénomène que l'on appelle écrantage de Debye.

Si on considère un ion simplement ionisé dans un plasma, le champ électrique de l'ion est alors

\[
	\textbf{E} = \frac{e}{4\pi \varepsilon_0 r^2d}.
\]

Bien que ce champ électrique soit directement associé à cet ion, les autres particules dans le plasma s'ajustent à ce champ, et la distribution résultante de leurs charges écrante la charge de l’ion et modifie le champ électrique effectif.
Les trajectoires des électrons proches de l'ion sont légèrement déplacées vers l'ion et les trajectoires des autres ions sont légèrement déplacées à l'opposé de celui-ci.
L'écrantage apparait pour chaque ion et l'effet inverse aussi pour chaque électron.
Ce comportement est similaire à celui observable dans les électrolytes.
Ce cas a été analysé par Debye et cet écrantage porte donc son nom.

La forme de l'écrantage peut être calculé pour un ion stationnaire en résolvant l'équation de Poisson pour le potentiel \( \phi \).
En géométrie sphérique l'équation est

\begin{equation}
	\frac{1}{r^2} \frac{\dd}{\dd r}r^2 \frac{\dd \phi}{\dd r} = - \frac{\rho_c}{\varepsilon_0},
	\label{eq:poisson}
\end{equation}

où la densité de charge \( \rho_c = \sum_j n_j e_j \). Les ions et les électrons vont chacun avoir une distribution de Boltzmann dans le potentiel, et donc leur densité \( n_j \) sera donnée par

\begin{equation}
	n_j = n_0 \exp - \frac{e_j \phi}{T},
	\label{eq:Boltzmann_densite}
\end{equation}

\( n_0 \) étant la densité d'électrons, et d'ions, à une grande distance de l'ion choisi, où le potentiel est pris égal à \( 0 \).
Donc en prenant l'\autoref{eq:Boltzmann_densite} dans l'\autoref{eq:poisson} et en prenant des ions simplement ionisés

\begin{equation}
	\frac{1}{r^2} \frac{\dd}{\dd r}r^2 \frac{\dd \phi}{\dd r} = - \frac{n_0 e}{\varepsilon_0} \left( e^{-e\phi / T} - e^{e\phi/T} \right).
	\label{eq:poisson_boltzmann}
\end{equation}

Il sera trouvé que \( e\phi/T \ll 1\) et en anticipant ce résultat, l'\autoref{eq:poisson_boltzmann} devient

\begin{equation}
	\frac{1}{r^2} \frac{\dd}{\dd r}r^2 \frac{\dd \phi}{\dd r} = \frac{2 n_0 e^2}{\varepsilon_0 T} \phi.
	\label{eq:poisson_ephi_T_1}
\end{equation}

En prenant le produit \( \left( r \phi \right) \) en variable, l'\autoref{eq:poisson_ephi_T_1} peut être écrite

\[
	\frac{\dd^2}{\dd r^2} \left( r \phi \right) = \frac{2}{\lambda_D^2} \left( r \phi \right)  
\]

et la solution qui décroit pour un grand \( r \) est

\begin{equation}
	\phi = \alpha \frac{1}{r} e^{-\sqrt{2} r / \lambda_D}
	\label{eq:solution_phi_alpha}
\end{equation}

où \( \alpha \) est une constante. Puisque \( \phi = e/4\pi \varepsilon_0 r \) lorsque \( r \rightarrow 0 \), l'\autoref{eq:solution_phi_alpha} devient

\begin{equation}
	\phi = \frac{e}{4\pi\varepsilon_0 r} e^{-\sqrt{2} r / \lambda_D}
	\label{eq:solution_phi}
\end{equation}

\vspace{2\baselineskip}
\begin{wrapfigure}{c}{0.3\textwidth}
    \centering
    
    \includegraphics{figures/chapter2/potential}

    \caption{Tracés du potentiel écranté et non-écranté autour d'un ion.}
    \label{fig:potential}
\end{wrapfigure}

Cette équation décrit l'écrantage de Debye.
Il est visible que le potentiel \( e/4\pi \varepsilon_0 r \) de l'ion est écranté selon un facteur exponentiel, la longueur caractéristique de l'écrantage étant la longueur de Debye
Les tracés des potentiels écrantés et non-écrantés sont donnés dans la \autoref{fig:potential}.

Avec l'\autoref{eq:solution_phi}, l'approximation \( e\phi/T \ll 1 \) utilisée pour le calcul implique

\begin{equation}
	r \gg \frac{e^2}{4\pi \varepsilon_0 T}.
	\label{eq:inegalite_r}
\end{equation}

L'aspect physique de ce pré-requis peut être visualisé en observant que la plus petite valeur pertinente de \( r \) est de l'ordre de la séparation entre les particules, qui est de \( 1/n^{1/3} \).
En utilisant cette valeur pour \( r \), l'inégalité \ref{eq:inegalite_r} devient

\[
	n \lambda_D^3 \gg \left(\frac{1}{4\pi} \right)^{3/2}
\]

et cette approximation est valide s'il y a beaucoup de particules dans un cube de côté \( \lambda_D \), autrement dit

\begin{equation}
	n \lambda_D^3 \gg 1.
	\label{eq:inegalite_nlambda}
\end{equation}

Dans un plasma de tokamak les valeurs typiques sont de \( n \simeq \SI{e20}{\per\meter\cubed} \) et \( \lambda_D \simeq \SI{e-4}{\meter} \).
Donc \( n \lambda_D^3 \simeq 10^8\) et l'inégalité \ref{eq:inegalite_nlambda} est satisfaite.
La valeur \( n \lambda_D^3 \) est appelé le paramètre du plasma.

L'hypothèse que la particule de test soit stationnaire a donné un compte rendu simplifié de l'écrantage.
En réalité, chaque particule a une vitesse relativement à la vitesse moyenne des particules de fond.
Quand la particule de test a une vitesse comparable à la vitesse thermique des particules d'écrantage, il y a un délai dans la formation de la charge d'écrantage et seul un écrantage partiel se produit.
Dans la limite de particules de test très rapides, la réponse de l'écrantage est négligeable et le potentiel associé à la charge de test approche le potentiel non écranté de la forme \( e/4\pi \varepsilon_0 r \).

Le comportement global est tel que chaque particule génère sa propre charge d'écrantage lorsqu'elle se déplace à travers le plasma et cela donne un champ électrique fluctuant.
Les particules reçoivent aussi de l'énergie du champ électrique et l'équilibre entre ces processus de forçage et d'amortissement donne le spectre thermique des fluctuations électrostatiques

\section{Fréquence plasma}
\label{sec:frequence_plasma}

\vspace{2\baselineskip}
\begin{wrapfigure}{c}{0.4\textwidth}
    \centering
    % Réduction du retrait interne du wrap
    \vspace{-10pt}

    \begin{subfigure}[b]{0.48\textwidth}
        \centering
        \includegraphics[width=\textwidth]{figures/chapter2/displaced_electrons.pdf}
        \caption{}
        \label{subfig:displaced_electrons}
    \end{subfigure}
    
    \begin{subfigure}[b]{0.48\textwidth}
        \centering
        \includegraphics[width=\textwidth]{{figures/chapter2/displaced_electrons_2.pdf}}
        \caption{}
        \label{subfig:displaced_electrons_2}
    \end{subfigure}

    \caption{(a) Les électrons déplacés créent une force de restauration et une accélération, (b) après un demi cycle, les charges sont interchangées}
    \vspace{-10pt}
    \label{fig:displaced_electrons}
\end{wrapfigure}

Dans la \autoref{sec:ecrantage_debye}, il a été montré qu'un plasma a une longueur caractéristique, la longueur de Debye.
Un plasma a aussi une fréquence caractéristique.
Cela peut être compris en considérant une couche déplacée d'électrons comme montré dans la \autoref{subfig:displaced_electrons}.
Il est admis qu'à cause de la plus grande masse des ions, ils sont stationnaires et ont une densité uniforme.
Les charges positives dans la figure représentent les défauts d'électrons dans les couches desquelles les électrons ont été déplacés.
Le champ électrique résultant est montré dans la figure.
Les électrons sont accélérés par le champ électrique et se déplacent pour annuler la charge positive.
Au moment de l'annulation, la quantité de mouvement des électrons est maximum et cela permet aux électrons de récréer la séparation des charges, maintenant dans la phase opposée.
Ce processus est répété et le déplacement des électrons en résultants constitue les oscillations du plasma avec la fréquence plasma caractéristique.

La fréquence plasma peut être calculée en utilisant les équations fluides.
Pour un plasma froid, l'équation du mouvement pour le fluide électron est

\begin{equation}
	m_e \frac{\partial v}{\partial t} = - e \textbf{E}
	\label{eq:motion_electron}
\end{equation}

et en assumant une oerturbation de la densité faible \( \tilde{n} \), le champ électrique est donné par

\begin{equation}
	\nabla \cdot \textbf{E} = - \frac{e \tilde{n}}{\varepsilon_0}
	\label{eq:E_perturbation_faible}
\end{equation}

Donc en prenant la divergeance de l'\autoref{eq:motion_electron} et en éliminant \( \nabla \cdot \textbf{E} \) en utilisant l'\autoref{eq:E_perturbation_faible}

\begin{equation}
	m_e \frac{\partial }{\partial t} \nabla \cdot v = \frac{e^2}{\varepsilon_0} \tilde{n}.
	\label{eq:motion_sans_E}
\end{equation}

L'équation de continuité est

\[
	\frac{\partial n}{\partial t} = - \nabla \cdot (nv)
\]

et la forme linéarisée est

\begin{equation}
	\frac{\partial \tilde{n}}{\partial t} = -n \nabla \cdot v.
	\label{eq:forme_linearise}
\end{equation}

En substituant \( \nabla \cdot v \) de l'\autoref{eq:forme_linearise} dans l'\autoref{eq:motion_sans_E} donne l'équation nécessaire pour les oscillations du plasma

\[
	\frac{\partial^2 \tilde{n}}{\partial t^2} = - \omega_p^2 \tilde{n}
\]

où \( \omega_p \) est la fréquence plasma d'un électron

\begin{equation}
	\omega_p = \left( \frac{n e^2}{\varepsilon_0 m_e} \right)^{1/2}.
	\label{eq:frequence_plasma}
\end{equation}

Une fréquence similaire est associée à l'oscillation des ions, la fréquence plasma d'un ion étant

\begin{equation}
	\omega_{pi} = \left( \frac{n e_i^2}{\varepsilon_0 m_i} \right)^{1/2}
	\label{eq:frequence_plasma_ion}
\end{equation}

où \( e_i \) est la charge d'un ion.

La \color{red}{Section 2.25 (lien à rajouter quand elle sera faite)}\color{black} donne un compte rendu plus général des oscillations du plasma et il y est montré que l'\autoref{eq:frequence_plasma} donne la fréquence des ondes du plasma d'électron dans la limite où la longueur d'onde est beaucoup plus grande que la longueur de Debye.

En utilisant des valeurs numériques dans l'\autoref{eq:frequence_plasma}

\[
	\omega_p = 56.4 n^{1/2} \SI{}{\per\second}.
\]

Pour un tokamak cette fréquence est très élevée. Par exemple, une densité \( n = \SI{e20}{\per\meter\cubed} \) donne \( \omega_p = \SI{5.6e11}{\per\second} \) avec un temps caractéristique \( \omega_p^{-1} \) inférieur à un centième de nanoseconde.

\section{Orbites de Larmor}
\label{sec:orbites_larmor}

L'équation du mouvement d'une particule de masse \( m_j \) et de charge \( e_j \) dans un champ magnétique est

\[
	m_j \frac{\dd v}{\dd t} = e_j v \times \textbf{B}
\]

Si le champ magnétique est uniforme et dans la direction \( z \), les composantes de l'équation sont

\begin{align}
	\frac{\dd v_x}{\dd t} = \omegaç{cj} v_y, \quad \frac{\dd v_u}{\dd t} = -\omega_{cj} v_x, \label{eq:composantes_x_y}\\
	\frac{\dd v_z}{\dd t} = 0 \label{eq:composante_z}
\end{align}

où

\[
	\omega_{cj} = \frac{e_j \textbf{B}}{m_j}
\]

est la fréquence cyclotron.
Par l'\autoref{eq:composante_z}, on a que la vitesse dans la direction du champ magnétique est constante.
La séparation des variables des équations \ref{eq:composantes_x_y} mène à

\[
	\frac{\dd^2 v_x}{\dd t^2} = -\omega_{cj}^2 v_x, \quad \frac{\dd^2 v_y}{\dd t^2} = -\omega_{cj}^2 v_y,
\]

et les solutions à ces équations peuvent être écrite

\begin{equation}
	v_x = v_\perp \sin \omega_{cj} t, \quad v_y = v_\perp \cos \omega_{cj} t.
	\label{eq:solutions_vx_vy}
\end{equation}

En utilisant \( v_x = \dd x / \dd t \) et \( v_y = \dd y / \dd t \), les équations \ref{eq:solutions_vx_vy} peuvent être intégrées pour avoir

\begin{equation}
	x = - \rho_j \cos \omega_{cj} t, \quad y = \rho_j \sin \omega_{cj} t,
	\label{eq:solution_x_y}
\end{equation}

où

\[
	\rho_j = \frac{v_\perp}{\omega_{cj}} = \frac{m_j v_\perp}{e_j \textbf{\textbf{B}}}
\]

est le rayon de Larmor.
Donc la particule a une orbite hélicoïdal composée des mouvements circulaires des équations \ref{eq:solution_x_y} et une vitesse constante dans la direction du champ magnétique.

Une particule ayant l'énergie thermique moyenne dans le plan perpendiculaire au champ magnétique a \( v_\perp^2 = 2 v_{Tj}^2 \) où \( \frac{1}{2} m_j v_{Tj}^2 = \frac{1}{2} T_j \).
Le facteur \( 2 \) apparait à cause des deux degrés de liberté en jeu.
Donc pour une particule thermique

\begin{equation}
	\rho_j = \sqrt{2} \frac{m_j v_{Tj}}{\vert e_j \vert \textbf{B}}
	\label{eq:rayon_larmor}
\end{equation}

En remplaçant par less valeurs numériques de la charge de l'électron \( e = \SI{1.605e-19}{\coulomb} \), de la masse de l'électron \( m_e = \SI{9.11e-31}{\kilogram} \) et de la masse du proton \( m_p = \SI{1.673e-27}{\kilogram} \),ainsi que par \( v_{Tj} = 1.27 \times 10^{-8} \left( T_j / m_j \right)^{1/2} \SI{}{\meter\per\second} \) avec la température exprimée en \( \SI {}{\kilo\electronvolt}\), donne

\[
	\left. \begin{array}{l l}
		\text{Électron} & \left\vert \omega_{ce} \right\vert = 1.76 \times 10^{11} B \SI{}{\per\second},\\
		 & \rho_e = 1.07 \times 10^{-4} T_e^{1/2}/ B \SI{}{\meter},\\
		 \text{Proton} & \omega_{cp} = 9.58 \times 10^{7} B \SI{}{\per\second},\\
		  & \rho_p = 4.57\times 10^{-3} T_p^{1/2}/B \SI{}{\meter},\\
		  \text{Particule } j & \omega_{cj} = 9.58 \times 10^{7} \left( Z/A\right) B \SI{}{\per\second},\\
		  & \rho_j = 4.57\times 10^{-3} \left(A^{1/2}/Z\right) T_j^{1/2}/B \SI{}{\meter},
	\end{array} \right\rbrace T \text{in} \SI{}{\kilo\electronvolt},
\]

où \( Z \) et \( A \) sont la charge et le ombre de masse de la particule \( j \).

\begin{table}
	\centering
	\begin{tabular}{llll}
\hline 
 & \multicolumn{3}{l}{Champ magnétique}\\ 
\hline 
Fréquence & \( 1 \) Tesla & \( 3 \) Teslas & \( 5 \) Teslas \\ 
\hline 
\(  \left\vert \omega_{ce} \right\vert (\SI{}{\per\second}) \) & \( \SI{1.76e11}{} \) & \( \SI{5.28e11}{} \) & \( \SI{8.79e11}{} \) \\ 
\( \omega_{cp} (\SI{}{\per\second}) \) & \( \SI{9.58e7}{} \) & \( \SI{2.87e8}{} \) & \( \SI{4.79e8}{} \) \\ 
\( f_{ce} ( \SI{}{\giga\hertz}) \) & \( 28 \) & \( 84 \) & \( 140 \) \\ 
\( f_{cp} ( \SI{}{\giga\hertz}) \) & \( 15 \) & \( 46 \) & \( 76 \) \\ 
\hline 
\end{tabular} 
\caption{Valeur de la fréquence cyclotron \( \omega_c \) et \( f_c( =\vert \omega_c\vert/2\pi) \) pour les électrons et les protons.}
\label{tab:frequence_larmor}
\end{table}

\begin{table}
	\centering
	\begin{tabular}{l l l l l l}
	\hline
	 &Rayon de & Température & & & \\
	 \hline
	 \( B \) & Larmor & \( \SI{10}{\electronvolt} \) & \( \SI{100}{\electronvolt} \) & \( \SI{1}{\kilo\electronvolt} \) & \( \SI{10}{\kilo\electronvolt} \)\\
	 \hline
	 \( 3 \) & \( \rho_e \) & \( \SI{0.003}{\milli\meter} \) & \( \SI{0.011}{\milli\meter} \) & \( \SI{0.035}{\milli\meter} \) & \( \SI{0.11}{\milli\meter} \)\\
	 Teslas  & \( \rho_p \) & \( \SI{0.15}{\milli\meter} \) & \( \SI{0.48}{\milli\meter} \) & \( \SI{1.5}{\milli\meter} \) & \( \SI{4.8}{\milli\meter} \)\\
	 \( 5 \) & \( \rho_e \) & \( \SI{0.002}{\milli\meter} \) & \( \SI{0.007}{\milli\meter} \) & \( \SI{0.021}{\milli\meter} \) & \( \SI{0.67}{\milli\meter} \)\\
	 Teslas  & \( \rho_p \) & \( \SI{0.09}{\milli\meter} \) & \( \SI{0.29}{\milli\meter} \) & \( \SI{0.91}{\milli\meter} \) & \( \SI{2.9}{\milli\meter} \)\\
	\end{tabular}
	\caption{Valeurs du rayon de Larmor, \( \rho \), pour les électrons et les protons ayant une vitesse thermique.}
	\label{tab:rayon_larmor}
\end{table}

Le \autoref{tab:frequence_larmor} et le \autoref{tab:rayon_larmor} donent quelques valeurs de \( \omega_c \) et de \( \rho \).
Il faut noter que le rayon deLarmor d'une particule thermique est parfois défini sans le facteur \( \sqrt{2} \) de l'\autoref{eq:rayon_larmor}. 

\end{document}