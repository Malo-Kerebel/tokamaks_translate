\documentclass{standalone}
\usepackage{pgfplots}
\pgfplotsset{compat=1.18} % Assure la compatibilité avec la dernière version de pgfplots
\usepackage[utf8]{inputenc}
\usepackage{siunitx}
\usepackage[french]{babel}
\usetikzlibrary{patterns}

\begin{document}

\begin{tikzpicture}
    \begin{axis}[
    	width=7.5cm, % Largeur de la figure
        height=5cm, % Hauteur de la figure
        ylabel={\( \hat{n} \tau_E \hat{T} \) (\( \SI{}{\second\kilo\electronvolt\per\meter\cubed} \))},
        xmin=1954, xmax=1998,
        ymin=5e13, ymax=5e22,
        ymode=log,
        xtick distance=5,
        ytick={1e14, 1e15, 1e16, 1e17, 1e18, 1e19, 1e20, 1e21, 1e22},
        grid=both,
        grid style={line width=.1pt, draw=gray!10},
        major grid style={line width=.2pt,draw=gray!50},
        legend pos=north west,
        y label style={font=\small},
        tick label style={font=\scriptsize},
        xticklabel style={/pgf/number format/set thousands separator={}}
    ]

    % Record par année
    \addplot[
        only marks,
        mark=o,
        color=black,
        mark size=2pt
    ]
    coordinates {
        (1955, 1.5e14)
        (1960,  1.2e15)
        (1965, 2e16)
        (1970, 1e17)
        (1975, 7e17)
        (1980, 9e18)
        (1985, 6e19)
        (1990, 9e20)
        (1995, 1e21)
    };
    
    % Zone hachurée au-dessus de 5e21
    \fill[
        pattern=north east lines, % Style de hachures
        pattern color=gray!50, % Couleur des hachures
        draw=gray, % Couleur du contour
        opacity=0.5 % Transparence (optionnel)
    ]
    (1954, 5e21) -- (2022, 5e21) -- (2022, 5e22) -- (1954, 5e22) -- cycle;

    % Texte "Réacteur" dans la zone hachurée
    \node[anchor=south, font=\bfseries] at (1976, 4e21) {Réacteur};

    \end{axis}
\end{tikzpicture}

\end{document}
