\documentclass{standalone}
\usepackage{pgfplots}
\pgfplotsset{compat=1.18} % Assure la compatibilité avec la dernière version de pgfplots
\usepackage[utf8]{inputenc}
\usepackage[french]{babel} % Optionnel : pour la langue française
\usepackage{siunitx}
\usepackage{amsmath}
\begin{document}

\begin{tikzpicture}
    \begin{axis}[
        xlabel={\( \dfrac{\varepsilon}{T} \)},
        xmin=-0, xmax=15,
        ymin=0, ymax=9,
        xtick={0, 5, 10, 15},
        ytick=\empty,
        grid=both,
        grid style={line width=.1pt, draw=gray!10},
        major grid style={line width=.2pt,draw=gray!50},
        tick label style={font=\scriptsize},
        legend style={font=\scriptsize}
    ]

    % Courbe pour D-T (approximation)
    \addplot[
        domain=0:15,
        samples=375,
        color=black,
        thick
    ]
    table [header=true] {integrand/integrand.txt};
    \node[anchor=north, font=\scriptsize] at (5.9, 8.25) {\(\sigma \varepsilon \exp{\left( -\frac{\mu}{m_\text{D}} \frac{\varepsilon}{T} \right)} \)};

    % Courbe de epsilon*exponentielle
    \addplot[
        domain=0:15,
        samples=200,
        color=black!70,
        thick
    ]
    ttable [header=true] {integrand/eps_exp.txt};
    \node[anchor=north, font=\scriptsize] at (2.25, 7.1) {\(\varepsilon \exp{ \left(-\frac{\mu}{m_\text{D}} \frac{\varepsilon}{T} \right) } \)};

    % Courbe de la section efficace
    \addplot[
        domain=0:15,
        samples=375,
        color=black!40,
        thick
    ]
    table [header=true] {integrand/sigma.txt};
    \node[anchor=north, font=\scriptsize] at (10.75, 5.6) {\(\sigma \)};

    \end{axis}
\end{tikzpicture}

\end{document}
