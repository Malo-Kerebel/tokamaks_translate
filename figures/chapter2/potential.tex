\documentclass{standalone}
\usepackage{pgfplots}
\pgfplotsset{compat=1.18} % Assure la compatibilité avec la dernière version de pgfplots
\usepackage[utf8]{inputenc}
\usepackage[french]{babel} % Optionnel : pour la langue française
\usepackage{siunitx}
\begin{document}

\begin{tikzpicture}
    \begin{axis}[
        xlabel={\( r/\lambda_D \)},
        ylabel={\( \phi \) },
        xmin=0, xmax=2.05,
        ymin=0, ymax=1e-4,
        xtick={0, 1, 2},
        ytick=\empty,
        scaled x ticks=false,
        grid=both,
        grid style={line width=.1pt, draw=gray!10},
        major grid style={line width=.2pt,draw=gray!50},
        legend pos=north west,
        tick label style={font=\scriptsize},
        legend style={font=\scriptsize}
    ]

    % Courbe pour phi écranté
    \addplot[
        domain=0:2.05,
        color=black,
        thick
    ]
    table [header=false] {shielding/potential_shield.dat};
    \node[anchor=east] at (0.7, 1.5e-5) {Écranté};
    
    % Courbe pour phi sans écrantage
    \addplot[
        domain=0:2.05,
        color=black!70,
        thick
    ]
    table [header=false] {shielding/potential.dat};
    \node[anchor=south west] at (0.9, 2.9e-5) {Non écranté};
    \end{axis}
\end{tikzpicture}

\end{document}
