\documentclass{standalone}

\usepackage{pgfplots}
\pgfplotsset{compat=1.18} % Assure la compatibilité avec la dernière version de pgfplots
\usepackage[utf8]{inputenc}
\usepackage[french]{babel} % Optionnel : pour la langue française
\usepackage{siunitx}
\usepackage{amsmath}

\usepackage{tikz}

\usetikzlibrary{patterns}
\usetikzlibrary{arrows.meta}

\usetikzlibrary{decorations.pathreplacing, decorations.markings}
\tikzset{
    midarrow/.style={
        decoration={markings, mark=at position 0.5 with {\arrow{Latex}}},
        postaction={decorate}
    }
}

\begin{document}

\begin{tikzpicture}

	\pgfmathsetmacro{\angle}{30}
	\pgfmathsetmacro{\lengthaxis}{4}
	\pgfmathsetmacro{\yendaxis}{\lengthaxis*-sin(\angle)}
	
	\pgfmathsetmacro{\lengthdroit}{0.15}

	\draw (0, -\lengthaxis) -- (0, 1);

	\draw[ -{Latex} ] (0, 0) -- (\lengthaxis, \yendaxis) node[ midway, anchor=south ] {\( R_c \)};
	\pgfmathsetmacro{\ydroit}{\lengthdroit*-sin(\angle)}
	\draw (\lengthdroit, \ydroit) -- (\lengthdroit, \ydroit-\lengthdroit);
	\draw (0, -\lengthdroit) -- (\lengthdroit, \ydroit-\lengthdroit);
	
	\draw[midarrow] (\lengthaxis,\yendaxis+0.5) to[ out=-150, in=-30 ] node[midway, anchor=south] {\( B \)} (-2, -1.5);
	
	\pgfmathsetmacro{\xvd}{3.5}
	\pgfmathsetmacro{\yvd}{\xvd*-sin(\angle)}
	\draw[ -{Latex} ] (\xvd, \yvd) -- (\xvd, \yvd-1) node[ anchor=west ] {\( v_d \)};
	\pgfmathsetmacro{\ydroit}{(\xvd*-sin(\angle)+\lengthdroit*-sin(\angle)}
	\draw (\xvd+\lengthdroit, \ydroit) -- (\xvd+\lengthdroit, \ydroit-\lengthdroit);
	\draw (\xvd, \yvd-\lengthdroit) -- (\xvd+\lengthdroit, \ydroit-\lengthdroit);
	
	\draw[dash pattern=on 2pt off 2pt] (\xvd, \yvd-1) to[ out=-150, in=-50 ] (-2, -1.5);
	
	\def\amplitude{3}
	\def\nbperiods{0.5} 
	
	\draw[decorate, decoration={snake, amplitude=\amplitude, segment length=10pt/\nbperiods}]
(-2, -1.5) to[out=-50, in=-150] node[ pos=0.75, anchor=north ] {\tiny  \shortstack {\\ \\  Trajectoire de l'ion}} (\xvd, \yvd-0.85);

	\def\xfleche{1.25}
	\def\yfleche{-3.2}
	\draw[ -{Latex} ] (\xfleche, \yfleche) -- (\xfleche + 0.0001, \yfleche);

\end{tikzpicture}

\end{document}