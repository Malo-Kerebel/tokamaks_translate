\documentclass{standalone}

\usepackage{pgfplots}
\pgfplotsset{compat=1.18} % Assure la compatibilité avec la dernière version de pgfplots
\usepackage[utf8]{inputenc}
\usepackage[french]{babel} % Optionnel : pour la langue française
\usepackage{siunitx}
\usepackage{amsmath}
\usepackage{bm}

\usepackage{tikz}
\usetikzlibrary{patterns}
\usetikzlibrary{arrows.meta}

\usetikzlibrary{decorations.markings}

\tikzset{
    midarrow/.style={
        decoration={markings, mark=at position 0.65 with {\arrow{Latex}}},
        postaction={decorate}
    }
}

\begin{document}

\begin{tikzpicture}

\def\offsetx{1}
\def\length{2}
\def\amplitude{1}
\def\decay{0.75}

\def\offsetsin{\length/10}
\def\nperiode{1.5}
\def\phase{-30}

\begin{axis}[ 
	  xtick=\empty,
	  ytick=\empty, 
	  axis lines=none]

\addplot[
        domain=\offsetx:\length+\offsetx,
        samples=100,
        color=black,
        thick
    ]
    {\amplitude*exp(-\decay*\x)};
    
\addplot[
        domain=\offsetx:\length+\offsetx,
        samples=100,
        color=black,
        thick
    ]
    {-\amplitude*exp(-\decay*\x)};
    
\addplot[
        domain=\offsetx+\offsetsin:\length/2+\offsetx+\offsetsin,
        samples=100,
        color=black,
        thick,
        -{Latex}
    ]
    {\amplitude*exp(-\decay*\x)*sin(\nperiode*\x/\length*720
     - \nperiode*\offsetx/\length*720
      - \nperiode*\offsetsin/\length*720
      +\phase};
      
\addplot[
    domain=\offsetx:\length+\offsetx,
    samples=100,
    color=black,
    thick,
    midarrow
    ] {0};
    
\pgfmathsetmacro{\xinter}{\offsetx + \offsetsin
      -4*\phase/\nperiode*\length/720};
\pgfmathsetmacro{\yinter}{\amplitude*exp(-\decay*\xinter)};

\pgfmathsetmacro{\derive}{\amplitude*\decay*exp(-\decay*\xinter)};

\def\vectorlength{0.25}

\def\offsetalpha{0.25}
\def\deltax{0.00125}
\draw (\xinter-\offsetalpha, \yinter) -- (\xinter, \yinter);

\pgfmathsetmacro{\xalpha}{\xinter-\offsetalpha/2+\deltax}
\pgfmathsetmacro{\yalpha}{\amplitude*exp(-\decay*\xalpha)}
\draw (\xinter-\offsetalpha/2, \yinter) to[ out=90, in=-30]  (\xalpha, \yalpha);

\draw[ {latex}- ] (\xinter/4+3*\xalpha/4, 3*\yinter/4+\yalpha/4) -- (\xinter/4+3*\xalpha/4+0.125, 3*\yinter/4+\yalpha/4+0.125) node[anchor=west, align=left] {\tiny \( \alpha \)};

%% tengeante
%\draw[ -{latex}, blue ] (\xinter-\vectorlength, \yinter+\vectorlength*\derive) -- (\xinter+\vectorlength, \yinter-\vectorlength*\derive);

% Normale
\draw[ -{latex} ]  (\xinter, \yinter) -- (\xinter- \vectorlength*\derive,  \yinter-\vectorlength*\derive);

\draw[ {latex}- ] (\xinter- \vectorlength/2*\derive,  \yinter-\vectorlength/2*\derive) --  (\xinter+0.25, \yinter+0.05) node[anchor=west, align=left] {\tiny Force \( e_j \bm{v} \times \bm{B} \)};

\node[ align=left ] at (0.85*\length+\offsetx, 0) {\tiny \shortstack{ Mouvement du \\ centre guide}};

\end{axis}

\end{tikzpicture}

\end{document}