\documentclass[10pt, twoside]{book}

% Packages utiles
\usepackage[utf8]{inputenc}
\usepackage[T2A,T1]{fontenc}
\usepackage[french]{babel}
\usepackage{amsmath}
\usepackage{ amssymb }
\usepackage{titlesec}
\usepackage{tocloft}
\usepackage{multicol}
\usepackage{graphicx}
\usepackage{xcolor}
\usepackage{lipsum}
\usepackage{float}
\usepackage{bm}

\usepackage{hyperref}

\usepackage{ mathrsfs }

\usepackage{standalone}

%SI
\usepackage{siunitx}
\sisetup{locale = FR} % Pour adapter les séparateurs décimaux aux conventions françaises


% TIKZ
\usepackage{tikz}
\usetikzlibrary{shapes.geometric}
\usetikzlibrary{tikzmark,arrows.meta,calc}

% Configuration de la mise en page
\usepackage[
    inner=5cm,   % Marge intérieure (côté reliure)
    outer=2cm,     % Marge extérieure
    top=2cm,       % Marge supérieure
    bottom=2cm,    % Marge inférieure
    twoside,       % Active le mode "double page" (paires/impaires)
    bindingoffset=0.5cm % Décalage pour la reliure (optionnel)
]{geometry}

% Captions et placement des figures
\usepackage[font=small,labelfont=bf,width=0.5\textwidth]{caption}
\usepackage{wrapfig}
\usepackage{subcaption}
\usepackage[bottom]{footmisc}

\usepackage{placeins}

% Style de numérotation
\numberwithin{equation}{section}
\numberwithin{figure}{section}
\renewcommand{\theequation}{\thesection.\arabic{equation}}
\renewcommand{\thefigure}{\thesection.\arabic{figure}}

\let\cleardoublepage\clearpage

% Macros

\newcommand{\dd}{\text{d}}
\newcommand{\sigmav}{\ensuremath{\left\langle \sigma v \right\rangle}}



\usepackage{subfiles}

\begin{document}

% Texte avant la table des matières
\frontmatter
\subfile{avant_propos}

% Table des matières
%\begin{multicols}{2}
    \tableofcontents
%\end{multicols}
\clearpage

\chapter*{Unités et symboles}
\addcontentsline{toc}{chapter}{Unités et symboles}

Le système d'unit" est le système SI.
Suivant la convention générallement accepté dans le domaine, les températures sont écrites soit en joules soit en électron-volts (ou \( \SI{}{\kilo\electronvolt} \)).
Donc à la place du conventionnel \( k_B T^\circ \) (où \( k_B \) est la constante de Boltzmann et \( T^\circ \) en degré Kelvin) on écrit \( T (\SI{}{joules}) \) de sorte que \( T^\circ = T (\SI{}{joules}) / \SI{1.381e-23}{} \).
La temérature en électron-volt est définie par la différence de potentiel e volts à travers laquelle un électron doit tomber pour obtenir une énergie \( T \)n c'est à dire \( T (\SI{}{\electronvolt}) = T (\SI{}{joules}) / e \) où \( e \) est la charge électronique.
Donc \( T (\SI{}{\electronvolt}) = T (\SI{}{joules}) / \SI{1.602e-19}{} \).
Lorsque la température donnée est en électron-volt, ce sera dit explicitement.

Pour éviter une continuelle redéfinition des symboles fréquemment utilisés, une liste de ces symbole est données en \color{red}{Section 14.16 (lien à ajouter lorsqu'elle sera fait)}\color{black}.

% Corps du texte
\mainmatter

\subfile{chapter1}

%\include{chapitre2}
% Ajoute d'autres chapitres ici

\end{document}